\subsection{酸化開裂反応(オゾン分解)}
\begin{screen}
  過マンガン酸カリウム\ce{KMnO4}やオゾン\ce{O3}のような強い酸化剤を
  \chemfig{C=C}結合に作用させるとケトンやアルデヒド,カルボン酸を生成する。
\begin{gather*}
  \ce{
    \chemfig{C(-[3]R_1)(-[5]R_2) = C(-[1]R_3)(-[-1]R_4)}
    -> \chemfig{C(-[3]R_1)(-[5]R_2) = O}
    +\chemfig{O = C(-[1]R_3)(-[-1]R_4)}}
\end{gather*}
ただし$\mathrm{R_1,R_2,R_3,R_4}$は炭化水素基。
\end{screen}

$\mathrm{R}$のいずれかがHのときはアルデヒドが生成物にあり,
\ce{KMnO4}の場合は強い酸化剤であるためカルボン酸になる。
\eg{
  \ce{
    \chemfig{C(-[3]CH_3)(-[5]CH_3) = C(-[1]CH_3)(-[-1]H)}
    -> \chemfig{C(-[3]CH_3)(-[5]CH_3) = O}
    +\chemfig{O = C(-[1]CH_3)(-[-1]OH)}
  }
}
