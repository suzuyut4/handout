\subsection{ハロゲン化(置換)}
\eg{\benzene \:\ce{+ Cl2 ->C[Fe] \phenyl Cl + HCl}}

\subsection{ニトロ化}
混酸(濃硝酸と濃硫酸の混合物)を用いて\ce{-NO2}に置換する。
濃硫酸は触媒としてはたらく。
\eg{\benzene \:\ce{+ HNO3 ->[50\text{\sim} 60\degreeC]C[H2SO4] \phenyl NO2 + H2O}}
$80\degreeC$ 以上ではベンゼンが蒸発,あるいはニトロ化が進んでジニトロベンゼンが生成することがある。

\subsection{スルホン化}
濃硫酸を用いて\ce{-SO3H}に置換する。
\eg{\benzene \:\ce{+ H2SO4 ->[60\degreeC] \phenyl SO3H + H2O}}

\subsection{ジアゾ化}
亜硝酸\ce{HNO2}を用いて\ce{-NH2}を酸性下で反応させ,
化合物中に\ce{N2^+}(\ce{-N+#N-})をつくる。
\eg{
  \ce{
    \phenyl \chemfig{N(-[1]H)(-[-1]H)} + HO-N=O + H+ -> \phenyl N2^+ + 2 H2O
  }
}
\ce{\phenyl N2^+}をベンゼンジアゾニウムイオンといい,
$5\degreeC$ 以上では水と反応してフェノールを生成する。
\eeg{
  \ce{
    \phenyl N2^+ + H2O -> \phenyl OH + N2 + H+
  }
}

ただし,実際は亜硝酸が不安定であるため亜硝酸ナトリウムを用いる。
亜硝酸ナトリウムは酸性下では弱酸遊離によって亜硝酸となって反応する。
\eeg{\ce{NaNO2 + HCl -> HNO2 + NaCl}}
\eg{
  \ce{
    \phenyl NH2 + NaNO2 + 2HCl -> \phenyl N2Cl + NaCl + 2H2O
  }
}

\subsection{カップリング}
ジアゾカップリングともいう。
$p-$\!フェニルアゾフェノール($p-$\!ヒドロキシアゾベンゼン)
\eg{
  \ce{
    \phenyl N2Cl + \phenyl ONa -> \phenyl N = N \para OH + NaCl
  }
}
\eg{
  \ce{
    \phenyl N2Cl + \phenyl OH + NaOH -> \phenyl N = N \para OH + NaCl + H2O
  }
}
\subsection{アセチル化}
アセチル基\chemfig{\ketone \ce{CH3}}をつくる反応。
無水酢酸のような酸無水物を用いる。
\eeg{
  \begin{tabular}{c}
    \ce{\chemfig{CH_3 - C(=[2]O) - OH}} \\
    \\
    \ce{\chemfig{CH_3 - C(=[-2]O) - OH}}
  \end{tabular}
  \ce{
    -> \chemfig{O(-[3]C(=[2]O)(-[4]CH_3))(-[-3]C(=[-2]O)(-[4]CH_3))} + H2O 
  }
}
\eg{
  \ce{
    \phenyl NH2 + (CH3CO)2O -> \phenyl \chemfig{N(-[-2]H)-C(=[-2]O)-CH_3} + CH3COOH
  }
}
