\subsection{\ce{H2O}・ハロゲン化水素・硫酸などの付加}
\begin{screen}
  \centerline{
    HXの形で結合している分子は\ce{H2}やハロゲンのように不飽和結合に付加する。
  }
\end{screen}
\eg{
  \ce{\chemfig{CH_2 = CH_2} + HCl -> \chemfig{C(-[2]H)(-[4]H)(-[6]H) - C(-[0]H)(-[2]H)(-[6]Cl)}}
}
\eg{
  \ce{\chemfig{CH_2 = CH_2} + H2O ->}
  \ce{\chemfig{C(-[2]H)(-[4]H)(-[6]H) - C(-[0]H)(-[2]H)(-[6]OH)}}
  \raisebox{5pt}{\tiny \hspace*{-94pt} 触媒}
}
\eg{
  \ce{\chemfig{CH ~ CH} + HCl -> \chemfig{C(-[3]H)(-[5]H) = C(-[1]H)(-[-1]OH)} ->}
  \ce{\chemfig{C(-[2]H)(-[4]H)(-[6]H)-C(=[6]O) - H}}
  \raisebox{5pt}{\tiny \hspace*{-90pt} *}
}
*:ビニルアルコールのように\chemfig{C=C}結合に直接\ce{-OH}がついた形(エノール形)は不安定であるため,
\ce{-OH}のH原子が隣のC原子移動し二重結合の位置を変えて安定なケトンやアルデヒド(ケト形)をつくる。
\vskip2\baselineskip
\begin{itembox}[l]{マルコフニコフ則}
  アルケンなどにH-Xを付加するとき
  \eg{
    \ce{CH2 = CHCH3 + HCl}
    &\ce{->}\chemfig{C(-[2]H)(-[4]H)(-[6]H) - C(-[2]H)(-[6]Cl) - C(-[0]H)(-[2]H)(-[6]H)}\quad\text{:2-クロロプロパン} \\
    & \\
    &\ce{->}\chemfig{C(-[2]H)(-[4]H)(-[6]Cl) - C(-[2]H)(-[6]H) - C(-[0]H)(-[2]H)(-[6]H)}\quad\text{:1-クロロプロパン}
  }
  のように複数の生成物が考えられることがある。このとき\\
  \centerline{
    「HはH原子を多くもつ方のC原子に,XはH原子の少ないC原子の方に結合しやすい。」
  }\\
  これをマルコフニコフ則という。
\end{itembox}

% \chemfig{CH_3-CH(-[2]CH_3)-CH_2-C(=[:60]O)-[:-60]O-H}
% \chemfig{CH_3 - CH (-[2] CH (-CH_3) -[2] CH_3) - CH_3}
% \chemfig{C(-[3]H)(-[5]H) = C}
