\subsection{脱水(縮合)}
\begin{screen}
  \centerline{
    分子内あるいは分子間で\ce{H2O}が取れて縮合する。 \\
    触媒として濃硫酸が用いられることが多い。
  }
\end{screen}
アルコール
\eg{\ce{2 CH3CH2OH ->[\Delta][130\text{\sim} 140\degreeC] CH3CH2-O-CH2CH3 + H2O}\quad\text{:分子間}}
\eg{\ce{CH3CH2OH ->[\Delta][160\text{\sim} 170\degreeC] CH2=CH2 + H2O}\quad\text{:分子内}}
カルボン酸とアルコール
\eg{\ce{CH3\ketone OH + HO-CH3 -> CH3\ketone OCH3 + H2O}}

\begin{itembox}[l]{ザイツェフ則}
  アルコールの分子内脱水では得られる分子が1つに定まらないケースがある。\\
  \eg{
    \ce{CH3\CHOH CH2CH3}
    &\ce{->C[-H2O]} \chemfig{CH_3 - C(-[2]H) = C(-[2]H) - CH_3}\quad\text{:2-ブテン} \\
    &\ce{->C[-H2O]} \chemfig{C(-[2]H)(-[4]H) = C(-[2]H) - CH_2CH_3}\quad\text{:1-ブテン}
  }
  しかし,\\
  \centerline{「アルコールの脱水反応で奪われる{H}原子は
  結合する{H}の少ない{C}側から奪われやすい」}\\
  これをザイツェフ則という。
  上の例では2-ブテンが多く生成し,これを主生成物,
  1-ブテンを副生成物という。
\end{itembox}

