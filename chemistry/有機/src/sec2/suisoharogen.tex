\subsection{水素・ハロゲン付加}
\begin{screen}
  \centerline{
    水素やハロゲンは不飽和の炭化水素に適切な触媒下で付加する。
  }
\end{screen}
適切な触媒:\ce{Pd},\ce{Pt},\ce{Ni},光(紫外線)など

\eg{\ce{CH2=CH2 + H2 ->C[Pd] CH3CH3}}

\begin{enumerate}[label=\cdot]
  \item 触媒は\ce{H2 -> 2H}に分けるはたらき
  \item この方法は接触還元と呼ばれる
\end{enumerate}

\eg{\ce{CH2=CH2 + Br2 -> CH2BrCH2Br}}

\begin{enumerate}[label=\cdot]
  \item \ce{Br2}単体の赤褐色が消えることから不飽和結合の検出に用いられる。
\end{enumerate}
