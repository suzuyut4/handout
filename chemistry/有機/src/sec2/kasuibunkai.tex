\subsection{加水分解}
\begin{screen}
  エステル\:\raisebox{5pt}{\chemfig{-[,0.7]C(=[-2]O)-O-[,0.7]}}\:やアミド\:\raisebox{5pt}{\chemfig{-[,0.7]C(=[-2]O)-N(-[-2]H)-[,0.7]}}\:は加水分解される。
\end{screen}
\eg{\ce{
  \chemfig{C(-[2]H)(-[4]H)(-[6]H) - C(=[6]O) - N(-[6]H) -H} + H2O -> 
  \chemfig{C(-[2]H)(-[4]H)(-[6]H) - C(=[6]O)-OH} + \chemfig{N(-[0]H)(-[4]H)(-[6]H)}
}}
\eg{\ce{
  \chemfig{C(-[2]H)(-[4]H)(-[6]H) - C(=[6]O) - O -  C(-[0]H)(-[2]H)(-[6]H)} + H2O -> 
  \chemfig{C(-[2]H)(-[4]H)(-[6]H) - C(=[6]O)-OH} + \chemfig{C(-[4]HO)(-[0]H)(-[2]H)(-[6]H)}
}}
また,一般にエステルをNaOHで加水分解することを「けん化」という。
