\subsection{アルコールの酸化}
\begin{screen}
  \centerline{
    \begin{tabular}{ccccc}
      一級アルコール & \ce{->C[-2H]} & アルデヒド & \ce{->C[+O]} & カルボン酸 
    \end{tabular}
  }
\end{screen}
\eg{\ce{CH3OH -> H\ketone H -> H\ketone OH}}
\eg{\ce{CH3CH2OH -> CH3 \ketone H -> CH3\ketone OH}}
\begin{screen}
  \centerline{
    \begin{tabular}{ccccc}
      二級アルコール & \ce{->C[-2H]} & ケトン  
    \end{tabular}
  }
\end{screen}
\eg{\ce{CH3 \CHOH CH3 -> CH3 \ketone CH3}}
\begin{screen}
  \centerline{
    三級アルコールは酸化されない
  }
\end{screen}
また,\ce{Ni},\ce{Pt},\ce{Pd}などを触媒に\ce{H2}で還元すると逆反応が起こる。
