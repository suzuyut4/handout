\subsection{エステル・アミド}
\chemfig{-[,0.7]C(=[-2]O)-O-[,0.7]}\raisebox{-4pt}{:エステル結合} \qquad
\chemfig{-[,0.7]C(=[-2]O)-N(-[-2]H)-[,0.7]}\raisebox{-4pt}{:アミド結合}

\begin{enumerate}[label=$\cdot$]
  \item 加水分解される\\
  \begin{screen}
    <エステルの加水分解>
    \vskip.3\baselineskip
    \centerline{エステル + $\mathrm{H_2 O}$ \rightarrow カルボン酸 $+$ アルコール}
    <アミドの加水分解>
    \vskip.3\baselineskip
    \centerline{アミド  + $\mathrm{H_2 O}$ \rightarrow カルボン酸 $+$ アミン}
  \end{screen}
  \eg{\ce{
    \chemfig{R_1 - C(=[-2]O)-OR_2} + H2O -> \chemfig{R_1- C(=[-2]O)-OH} + \chemfig{HO-R_2}
  }}
  \eg{\ce{
    \chemfig{R_1 - C(=[-2]O)-N(-[-2]H)-R_2} + H2O -> \chemfig{R_1- C(=[-2]O)-OH} + \chemfig{H-N(-[-2]H)-R_2}
  }}
  \item エステルの命名 \\
  \begin{fleqn}[10pt] \begin{align*}
    &\text{\hspace{0pt}e.g.\quad} \\\\ 
    &\qquad\begin{matrix}
      \text{酢酸メチル} & \text{酪酸エチル}  \\
      \\
      \chemfig{CH_3 - C(=[-2]O)-O - CH_3} & 
      \hspace{10pt}\chemfig{CH_3 - CH_2 - CH_2 - C(=[-2]O) - O - CH_2 - CH_3}
    \end{matrix}
  \end{align*} \end{fleqn}
  
\end{enumerate}
