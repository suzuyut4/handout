\subsection{アルカン\:\:\ce{C_{n}H_{2n+2}}\:(Alkane)}

\begin{table}[H]\centering
\begin{tabular}{ccl}
\toprule
  \ce{C}の数 & 名称 & \\
\midrule
  1 & メタン & \multirow{4}{*}{気体}\\
  2 & エタン & \\
  3 & プロパン & \\
  4 & ブタン & \\
\midrule
  5 & ペンタン & \multirow{2}{*}{液体(油状)} \\
  6 & ヘキサン & \\
\bottomrule
\end{tabular}\end{table}

\begin{enumerate}[label=$\cdot$]
  \item すべて単結合 \\
  非共有電子対の価標で共有結合されている。
  \item 水に不溶
  \item 構造異性体がある\\
  側鎖がついて枝分かれ構造をとる。
  \eg{
    \begin{matrix}
      \ce{\chemfig{CH_3-CH_2-CH(-[-2]CH_3)-CH_2-CH_2-CH_3}} \\
      \text{\raisebox{-10pt}{3-メチルヘキサン}}
    \end{matrix}
  }
  \begin{screen}
    \begin{gather*}
      \begin{matrix}
        \chemfig{-CH_3} & \chemfig{-CH_2-CH_3} & \chemfig{-CH_2-CH_2-CH_3} & \chemfig{CH_3 - CH(-[-2]) - CH_3} \\
        \text{\raisebox{-7pt}{メチル基}} & 
        \text{\raisebox{-7pt}{エチル基}} & 
        \text{\raisebox{-7pt}{プロピル基}} & 
        \text{\raisebox{-7pt}{イソプロピル基}} 
      \end{matrix}
    \end{gather*}
  \end{screen}
  \item 環状構造をとる
\end{enumerate}

\eg{
\raisebox{-15pt}{\chemfig{
  C?(-[:-84,0.8]H)(-[:-168,0.8]H) - C(-[:-12,0.8]H)(-[:-96,0.8]H) -[::72] C(-[:-24,0.8]H)(-[:60,0.8]H) -[::72] C(-[:48,0.8]H)(-[:132,0.8]H) -[::72] C?(-[:120,0.8]H)(-[:204,0.8]H)}
}}
あるいは
\eeg{
  \chemfig{C?H_2 -[::0,,2,1] CH_2 -[::72,,1,1] CH_2 -[::72,1.3,1,1] CH_2 -[::72,1.3,1,2] H_2C?}\qquad
  \chemfig{[:18]*5(-----)}
}
とも描く。
