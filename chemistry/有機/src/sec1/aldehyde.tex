\subsection{アルデヒド}
\chemfig{R - C(=[-2]O)-H}:アルデヒド基(ホルミル基)
\begin{enumerate}[label=$\cdot$]
  \item アルカンの名称をe\rightarrow alにする \\
  \eg{
    &\text{ペンタナール}\\
    &\chemfig{CH_3 - CH_2 - CH_2 - CH_2 - C(=[-2]O) - H}
  }
  慣用名があるものはそれを用いる。
  \eeg{
    &\text{ホルムアルデヒド}\\
    &\chemfig{H - C(=[-2]O) - H}
  }
  \eeg{
    &\text{アセトアルデヒド}\\
    &\chemfig{CH_3 - C(=[-2]O) - H}
  }
  \eeg{
    &\text{プロピオンアルデヒド}\\
    &\chemfig{CH_3 - CH_2 - C(=[-2]O) - H}
  }
  \item 水に溶解しやすく刺激臭
  \item アセトアルデヒドの製法 \\
  \quad<ヘキストワッカー法> \\
  \vskip.3\baselineskip
  \quad\qquad 
  \ce{\chemfig{C(-[3]H)(-[-3]H) = C(-[1]H)(-[-1]H)} + O2 ->C[PdCl2,CuCl2] \chemfig{CH_3 - C(=[-2]O)-H}}
\end{enumerate}


