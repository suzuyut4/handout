\subsection{アルケン \:\:\ce{C_{n}H_{2n}}\:(Alkene)}

\begin{enumerate}[label=$\cdot$]
  \item アルカンの名称をane\rightarrow eneにする
  \item 水に不溶
  \item 構造異性体がある
  \begin{enumerate}[label=\arabic*.]
    \item 1-ブテン \\
    \chemfig{CH_3 - CH_2 - CH = CH_2}\\
    \item 2-ブテン \\
    \chemfig{CH_3 - CH = CH - CH_3} 
    \vskip1\baselineskip
    2-ブテンには幾何異性体(シストランス異性体)がある。\\
    \begin{fleqn}[0pt]\begin{align*}
      \begin{matrix}
        \chemfig{C(-[3]H)(-[-3]H_3C) = C(-[1]H)(-[-1]CH_3)} &
        \chemfig{C(-[3]H)(-[-3]H_3C) = C(-[1]CH_3)(-[-1]H)} \\
        \text{\raisebox{-10pt}{シス-2-ブテン}} &
        \text{\raisebox{-10pt}{トランス-2-ブテン}}
      \end{matrix}
    \end{align*}\end{fleqn}
    \\    
    \item 2-メチルプロピレン \\
    \chemfig{CH_3 - C(=[-2]CH_2) - CH_3}\\
  \end{enumerate}
  
\end{enumerate}

\begin{itembox}[l]{不飽和度}
  ある化合物\ce{C_{k}H_{l}N_{m}O_{n}}における不飽和結合の数を計算できる。
  \begin{gather*}
    I = \frac{2k+m-l+2}{2}
  \end{gather*}
  不飽和結合とは二重結合${(I = 1)}$,三重結合${(I = 2)}$,環状結合${(I = 1)}$のことである。\\
  官能基の不飽和度は次の通り。
  \begin{gather*}
    \begin{array}{cccc}
       & -\chemfig{NH_2} & -\chemfig{COOH} & -\chemfig{NO_2} 
      %  \quad-\chemfig{N(-[1,1.05]O)(=[-1,1.05]O)}\hspace{-16pt}\raisebox{6.42pt}{\rotatebox{45}{\rightarrow}} 
      \\
      \hline
      \raisebox{-3pt}{$I$} & \raisebox{-3pt}{0} & \raisebox{-3pt}{1} & \raisebox{-3pt}{1}
    \end{array}
  \end{gather*}
\end{itembox}
