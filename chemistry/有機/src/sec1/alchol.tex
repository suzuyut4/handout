\subsection{アルコール}
-\chemfig{OH}:ヒドロキシ基を持つもの

\begin{enumerate}[label=$\cdot$]
  \item アルカンの名称をane\rightarrow ol
  \item 価数と沸点
  \begin{fleqn}[0pt]
    \begin{align*}
      \begin{dcases*}
        \text{1価}\colon 60\text{\sim} 90\degreeC \\
        \text{2価}\colon 200\degreeC \\
        \text{3価}\colon 300\degreeC \\
      \end{dcases*}
    \end{align*}
  \end{fleqn}
  
  \eg{
    &\text{エチレングリコール(2価)} \\ 
    &\qquad \chemfig{CH_2(-[-2]CH_2 - OH) - OH}
  }
  % \eeg{
  % }  
  \eg{
    &\text{グリセリン(3価)} \\
    &\qquad \chemfig{CH_2(-[-2]CH\twosizebox(-[-2]CH_2 - OH) - OH) - OH}    
  }
  % \eeg{
  % }

  \item 水とは任意の割合で混ざる
  \item 級数
  -\ce{OH}のついている先の炭素原子についた水素原子の個数。
  1級のほうが分子間で水素結合を形成しやく沸点は高い。  
  \begin{fleqn}[0pt]
    \begin{align*}
      \begin{dcases*}
        \text{1級}\colon 2,3個 \\
        \text{2級}\colon 1個 \\
        \text{3級}\colon 0個 \\
      \end{dcases*}
    \end{align*}
  \end{fleqn}
\end{enumerate}


