
\pagestyle{plain} % empty:ページ番号削除

%-------------------------------------------------------%

% セクション・サブセクションの見出しのサイズ
\titleformat*{\section}{\Large\bfseries}
\titleformat*{\subsection}{\large\bfseries}


% captionパッケージを用いないでも変更できるっぽい
\renewcommand{\figurename}{Fig.}
\renewcommand{\thefigure}{\arabic{figure}}
\renewcommand{\tablename}{Table}
\renewcommand{\thetable}{\arabic{table}}

% 再帰的にコマンドを改変する方法
% これによってrenewcommandで改変先のコマンドも使える
% \let\origincaption\caption
% \renewcommand{\caption}[1]{\origincaption{#1}}

% 式番号にセクションを併記する場合
\numberwithin{equation}{section}

%-------------------------------------------------------%

% \usepackage以外で@を含むときはこれで囲む
\makeatletter

\newcommand{\reference}[0]{\setlength{\hangindent}{18pt}\noindent}

\newcommand{\yearr}[0]{\number\year}

% 数式や図表のref
\renewcommand{\refeq}[1]{\eqref{#1}式}
\newcommand{\reffig}[1]{図\ref{#1}}
\newcommand{\reftbl}[1]{表\ref{#1}}

% よく使う単位
\newcommand{\degreeC}[0]{\mathrm{{}^\circ \hspace*{-0.5pt} C}}
\newcommand{\degree}[0]{\mathrm{{}^\circ}}

% よく使う演算子
\renewcommand{\Vector}[1]{{\mbox{\boldmath$#1$}}}
\newcommand{\tensor}[1]{\undertilde{#1}}
\renewcommand{\rm}[1]{\mathrm{#1}}


% 参考文献
\newcommand{\refcite}[2]{\cite{#1}${}^{\text{#2}}$}
\renewcommand{\citeform}[1]{#1)}
\renewcommand{\@biblabel}[1]{#1)}


% %%%%%%%%%%%%%%%%%%%%%%%%%%%%%%%%%%%%%%%%%%%%%%%%%%%%%%%%%%% 

% sectionの装飾
% あとはDists/chemistry/theoreticalChemistryを参照
\titleformat{\section}[block]
{\Large\bfseries \S \thesection}{}{0pt}
{
  \normalfont \Large\bfseries
}[]

\titleformat{\subsection}[block]
{
  % sectionとしたら絶対につく装飾
  {\colorbox{black}{\begin{picture}(0,10)\end{picture}}}
  \hspace*{-13pt}
  \raisebox{-4.5pt}{\begin{picture}(0,0)
    \put(0,0){\color{black}\line(1,0){300}}
  \end{picture}}
}
{
  % section*では消える場所
  \hspace{0pt}
  {\normalfont \large\bfseries }
}{0pt}
{
  \normalfont \large\bfseries 
}
[ 
]



% %%%%%%%%%%%%%%%%%%%%%%%%%%%%%%%%%%%%%%%%%%%%%%%%%%%%%%%%%%% 
% 
% newenvironment
% 
% leftbar の定義
\renewenvironment{leftbar}{%
\def\FrameCommand{\hspace{10pt}\vrule width 2pt \hspace{10pt}}%
\MakeFramed {\advance\hsize-\width \FrameRestore}}%
{\endMakeFramed}

% %%%%%%%%%%%%%%%%%%%%%%%%%%%%%%%%%%%%%%%%%%%%%%%%%%%%%%%%%%% 
% 
% その他のコマンドたち

% 数式環境(displaystyle math)
\newcommand{\dm}[1]{
  $\displaystyle #1 $
}

% 数式環境alignをいちいち入力しなくてよい
\newcommand{\sushiki}[1]{\begin{align}#1\end{align}}

% 化学式をalignかつ左寄せにする
\newcommand{\kagakushiki}[1]{\begin{fleqn}[0pt] \begin{align*} #1 \end{align*}\end{fleqn}}

% 穴埋め用のbox
% \renewcommand{\fbox}[0]{\:\raisebox{-2pt}{\framebox[1.5cm][c]{\rule{0pt}{1em}}}\:}


% 問題環境用のコマンド(problem)
\newcommand{\pb}[1]{
  \begin{leftbar}#1\end{leftbar}
}

% 引用(インデントブロック)環境
\newcommand{\blk}[1]{\begin{quote}#1\end{quote}}


% カウンター
% - 通し番号を\qcとして振れる
% - \setcounter{qc}{0} とすれば1からカウントにリセットできる
\newcounter{qc}
\newcommand{\qc}{\stepcounter{qc}\theqc}


% 定理スタイルの定義
\newtheoremstyle{mystyle}
  {\topsep}   % スペース上
  {\topsep}   % スペース下
  {\normalfont}  % 本文のフォント
  {0pt}       % インデント
  {\bfseries} % タイトルのフォント
  {.}         % 句読点
  {.5em}      % タイトルと本文のスペース
  {}          % タイトルのスペース
% 定理環境の作成
\theoremstyle{mystyle}
% newtheorem
\newtheorem{proposition}{命題}
\newtheorem{theorem}{定理}[section] % 章ごとに番号付け
\newtheorem{lemma}[theorem]{補題} % 定理と同じ番号付け
\newtheorem{corollary}{系}[section]
\newtheorem*{appendix*}{補遺}
\newtheorem*{supplement*}{補足}
% \newtheorem*{eg*}{e.g.} % exempli gratia

\newcommand{\supple}[1]{\begin{itembox}[l]{補足} #1\end{itembox}}
\newcommand{\theme}[1]{\textbf{#1}}
\newcommand{\supplement}[2]{\begin{itembox}[l]{補足: \:\theme{#1}} #2\end{itembox}}
\setchemfig{atom sep=1.9em}

\newsavebox{\mybox}      % ボックスレジスタ\myboxを新しく作成
\sbox{\mybox}{e.g.}     % ボックスレジスタ\myboxに横ボックスを作成し、その中に"abcde"というテキストを配置
\newcommand{\mysizebox}{\hspace{\wd\mybox}}
\newsavebox{\twobox}
\sbox{\twobox}{2} 
\newcommand{\twosizebox}{\hspace{\wd\twobox}}
\newsavebox{\empbox}
\sbox{\empbox}{$_2$} 
\newcommand{\utwosizebox}{\hspace{\wd\empbox}}

% e.g.~
\newcommand{\eg}[1]{
  % \begin{fleqn}[0pt]\begin{align*} \text{e.g.} \end{align*}\end{fleqn}%
  % \vskip-1\baselineskip
  \begin{fleqn}[10pt] \begin{align*}\text{e.g.\quad} #1 \end{align*} \end{fleqn}%
  % \vskip1\baselineskip
}

% (e.g.と同じ幅のスペース)~
\newcommand{\eeg}[1]{
  % \begin{fleqn}[0pt]\begin{align*} \text{e.g.} \end{align*}\end{fleqn}%
  % \vskip-1\baselineskip
  \begin{fleqn}[10pt] \begin{align*}\text{\hspace{\wd\mybox}\quad} #1 \end{align*} \end{fleqn}%
  % \vskip1\baselineskip
}
