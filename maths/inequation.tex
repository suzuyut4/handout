\documentclass[autodetect-engine,ja=standard, 10.5pt, a4paper, titlepage]{bxjsarticle}
% fleqn:数式を左詰めにする(titlepageの前に挿入可能)
% titlepage:表紙を独立させる
%\setlength{\mathindent}{50pt}
%-------------------------------------------------------%


\usepackage{graphicx} % Required for inserting images
\usepackage{titlesec}
\usepackage{caption}
\usepackage{amsmath} % 数式用
\usepackage{amssymb}
\usepackage{amsmath}
\usepackage{enumerate} % 箇条書き
\usepackage{comment} % コメントアウト
\usepackage[super]{cite} % 参考文献 上付き
\usepackage[version=4]{mhchem}
\usepackage{booktabs} % tableのmidrule
\usepackage{multirow} % tableのmultirow
\usepackage{float} % [H]で厳密に位置を固定
\usepackage{nccmath} % 数式を左に動かす
\usepackage{mathtools}
\usepackage{empheq}
\usepackage{accents} % \undertildeで下付きチルダ
\usepackage{nccmath} % 数式左寄せ環境fleqn
\usepackage{siunitx} % 単位書き方(si)

%-------------------------------------------------------%

\pagestyle{plain} % empty:ページ番号削除

%-------------------------------------------------------%

% セクション・サブセクションの見出しのサイズ
\titleformat*{\section}{\Large\bfseries} % サイズ・太字
\titleformat*{\subsection}{\large\bfseries}

%-------------------------------------------------------%

\newcommand{\reference}[0]{\setlength{\hangindent}{18pt}\noindent}
\renewcommand{\refeq}[1]{\eqref{#1}式}
\newcommand{\reffig}[1]{図\ref{#1}}
\newcommand{\reftbl}[1]{表\ref{#1}}
\newcommand{\degreeC}[0]{\mathrm{{}^\circ \hspace*{-0.5pt} C}}
\newcommand{\degree}[0]{\mathrm{{}^\circ}}
\newcommand{\Vector}[1]{{\mbox{\boldmath$#1$}}}
\renewcommand{\v}[1]{\overrightarrow{\mathstrut{#1}}}
\newcommand{\tensor}[1]{\undertilde{#1}}
\renewcommand{\rm}[1]{\mathrm{#1}}

\newcommand{\refcite}[2]{\cite{#1}${}^{\text{#2}}$}
\renewcommand{\citeform}[1]{#1)}
\makeatletter % \usepackage以外で@を含むときはこれで囲む
\renewcommand{\@biblabel}[1]{#1)}
\makeatother

\numberwithin{equation}{section} % 式番号にセクションを併記する場合

%**************************************************************
\begin{document}
%\parindent = 0pt % 常に字下げなし
\leftline{Citation: 東京出版 大学への数学 2020-07 vol.64, 安田 亨}
\vskip.2cm
\leftline{\large 要点の整理/数I II}
\vskip.1cm
\centerline{\LARGE 不等式―基本の総括}
\vskip0.5cm


\section{不等式の3つの意味}\label{sec:3-meanings}
不等式にはいくつかの意味があり,
どの意味で使っているのかを間違えると誤答につながるので注意しよう.

\begin{enumerate}[1.]
  \item 単なる大小関係をあらわすときに使う
  \begin{align*}
    a\geq b \text{ は } a > b \text{ または } a = b  
  \end{align*}
  の少なくとも一方が成り立つと主張しているだけである.例えば 3\geq 1 は 3>1 が成り立つので正しい.\\

  \item 解集合を表す\\
  たとえば $x^2-3x+2=0$ を解くと $x$ は $1$ と $2$ となり,$\{1,2\}$  という解の集合が求められる.
  $x^2-3x+2<0$ を解くと1<x<2 となり,$1<x<2$ を満たす $x$ 全体のことである.\\

  \item 取りうる値の範囲を表す\\
  実数 $x$ が $1$ と $2$ の間のすべての値をとって動くことを $1<x<2$ と表す場合がある.
\end{enumerate}


\section{不等式を解く}\label{sec:solve-ineq}
$a<b$ のとき
\begin{enumerate}[1.]
  \item
  \begin{align}
    (x-a)(x-b) > 0 & \Longleftrightarrow x < a, \: x > b \\
    (x-a)(x-b) \geq 0 & \Longleftrightarrow x \leq a, \: x \geq b \label{eq:xaxb-out}
  \end{align}
  「$a, \: b$ の外側」と理解する.
  \item
  \begin{align}
    (x-a)(x-b) <0 & \Longleftrightarrow a < x < b \\
    (x-a)(x-b) \leq 0 & \Longleftrightarrow a \leq x \leq b \label{eq:xaxb-in}
  \end{align}
  「$a, \:b$ の内側」と理解する.
\end{enumerate}
分数不等式を解く場合は,分母を払う人が多いが,
符号を考えないでいきなり払ってしまうミスが少なくない.
それを防ぐためには「移項して通分」「各区間の $x$ に対して符号を判別」をを行う方がよい.


\section{不等式証明の基本}\label{sec:basis-ineq}
「不等式 $A\geq B$ を証明せよ」という問題の場合,$A-B$ を作り,
整理し,因数分解化平方完成するというのは基本的手法である.

\section{絶対値の外し方}\label{sec:how-to-getridof-abs}
「絶対値をはずすときには符号で場合分けせよ」と習うが,
それは基本であり,込み入った問題では場合分けを減らすことが,
その後の処理を簡単にしてくれる.

\begin{enumerate}[1.]
  \item 
  $|X| = |A|$のときは,$A\geq \pm A$と外す。
  $X$の正負には言及しない。\\
  $|X| = A$のときは$A\geq 0$かつ$|X| = \pm A$と外す。\\
  
  \item 
  $|X|\leq A$のときは$-A\leq X\leq A$と外す。\\
  このとき,$A\geq 0$でなければならないが,
  $-A\leq X\leq A$のときには$-A\leq A$だから$A\geq 0$となる。
  $A,\:X$の符号に言及する必要はない。\\
  
  \item 
  $|X|\geq A$の解について,$A<0$のとき$X$は任意,
  $A\geq 0$のとき$X\leq -A$,$X\geq A$\\
  実は,$A$の符号によらず$X\leq -A,\:X\geq A$を外すことができる。
\end{enumerate}

\section{有名な不等式を利用する}\label{sec:use-famous-ineqs}
\subsection{三角不等式}

\begin{enumerate}[1.]
  \item 
  実数について$||x|-|y||\leq |x+y|\leq |x|+|y|$\\
  左の等号は$xy\leq 0$,右の等号は$xy\geq 0$で成り立つ。

  数学I\!I\!Iの複素数ではこの不等式を使うチャンスは多いが,
  実数では,使う問題がほとんど出ない。\\
  
  \item 
  空間の3点$\rm{P}$,$\rm{Q}$,$\rm{R}$について$\rm{PR}\leq \rm{PQ} + \rm{QR}$\\
  等号は$\rm{P}$,$\rm{Q}$,$\rm{R}$の順で一直線上にあるとき成り立つ。\\
  
  \item 
  $||\v{x}|-|\v{y}|| \leq |\v{x} + \v{y}| \leq |\v{x}| + |\v{y}|$\\
  左の等号は$\v{x}$,$\v{y}$の一方が$\v{0}$か逆向きに平行のとき,
  右の等号は$\v{x}$,$\v{y}$の一方が$\v{0}$か同じ向きに平行のとき
  成り立つ。
\end{enumerate}

\subsection{$x^2 + y^2 + z^2 \geq xy + yz + zx$}
【証明】
\begin{fleqn}[20pt]
  \begin{align*}
    & x^2 + y^2 + z^2 - (xy + yz + zx)\\
    =& \frac{1}{2} {(x-y)^2 + (y-z)^2 + (z-x)^2} \geq 0
  \end{align*}
\end{fleqn}


\subsection{相加・相乗平均の不等式(AM-GM Inequality)}
文字はすべて正,$n$は自然数とする。\\
\begin{fleqn}[20pt]
  \begin{align*}
    &\frac{x+y}{2} \geq \sqrt{xy} \tag*{(A)}\label{eq:a}\\
    &\frac{x+y+z}{3} \geq \sqrt[3]{xyz} \tag*{(B)}\label{eq:b}\\
    &\frac{x_{1}+ \cdots +x_{n}}{n} \geq \sqrt[n]{x_{1}\cdots\cdots x_{n}} \label{eq:n}
  \end{align*}
\end{fleqn}
\ref*{eq:a}の等号は$x=y$のとき成り立つ。
他も同様である。
\ref*{eq:a}の証明は$\cfrac{x+y}{2} - \sqrt{xy} = \cfrac{1}{2}(\sqrt{x}-\sqrt{y})^2 \geq 0$\\
\ref*{eq:b}の証明は
\begin{fleqn}[20pt]
  \begin{align*}
    &A^3+B^3+C^3-3ABC\\
    =&(A+B+C)(A^2+B^2+C^2-AB-BC-CA)\\
    =&(A+B+C)\frac{(A-B)^2 + (B-C)^2 + (C-A)^2}{2}\geq 0
  \end{align*}
\end{fleqn}
として,$A^3+B^3+C^3-3ABC\geq 0$で
$A=\sqrt[3]{x}$,
$B=\sqrt[3]{y}$,
$C=\sqrt[3]{z}$
とおけばよい。
次のコーシーの証明方法も有名である。
\ref*{eq:a}を2回使って,
\begin{fleqn}[20pt]
  \begin{align*}
    \frac{x+y+z+w}{4}
    &\geq \frac{2\sqrt{xy}+2\sqrt{zw}}{4} = \frac{\sqrt{xy}+\sqrt{zw}}{2}\\
    &\geq \sqrt{\sqrt{xy}\cdot \sqrt{zw}}=\sqrt[4]{xyzw}
  \end{align*}
\end{fleqn}
よって$\cfrac{x+y+z+w}{4}\geq \sqrt[4]{xyzw}$となり,ここで
$w=\cfrac{x+y+z}{3}$とおくと,不等式は
$\cfrac{x+y+z}{3}\geq \sqrt[4]{xyz\cdot \cfrac{x+y+z}{3}}$となる。
両辺を4乗し$\cfrac{x+y+z}{3}$で割ると
$\left(\cfrac{x+y+z}{3}\right)^3\geq xyz$
両辺の3乗根をとると\ref*{eq:b}を得る。


\subsection{コーシー・シュワルツの不等式(Cauchy-Schwarz inequality)}
\begin{fleqn}[20pt]
  \begin{align*}
    (a_{1}^2+a_{2}^2 + \cdots\cdots + a_{n}^2)(b_{1}^2+b_{2}^2 + \cdots\cdots + b_{n}^2)
    \geq (a_{1}b_{1} + a_{2}b_{2} + \cdots \cdots + a_{n}b_{n})^2
  \end{align*}
\end{fleqn}
等号は$\cfrac{b_{1}}{a_{1}}=\cfrac{b_{2}}{a_{2}}= \cdots\cdots =\cfrac{b_{n}}{a_{n}}$
(分母が0の項は分子も0とする)のとき成り立つ,と書くのが普通である。
厳密には「$a_{1}$から$a_{n}$までがすべて0か,
または$a_{1}$から$a_{n}$までの中に0でないものがあるときは
$\cfrac{b_{1}}{a_{1}}=\cfrac{b_{2}}{a_{2}}= \cdots\cdots =\cfrac{b_{n}}{a_{n}}$
(分母が0の項は分子も0とする)」のときだが,
0の記述は実用上意味がない。\\
【証明】\\
$A=a_{1}^2+\cdots\cdots + a_{n}$,$B=b_{1}+\cdots\cdots+b_{n}$,
$C=a_{1}b_{1}+\cdots\cdots +a_{n}b_{n}$とおいて$AB\geq C^2$を示す。
$A=0$のときは$a_{1} = \cdots\cdots =a_{n}=0$で,
証明すべき不等式の等号が成り立つ。$A>0$のときは


\section{最大値・最小値への応用}\label{sec:apply-to-max-min}

\end{document}
