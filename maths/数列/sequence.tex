\documentclass[a4paper]{ltjsarticle}

\input{../asset/preamble.tex}

\title{
  数列と漸化式 \\
  {\Large ---問題編---}
}
\author{{\normalsize 鈴木}}
\date{}


%**************************************************************
\begin{document}

\maketitle

\section{数列と漸化式の基本}

\subsection{漸化式}
そもそも、公式を用いて簡単に解ける漸化式は次の3つに限られます。

\begin{enumerate}[label=(\roman*)]
  \item
  $a_{n+1} = a_n + d$  \rightarrow 公差$d$の等差数列
  \fla{
      \Rightarrow a_n = a_1 + (n-1)d \label{eq:tousa}
    }
  \item
  $a_{n+1} = ra_n$     \rightarrow 公比$r$の等比数列
  \fla{
      \Rightarrow a_n = a_1 r^{n-1} \label{eq:touhi}
    }
  \item
  $a_{n+1} = a_n + b_n$\rightarrow 階差数列が$\{b_n\}$の数列$\{a_n\}$
  \fla{
      \Rightarrow (n\geqq 2 \text{のとき、})\:\:a_n = a_1 + \sum_{k=1}^{n-1}b_k \label{eq:kaisa}
    }
\end{enumerate}
以下、\eqref{eq:tousa}の形を等差型、\eqref{eq:touhi}の形を等比型、\eqref{eq:kaisa}の形を階差型
と呼ぶことにします。
この形以外のほぼすべての漸化式はなんとかしてこの形に帰着させることが目的で、
多くの場合は等比数列の形に変形してから一般項を求めるということも覚えておくとよいです。

せっかくなので\eqref{eq:kaisa}の形だけはここで証明しておきましょう。
\begin{prf*}
$a_{n+1} = a_n + b_n \Leftrightarrow a_{n+1} - a_n = b_n$であるので、
$n$を$n-1,n-2,\cdots,2,1$として足し合わせると
{\newcommand{\dl}[1]{\raisebox{-2pt}{$#1$}}
\flan{
    &\begin{matrix}
      & a_n     & - & a_{n-1} & = & b_{n-1} \\
      & a_{n-1} & - & a_{n-2} & = & b_{n-2} \\
      & a_{n-2} & - & a_{n-3} & = & b_{n-3} \\
      &       & \vdots &    &\vdots &       \\
      & a_3     & - & a_2     & = & b_2     \\
  +) & a_2     & - & a_1     & = & b_1     \\
  \hline
      & \dl{a_n}     & \dl{-} & \dl{a_1}     & \dl{=} & \dl{b_{n-1} + b_{n-2} + \cdots + b_2 + b_1}
    \end{matrix}
    \\
    &\Leftrightarrow a_n = a_1 + \sum_{k=1}^{n-1}b_k \qquad \qed
}
}
\end{prf*}

ここで、証明一行目において$n-1$以下のケースを考えることによってこの式を得ています。
しかし、この変形ができるのは$n-1\geqq 1$すなわち$n\geqq 2$のときのみです。
そのため、階差型においては条件$n\geqq 2$を忘れてはいけません。
すなわち、一般項を求めたとき、$n=1$でもその式が成り立っているか必ず確認し、
成り立っていればそのように書き、成り立っていなければ場合分けして一般項を示す必要があります。
解答の書き方など詳しくは、基本編で確認してください。


\subsection{数列の和$S_n$}
一般項$a_n$で表される数列について第一項から第$n$項までの和を$S_n$で表すことがあります。
すなわち
\flan{S_n = \sum_{k=1}^{n}a_k}
です。以下、特に断りがなければ$S_n$を数列$\{a_n\}$に対する和を表すものとします。

\clearpage
\section{数列とその周辺}
\begin{question*}
  次の値を$n$を用いて表せ。
\begin{enumerate}[label=\arabic*.]
  \item
    \dm{\sum_{k=1}^{n}k}
    \vskip.5\baselineskip
  \item
    \dm{\sum_{k=1}^{n}k^2}
    \vskip.5\baselineskip
  \item
    \dm{\sum_{k=1}^{n}k^3}
    \vskip.5\baselineskip
  \item
    \dm{\sum_{k=1}^{n}\frac{3}{k(k+2)}}
    \vskip.5\baselineskip
  \item
    \dm{\sum_{k=1}^{n}\frac{3}{\sqrt{k+2} + \sqrt{k}}}
    \vskip.5\baselineskip
  \item
    \dm{\sum_{k=1}^{n}\{(2k-1)\cdot 2^{k-1}\}}
    \vskip.5\baselineskip
  \item
    \dm{\frac{1}{1} + \frac{1}{1+2} + \frac{1}{1+2+3} + \cdots + \frac{1}{1+2+\cdots +k} + \cdots + \frac{1}{1+2+3+\cdots +(n-1)+n}}
    \vskip.5\baselineskip
  \item
    \dm{1+\frac{2}{3}+\frac{3}{3^2} + \cdots + \frac{k}{3^{k-1}} + \cdots + \frac{n}{3^{n-1}}}
    \vskip.5\baselineskip
\end{enumerate}
\end{question*}

\clearpage
\section{漸化式 基本編}

\begin{question*}
  次の式を満たす数列$\{a_n\}$を$n$を用いて表せ。
\begin{enumerate}[label=\arabic*.]

  \item $\disp a_1 = 2,\; a_{n+1} = a_n + 3$\\

  \item $\disp a_1 = 5,\; a_{n+1} = 7a_n$\\

  \item $\disp a_1 = 3,\; a_{n+1} = 3a_n -4$\\

  \item $\disp a_1 = 1,\; a_{n+1} = a_n + 2n$\\

  \item $\disp a_1 = 1,\; a_{n+1} = a_n + 3^n - 4n$\\

  \item $\disp a_1 = 1,\; a_2 = 2,\; a_{n+2} = a_{n+1} + 6a_n$\\

  \item $\disp a_1 = 2,\; a_{n+1} = 16a_n^5$\\

  \item $\disp a_1 = 1,\; a_{n+1} = 2a_n + n^2 -6$\\

  \item $\disp a_1 = 1,\; a_{n+1} = 4a_n + n\cdot 2^n$\\

  \item $\disp a_1 = 1,\; a_2 = 3,\; a_{n+2} = 4a_{n+1} - 4a_n$\\

  \item $\disp a_1 = 1,\; a_2 = 4,\; a_{n+2} = 4a_{n+1} - 3a_n - 2$\\

  \item $\disp a_1 = 1,\: a_{n+1} = \frac{a_n}{2a_n + 3}$\\

  \item $\disp a_1 = 3,\; a_{n+1} = \frac{3a_n-4}{a_n-2}$\\

  \item $\disp a_1 = 3,\; a_{n+1} = \frac{3a_n-4}{a_n-1}$\\

  \item $\disp a_1 = 1,\; a_{n+1}=(n+1)a_n$\\

  \item $\disp a_1 = 1,\; (n+2)a_{n+1}=na_n$\\

  \item $\disp a_1 = 1,\; na_{n+1}=2(n+1)a_n+n(n+1)$\\

  \item $\disp a_1 = 2,\; a_{n+1} = \frac{n+2}{n} a_n + 1$\\

  \item $\disp a_1 = 1,\; a_{n+1} = 2^{2n-2}(a_n)^2$\\

  \item $\disp S_n = 3a_n + 2n - 1$

\end{enumerate}
\end{question*}


\clearpage
\section{漸化式 演習編}
ここでは、入試問題(二次試験)レベルの数列の問題を扱います。
少々難しい問題もありますが、ぜひ取り組んでみてください。

\begin{question*}
次の問いに答えよ。
\begin{enumerate}[label=\arabic*.]
  \item %@ 問1
  $n$を自然数として次の条件で定められた数列$\{a_n\}$について2通りの解き方を考えよう。\\ % (福井大 改 +α)

    \al{
      a_1 = 1,\:\: a_{n+1} = \frac{3}{n} (a_1 + a_2 + a_3 + \cdots + a_n)  \quad\cdots (*)
    }

  \begin{enumerate}[label=(\arabic*)]
    \item $a_2,\;a_3,\;a_4$を計算せよ。\\
    \item 一般項$\{a_n\}$を推定し、それが正しいことを数学的帰納法を用いて示せ。\\
    \item 上の漸化式$(*)$について、$a_1 + a_2 + a_3 + \cdots + a_{n-1}$を$a_n$と$n$を用いて表せ。\\
    \item $a_{n+1}$と$a_n$の関係を導いた上で、一般項$a_n$を$n$を用いて表せ。\\
  \end{enumerate}

\vskip1.5\baselineskip

  \item %@ 問2
  \begin{enumerate}[label=(\arabic*)]
    \item 次の初項、二つの漸化式で与えられる数列$\{a_n\},\, \{b_n\}$を考える。\\
      \al{
        &a_1 = 5,\quad b_1 = 3\\
        &a_{n+1} = 5a_n + 3b_n \\
        &b_{n+1} = 3a_n + 5b_n \\
      }
    2つの数列$\{a_n\pm b_n\}$を求め、一般項$a_n,\, b_n$を求めよ。\\

    \item (1)を踏まえて次の初項、二つの漸化式で与えられる数列$\{p_n\},\, \{q_n\}$の一般項をそれぞれ求めよ。% (大阪医科大学)

        \al{
          &p_1 = 1,\quad q_1 = 4\\
          &p_{n+1} = 2p_n + q_n \\
          &q_{n+1} = 4p_n - q_n \\
        }
      % ヒント:
      % \ $\{p_n + t q_n\}$が等比数列となるような$t$を二つ求める。
      % すなわち、\ $p_{n+1} + tq_{n+1} = s(p_n + t q_n) $を満たす$s,t$の組を二つ見つける。
    \end{enumerate}
\vskip1.5\baselineskip
  \item %@ 問3
  $n$を自然数、$x_1 = \sqrt{a}$として次の漸化式で与えられる数列$\{x_n\}$を考える。
  \al{
    x_{n+1} = \sqrt{x_n + a}
  }
  すなわち、
  \al{
    x_2 = \sqrt{a + \sqrt{a}}, \qquad x_3 = \sqrt{a + \sqrt{a + \sqrt{a}}}, \qquad\dots
  }
  である。
  この数列が収束するかどうかを調べたい。
  次の問いに答えよ。

  \begin{enumerate}[label=(\arabic*)]
    \item 数列$\{x_n\}\:(n\in \mathbb{N})$が収束すると仮定して、その極限値を求めよ。
    \item 数列$\{x_n\}\:(n\in \mathbb{N})$が(1)で得た値に実際に収束することを示せ。
  \end{enumerate}

\end{enumerate}
\end{question*}





\end{document}
