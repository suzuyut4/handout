\documentclass[a4paper]{ltjsarticle}

\input{../asset/preamble.tex}

\title{
  数列と漸化式 \\
  {\Large ---解答編---}
}
\author{\normalsize 鈴木}
\date{}

%**************************************************************
\begin{document}

\maketitle

\section{数列と漸化式の基本}

\subsection{漸化式}

そもそも、公式を用いて簡単に解ける漸化式は次の3つに限られます。

\begin{enumerate}[label=(\roman*)]
  \item
  $a_{n+1} = a_n + d$  \rightarrow\, 公差$d$の等差数列
  \fla{
      \Rightarrow a_n = a_1 + (n-1)d \label{eq:tousa}
    }
  \item
  $a_{n+1} = ra_n$     \rightarrow\, 公比$r$の等比数列
  \fla{
      \Rightarrow a_n = a_1 r^{n-1} \label{eq:touhi}
    }
  \item
  $a_{n+1} = a_n + b_n$\rightarrow\, 階差数列が$\{b_n\}$の数列$\{a_n\}$
  \fla{
      \Rightarrow n\geqq 2 \text{のとき、}\:a_n = a_1 + \sum_{k=1}^{n-1}b_k \label{eq:kaisa}
    }
\end{enumerate}
以下、\eqref{eq:tousa}の形を等差型、\eqref{eq:touhi}の形を等比型、\eqref{eq:kaisa}の形を階差型
と呼ぶことにします。
この形以外のほぼすべての漸化式はなんとかしてこの形に帰着させることが目的で、
多くの場合は等比数列の形に変形してから一般項を求めるということも覚えておくとよいです。

せっかくなので\eqref{eq:kaisa}の形だけはここで証明しておきましょう。
\begin{prf*}
$a_{n+1} = a_n + b_n \Leftrightarrow a_{n+1} - a_n = b_n$であるので、
$n\geqq 2$のとき$n$を$n-1,n-2,\cdots,2,1$として足し合わせると
{\newcommand{\dl}[1]{\raisebox{-2pt}{$#1$}}
\flan{
    &\begin{matrix}
      & a_n     & - & a_{n-1} & = & b_{n-1} \\
      & a_{n-1} & - & a_{n-2} & = & b_{n-2} \\
      & a_{n-2} & - & a_{n-3} & = & b_{n-3} \\
      &       & \vdots &    &\vdots &       \\
      & a_3     & - & a_2     & = & b_2     \\
  +) & a_2     & - & a_1     & = & b_1     \\
  \hline
      & \dl{a_n}     & \dl{-} & \dl{a_1}     & \dl{=} & \dl{b_{n-1} + b_{n-2} + \cdots + b_2 + b_1}
    \end{matrix}
    \\
    &\Leftrightarrow a_n = a_1 + \sum_{k=1}^{n-1}b_k \qquad \qed
}
}
\end{prf*}

ここで、証明一行目において$n-1$以下のケースを考えることによってこの式を得ています。
しかし、この変形ができるのは$n-1\geqq 1$すなわち$n\geqq 2$のときのみです。
そのため、階差型においては条件$n\geqq 2$を忘れてはいけません。
すなわち、一般項を求めたとき、$n=1$でもその式が成り立っているか必ず確認し、
成り立っていればそのように書き、成り立っていなければ場合分けして一般項を示す必要があります。
解答の書き方など詳しくは、基本編の演習で確認してください。


\subsection{数列の和$S_n$}
一般項$a_n$で表される数列について第一項から第$n$項までの和を$S_n$で表すことがあります。
すなわち
\flan{S_n = \sum_{k=1}^{n}a_k}
です。以下、特に断りがなければ$S_n$を数列$\{a_n\}$に対する和を表すものとします。

\newpage
\section{数列とその周辺}
\begin{question*}
次の値を$n$を用いて表せ。
\begin{ans*}
${}$
\begin{enumerate}[label=\arabic*.]
  \item
    \flan{
        \sum_{k=1}^{n}k
        &= \frac{1}{2}n(n+1)
    }
  \item
    \flan{
        \sum_{k=1}^{n}k^2
        &= \frac{1}{6}n (n+1)(2n+1)
    }
  \item
    \flan{
        \sum_{k=1}^{n}k^3
        &= \left\{ \frac{1}{2}n(n+1) \right\}^2
    }
  \begin{supple*}
  $\disp \sum_{k=1}^{n}k^m$の公式については次のように求められていることを思い出しておきましょう。

  $(k+1)^m$の展開公式から得る次の恒等式
  \flan{
      (k+1)^m - k^m = \sum_{i=0}^{m-1}{}_{m}\rm{C}_{i}k^i
  }
  を用いる。たとえば$m=4$ならば
  \flan{
      (k+1)^4 - k^4 = 1 + 4k + 6k^2 + 4k^3
  }
  この両辺に$k=1$から$n$までの和をとって
  \flan{
      &\sum_{k=1}^{n} \{(k+1)^4 - k^4 \} \\
      &= 2^4 - 1^4 + 3^4 - 2^4 + \cdots + n^4 - (n-1)^4 + (n+1)^4 - n^4 \\
      &= (n+1)^4 - 1 \\
      &=\sum_{k=1}^{n}(1 + 4k + 6k^2 + 4k^3)
  }
  すなわち
  \flan{
      (n+1)^4 - 1 = \sum_{k=1}^{n}(1 + 4k + 6k^2 + 4k^3)
  }
  であり、これを\dm{\sum_{k=1}^{n}k^3}を左辺に、その他を右辺に移項して整理すると
  $m=1,2$の$\disp \sum_{k=1}^{n}k^m$がわかっていれば
  左辺を$n$についての式として得られるということになる。
  これを繰り返せば$m=4$以上でも同様に公式を得ることができる。
\end{supple*}
といった感じです。
  \item
    \flan{
        \sum_{k=1}^{n}\cfrac{3}{k(k+2)}
        &= \frac{3}{2}\sum_{k=1}^{n}\left( \frac{1}{k} - \frac{1}{k+2} \right) \\
        \begin{split}
          &= \frac{3}{2} \left\{
            \left(1-\frac{1}{3}\right) + \left(\frac{1}{2}-\frac{1}{4}\right)
            + \left(\frac{1}{3}-\frac{1}{5}\right)
            + \cdots \right. \\
          & \left.\hspace*{50pt}
            + \left(\frac{1}{n-1}-\frac{1}{n+1}\right)
            + \left(\frac{1}{n}-\frac{n}{n+2}\right)
            \right\}
        \end{split} \\
        &= \frac{3}{2}\left(1+\frac{1}{2}-\frac{1}{n+1}-\frac{1}{n+2}\right) \\
        &= \frac{9}{4} - \frac{3(2n+3)}{2(n+1)(n+2)}
      }
  \item
    \flan{
        \sum_{k=1}^{n}\cfrac{3}{\sqrt{k+2} + \sqrt{k}}
        &= 3 \sum_{k=1}^{n}\frac{\sqrt{k+2}-\sqrt{k}}{(\sqrt{k+2}+\sqrt{k})(\sqrt{k+2}-\sqrt{k})} \\
        &= 3 \sum_{k=1}^{n}\frac{\sqrt{k+2}-\sqrt{k}}{2} \\
        \begin{split}
          &= \frac{3}{2}\bigl\{
            \bigl(\sqrt{3}-\sqrt{1}\bigr)
            + \bigl(\sqrt{4}-\sqrt{2}\bigr)
            + \cdots \bigr. \\
          &\bigl.\hspace*{50pt}
              + \bigl(\sqrt{n+1}-\sqrt{n-1}\bigr)
              + \bigl(\sqrt{n+2}-\sqrt{n}\bigr)
            \Bigr\} \\
        \end{split} \\
        &= \frac{3}{2}\left( \sqrt{n+2} + \sqrt{n+1} - 1 - \sqrt{2} \right)
      }
  \item
    \flan{
        S_n = \sum_{k=1}^{n}\{(2k-1)\cdot 2^{k-1}\}
      }
    とおく。
{\newcommand{\dl}[1]{\raisebox{-2pt}{$#1$}}
    このとき
      \flan{
          &\begin{matrix}
            & S_n  & = & 1\cdot1             & + \hspace*{2pt} 3\cdot2 \hspace*{2pt} + \hspace*{2pt} 5\cdot2^2 &+& \cdots &+& (2n-1)\cdot2^{n-1} &                  \\
         -) & 2S_n & = & \hphantom{1\cdot 1} & + \hspace*{2pt} 1\cdot2 \hspace*{2pt} + \hspace*{2pt} 3\cdot2^2 &+& \cdots &+& (2n-3)\cdot2^{n-1} & \hspace*{-6pt}+ \,(2n-1)\cdot2^n \\
         \hline
            & \dl{-S_n} &\dl{=}& \dl{1\cdot1}   & \dl{+ \hspace*{2pt} 2\cdot2 \hspace*{2pt} + \hspace*{2pt} 2\cdot2^2} &\dl{+}& \dl{\cdots} &\dl{+}& \dl{2\cdot2^{n-1}} & \hspace*{-6pt}\dl{-\,(2n-1)\cdot2^n} \\
           \end{matrix} \\
           &\Leftrightarrow -S_n = 1 + 2^2 + 2^3 + \cdots + 2^{n-1} + 2^n - (2n-1)\cdot2^n \\
           &\Leftrightarrow -S_n = 1 + \frac{4(2^{n-1}-1)}{2-1} - (2n-1)\cdot2^n \\
           &\Leftrightarrow -S_n = 1 + 2^{n+1} - 4 - (2n-1)\cdot2^n \\
           &\Leftrightarrow  S_n = (2n-1-2)\cdot2^n + 3 \\
           &\therefore S_n = (2n-3)\cdot2^n + 3
        }
}

  \item
          \flan{
              &\frac{1}{1} + \frac{1}{1+2} + \frac{1}{1+2+3} + \cdots + \frac{1}{1+2+3+\cdots +(n-1)+n} \\
              &= \sum_{k=1}^{n}\frac{1}{\cfrac{1}{2}k(k+1)} \\
              &= \sum_{k=1}^{n}2\Bigl(\frac{1}{n} - \frac{1}{n+1}\Bigr) \\
              &= 2\tm\Bigl(\frac{1}{1} - \frac{1}{2} + \frac{1}{2} - \cdots + \frac{1}{n} - \frac{1}{1+n}\Bigr) \\
              &= \frac{2n}{n+1}
            }
  \item \dm{S_n = 1+\frac{2}{3}+\frac{3}{3^2}+\cdots+\frac{n}{3^{n-1}}}とおくと、\dm{\frac{S_n}{3}}との差を考えて
          \flan{
              &1+\frac{2}{3}+\frac{3}{3^2}+\cdots+\frac{n}{3^{n-1}} \\
              &= \cdots = \frac{9}{4} - \frac{2n+3}{4\cdot 3^{n-1}}
            }
\end{enumerate}
\end{ans*}
\end{question*}


\newpage
\section{漸化式 基本編}

漸化式の問題はしばしば誘導がされます。
そのため、何も誘導がなければ好きな方法で解けばいいですが、
誘導があったりするとそれに乗らなくてはならないということです。
それぞれのパターンの漸化式に対して別解がある場合は抑えておき、
様々な問題に対応できるようにしておきましょう。

\begin{question*}
次の式を満たす数列$\{a_n\}$の一般項を$n$を用いて表せ。
\begin{ans*}
${}$
\begin{enumerate}[label=\arabic*.]
  \item $a_1 = 2,\; a_{n+1} = a_n + 3 $\tousa
  \flan{
    a_n
    &= 2 + (n-1) \times 3 \\
    &= 3n - 1
  }
  \item $a_1 = 5,\; a_{n+1} = 7a_n$ \touhi
  \flan{
    a_n
    &= 5\cdot 7^{n-1}
  }
  \item $a_1 = 3,\; a_{n+1} = 3a_n -4$ \tokusyukai \\
  特性方程式$\gra = 3\gra -4 \Leftrightarrow \gra = 2$より
  $a_{n+1} - 2 = 3(a_n - 2)$ \touhi と変形できるので
  \flan{
    &a_n - 2 \\
    &= (a_1 - 2) \times 3^{n-1} \\
    &= 3^{n-1}
  }
  よって、
  \flan{
    a_n = 3^{n-1} + 2
  }
  \begin{supple*}
    慣れない間は$b_n = a_n - 2$とおく習慣をつけるといいです。
  \end{supple*}

  \item $a_1 = 1,\; a_{n+1} = a_n + 2n$ \kaisa \\
  $n\geqq 2$のとき
  \flan{
    a_n
    &= 1 + 2\sum_{k=1}^{n-1} k \\
    &= n^2 - n + 1
  }
  これは$n = 1$のときも成り立つ。
  \item $a_1 = 1,\; a_{n+1} = a_n + 3^n - 4n$\kaisa \\
  $n\geqq 2$のとき
  \flan{
    a_n
    &= 1 + \sum_{k=1}^{n-1} (3^k-4k) \\
    &= 1 + \frac{3^{n} -3}{3-1} - \frac{1}{2} (n-1) n \\
    &= \frac{1}{2}\cdot 3^{n} - 2n^2 + 2n - \frac{1}{2}
  }
  これは$n = 1$のときも成り立つ。
  \item $a_1 = 1,\; a_2 = 2,\; a_{n+2} = a_{n+1} + 6a_n$ \sankoukan \\
  特性方程式$\gra^2 = \gra + 6 \Leftrightarrow \gra = -2,3$より、
  \flan{
    &\begin{dcases*}
      a_{n+2} + 2a_{n+1} = 3(a_{n+1} + 2a_n)  \:\: \touhi \\
      a_{n+2} - 3a_{n+1} = -2(a_{n+1} - 3a_n) \:\: \touhi
    \end{dcases*}\\
    &\Rightarrow
    \begin{dcases*}
      a_{n+1} + 2a_n = 4\cdot 3^{n-1}\\
      a_{n+1} - 3a_n = -(-2)^{n-1}
    \end{dcases*}
  }
  この2式より、
  \flan{
    a_n = \frac{4\cdot 3^{n-1} + (-2)^{n-1}}{5}
  }
  \begin{other*}
    特性方程式より
    \flan{
      a_{n+2} + 2a_{n+1} = 3(a_{n+1} + 2a_n)
    }
    よって、
    \flan{
      &\begin{aligned}
        a_{n+1} + 2a_n
        &= (a_2 + 2a_1)\cdot 3^{n-1} \\
        &= 4\cdot3^{n-1} \\
      \end{aligned} \\
      &\Leftrightarrow a_{n+1} = -2a_n + 4\cdot3^{n-1}
    }
    $3^{n+1}\neq 0$より、
    \flan{
      \frac{a_{n+1}}{3^{n+1}} = -\frac{2}{3}\frac{a_n}{3^n} + \frac{4}{9} \:\:\touhi
    }
    すなわち、
    \flan{
      & \frac{a_{n+1}}{3^{n+1}} - \frac{4}{15} = -\frac{2}{3}\left(\frac{a_n}{3^n} - \frac{4}{15} \right) \\
      & \frac{a_n}{3^n} - \frac{4}{15} = \frac{1}{15} \cdot \left( -\frac{2}{3} \right)^{n-1} \\
      & a_n
      = \left( \frac{4}{15} + \frac{1}{15} \cdot \left( -\frac{2}{3} \right)^{n-1} \right) \cdot 3^n
      = \frac{4\cdot3^{n-1} + (-2)^{n-1}}{5}
    }
  \end{other*}


  \item  $a_1 = 2,\; a_{n+1} = 16a_n^5$ \jisuusoui \\
  与えられた漸化式より$a_n> 0$より、両辺に底を2として対数をとると、
  \flan{
    \log_2 a_{n+1} = 5\log_2 a_n + 4
  }
  ここで、$b_n = \log_2 a_n$とおくと$b_1 = 1$であるので
  \flan{
    &b_{n+1} = 5b_n + 4\\
    &b_{n+1} + 1 = 5(b_n + 1) \:\: \touhi \\
    &b_n = 2\cdot 5^{n-1} -1
  }
  よって、
  \flan{
    a_n = 2^{2\cdot 5^{n-1} -1}
  }
  \item $a_1 = 1,\; a_{n+1} = 2a_n + n^2 -6$ \\
  この漸化式が、$a_{n+1} + p(n+1)^2 + q(n+1) + r = 2(a_n + pn^2 + qn + r)$と
  表せるとすると、これを展開した$a_{n+1} = 2a_n + pn^2 + (-2p + q)n + (-p-q+r)$と
  与えられた漸化式の係数は常に等しいので(恒等式)
  \flan{
    &\begin{dcases*}
        p = 1\\
      -2p +q =0 \\
      - p -q +r =-6
    \end{dcases*}
    \\
    &\therefore (p,q,r) = (1\;,\;2\;,\;-3)
  }
  ゆえに
  \flan{
    &a_{n+1} + (n+1)^2 + 2(n+1) - 3
            = 2( a_n + n^2 + 2n - 3 ) \\
    & a_n + n^2 + 2n - 3 = (a_1 + 1 + 2 - 3 )\cdot 2^{n-1} = \cdot 2^{n-1} \\
    & \therefore a_n = 2^{n-1} - n^2 - 2n + 3
  }

  \vskip.3\baselineskip
  この問題は上のような解法が最もよいと思われるが、次の誘導がされる場合がある。
  \begin{practice*}
    $b_n = a_{n+1} - a_n$ とおいたときの$b_n$の漸化式を導き、さらに
    $c_n = b_{n+1} - b_n$とおいたときの$c_n$を求めることで$a_n$の一般項を求めよ。 \\
    \begin{ans*}
      与えられた漸化式より$a_{n+2} = 2a_{n+1} + (n+1)^2 - 6$であるので、与式とこの式の差をとって
      \flan{
        &a_{n+2} - a_{n+1} = 2(a_{n+1} - a_n) + 2n +1 \\
        & b_{n+1} = 2b_n + 2n + 1 \:\: (b_1 = -4,\: b_2 = -5)
      }
      さらに、$b_{n+2} = 2b_{n+1} + 2(n+1) + 1$であるので、上式とこの式の差をとって
      \flan{
        &b_{n+2} - b_{n+1} = 2(b_{n+1} - b_n) + 2 \\
        &c_{n+1} = 2c_n + 2 \:\: (c_1 = -1) \\
        & \Leftrightarrow c_{n+1} + 2 = 2(c_n + 2) \:\: \touhi
      }
      すなわち、
      \flan{
        &c_n = 2^{n-1} - 2
      }
      また、$c_n$は$b_n$の階差数列より、
      \flan{
        &b_n = b_1 + \sum_{k=1}^{n-1}c_k \:\:(n\geqq 2) \\
        &\Rightarrow b_n = 2^{n-1} - 2n - 3 \:\:(\text{これは$n=1$でも成り立つ})
      }
      $a_n$についても、階差数列$b_n$が上のように得られるので、
      \flan{
        &a_n = a_1 + \sum_{k=1}^{n-1}b_k \:\:(n\geqq 2) \\
        &\therefore a_n = 2^{n+1} - n^2 - 2n + 3 \:\:(\text{これは$n=1$でも成り立つ})
      }
    \end{ans*}
  \end{practice*}

  \item $a_1 = 1,\; a_{n+1} = 4a_n + n\cdot 2^n$ \sisu \\
  両辺を$2^{n+1}$でわると
  \flan{
    \frac{a_{n+1}}{2^{n+1}} = 2 \frac{a_n}{2^n} + \frac{n}{2}
  }
  これが$\disp\frac{a_{n+1}}{2^{n+1}} + p(n+1) + q = 2 \biggl( \frac{a_n}{2^n} + pn + q \biggr) $
  と表せるとすると、これを展開した\\
  $\disp \frac{a_{n+1}}{2^{n+1}}=2\frac{a_n}{2^n} + pn + (-p + q)$
  と与えられた漸化式の係数は常に等しいので
  \flan{
    (p,q) = \left(\frac{1}{2}\;,\;\frac{1}{2}\right)
  }
  すなわち
  \flan{
    \frac{a_{n+1}}{2^{n+1}} + \frac{1}{2}(n+1) + \frac{1}{2}
    = 2 \left( \frac{a_n}{2^n} + \frac{1}{2}n + \frac{1}{2} \right) \:\: \touhi
  }
  ゆえに
  \flan{
    &\frac{a_n}{2^n}+\frac{1}{2}n+\frac{1}{2} = \frac{3}{2}2^{n-1} \\
    &a_n =2^n \left( 3\cdot 2^{n-2} - \frac{n}{2} - \frac{1}{2} \right) \\
    &\therefore a_n = 3\cdot 2^{2n-2} - (n+1)2^{n-1}
  }

  \item $a_1 = 1,\; a_2=3,\; a_{n+2} = 4a_{n+1} - 4a_n$\sankoukanjuukai \\
  特性方程式$\gra^2 = 4\gra - 4 \Leftrightarrow \gra = 2$より
  \flan{
    a_{n+2} - 2a_{n+1} = 2(a_{n+1} - 2a_n) \:\: \touhi
  }
  よって、
  \flan{
    a_{n+1} - 2a_n = 2^{n-1}
  }
  $-2a_n$を移項し、両辺を$2^{n+1}$でわると
  \flan{
    &\frac{a_{n+1}}{2^{n+1}} = \frac{a_n}{2^n} + \frac{1}{4} \:\: \tousa \\
    &\frac{a_n}{2^n} = \frac{1}{2} + \frac{1}{4}(n-1) = \frac{1}{4}(n+1) \\
    & \therefore a_n = (n+1)2^{n-2}
  }

  \item $a_1 = 1,\; a_2 = 4,\; a_{n+2} = 4a_{n+1} - 3a_n - 2$ \sankoukanteisuukou \\
  特性方程式$\gra^2 = 4\gra -3 \Leftrightarrow \gra = 1,3$より
  \flan{
    &\begin{dcases*}
      a_{n+2} - a_{n+1} = 3(a_{n+1} - a_n) - 2  \\
      a_{n+2} - 3a_{n+1} = (a_{n+1} - 3a_n) - 2
    \end{dcases*}
  }
  また、$b_n = a_{n+1} - a_n,\:c_n = a_{n+1} - 3a_n$とおくと$b_1 = 3,\:c_1 = 1$であり
  \flan{
    &\begin{dcases*}
      b_{n+1} = 3b_n - 2 \\
      c_{n+1} = c_n -2
    \end{dcases*}
    \\
    &\Leftrightarrow
    \begin{dcases*}
      b_{n+1} - 1 = 3(b_n - 1) \:\: \touhi \\
      c_{n+1} = c_n - 2  \:\: \tousa
    \end{dcases*}
    \\
    &\Leftrightarrow
    \begin{dcases*}
      b_n = a_{n+1} - a_n = 2\cdot 3^{n-1} + 1 \\
      c_n = a_{n+1} - 3a_n = -2n + 3
    \end{dcases*}
  }
  この2式から$\disp a_n = \frac{b_n - c_n}{2}$であるので一般項は
  \flan{
    a_n
    = \frac{2\cdot 3^{n-1} + 2n - 2}{2}
    = 3^{n-1} + n - 1
  }

  \item $\disp a_1 = 1,\: a_{n+1} = \frac{a_n}{2a_n + 3}$ \bunsugyakusutikan \\
  ある$k\in \mathbb{N}$に対して$a_k = 0$であると仮定すると
  \flan{
    0 = a_k = \frac{a_{k-1}}{2a_{k-1} + 3} \Leftrightarrow a_{k-1} = 0
  }
  であり、これを繰り返すことで$0 = a_{k-1} = a_{k-2} = \cdots = a_1$となるが、これは$a_1 = 1$に矛盾。\\
  よって、任意の自然数$n$で$a_n \neq 0$であるので、$\disp b_n = \frac{1}{a_n}$とおくと、$b_1 = 1$で
  \flan{
    &\begin{aligned}
      b_{n+1}
      &= \frac{1}{a_{n+1}} \\
      &= \frac{2a_n + 3}{a_n} \\
      &= \frac{3}{a_n} + 2 \\
      &= 3b_n + 2 \\
    \end{aligned} \\
    &\Leftrightarrow b_{n+1} + 1 = 3(b_n + 1)
  }
  すなわち、
  \flan{
    &b_n = 2\cdot 3^{n-1} - 1 \\
    & \Rightarrow a_n = \frac{1}{2\cdot 3^{n-1} - 1}
  }

  \item $\disp a_1 = 3,\; a_{n+1} = \frac{3a_n-4}{a_n-2}$ \bunsutokusei \\
  特性方程式$\disp \gra =\frac{3\gra - 4}{\gra - 2} \Leftrightarrow \gra = 1,4$より \\
  ここで、ある$k\in \mathbb{N}$に対して、$a_k = 1$であると仮定すると
  \flan{
    a_k = 1 = \frac{3a_{k-1} - 4}{a_{k-1} - 2} \Leftrightarrow a_{k-1} = 1
  }
  であり、これを繰り返すことで$1 = a_k = a_{k-1} = \cdots = a_1$となるが、これは$a_1 = 3$に矛盾。\\
  よって、任意の自然数$n$で$a_n \neq 1$であるので、
  $\disp b_n = \frac{a_n - 4}{a_n - 1}$とおくと$\disp b_1 = -\frac{1}{2}$で
  \flan{
    b_{n+1}
    &= \frac{3a_{n+1} - 4}{a_{n+1} - 1}
    = \cfrac{\cfrac{3a_n-4}{a_n-2}-4}{\cfrac{3a_n-4}{a_n-2}-1} \\
    &= -\frac{1}{2}b_n \:\: \touhi
  }
  よって、$\disp b_n = \biggl(-\frac{1}{2}\biggr)^n$であり、$b_n \neq 1$ \\
  また、
  \flan{
    b_{n} = \frac{a_n - 4}{a_n - 1} \Rightarrow a_n = \frac{b_n - 4}{b_n - 1} \:\:(\because b_n \neq 1)
  }
  であるので、
  \flan{
    a_n
    = \frac{\biggl(-\cfrac{1}{2}\biggr)^n - 4}{\biggl(-\cfrac{1}{2}\biggr)^n - 1}
    = \frac{2^{n+2} - (-1)^n}{2^n - (-1)^n}
  }

  \item $\disp a_1 = 3,\; a_{n+1} = \frac{3a_n-4}{a_n-1}$ \bunsujuukai \\
  特性方程式$\disp \gra =\frac{3\gra - 4}{\gra - 1} \Leftrightarrow \gra = 2$より \\
  ここで、ある$k\in \mathbb{N}$に対して、$a_k = 2$であると仮定すると、
  \flan{
    a_k = 2 = \frac{3a_{k-1} - 4}{a_{k-1} - 1} \Leftrightarrow a_{k-1} = 2
  }
  であり、これを繰り返すことで$2 = a_k = a_{k-1} = \cdots = a_1$となるが、これは$a_1 = 3$に矛盾。\\
  よって、任意の自然数$n$で$a_n \neq 2$であるので、$\disp b_n = \frac{1}{a_n - 2}$とおくと$b_1 = 1$で
  \flan{
    b_{n+1}
    &= \frac{1}{a_{n+1} - 2}
      = \frac{1}{\cfrac{3a_n - 4}{a_n - 1} - 2} \\
    &= \frac{a_n - 1}{a_n - 2} \\
    &= b_n + 1 \:\: \tousa  \\
    &\Rightarrow b_n = 1 + (n - 1) \cdot 1 = n
  }
  また、任意の自然数$n$で$b_n \neq 0 \not\equiv 0$から
  \flan{
    b_n = \frac{1}{a_n - 2} \Rightarrow a_n = \frac{1}{b_n} + 2
  }
  であるので
  \flan{
    a_n = \frac{1}{n} + 2 \left( = \frac{2n + 1}{n} \right)
  }

  \begin{supple*}
    分数型の漸化式の解き方について簡単におさらいしましょう。\\
    $\disp a_{n+1} = \frac{pa_n + q}{ra_n + s}$について特性方程式$\disp\gra = \frac{p\gra + q}{r\gra + s}$
    を考え、その2解を$\grl_1,\grl_2$とおいたときに
    \flan{
      b_n = \frac{a_n - \grl_1}{a_n - \grl_2}
    }
    あるいは
    \flan{
      b_n = \frac{1}{a_n - \grl}
    }
    とおくことで漸化式を解くことができました。 \\
    さて、分数型での$q=0$である場合に$a_n$の逆数を置き換えたのは
    特性方程式の解のうちいずれか一つは必ず$0$であるからです。
    \flan{
      \gra = \frac{p\gra}{r\gra + s} \Leftrightarrow \gra = 0,\,\frac{p-s}{r}
    }
    要するに、上の置き換えにおける$\grl = 0$の場合であると思えばよいということです。
    逆に、逆数置換をするということはあくまで特性方程式の解の一つを用いて
    $\grl = 0$としたということに過ぎないわけですから
    $\disp\grl = \frac{p-s}{r}$のようにもう一つの解を置いても問題なく解くことはできます。
    加えて、$\disp b_n = \frac{1}{a_n - \grl}$とおくパターンは特性方程式が重解の場合も
    解けるということですから、重解でない場合にも同じようにこの置換で解くことができます。

    いずれの場合についても実際に自分で手を動かして解いてみてください。
  \end{supple*}

  \item $a_1 = 1,\; a_{n+1}=(n+1)a_n$ \kaihi \\
  両辺を$(n+1)!\neq 0$でわると
  \flan{
    \frac{a_{n+1}}{(n+1)!} = \frac{a_n}{n!} \:\: \touhi
  }
  よって
  \flan{
    &\frac{a_n}{n!} = \frac{a_1}{1!} = 1 \\
    &\therefore a_n = n!
  }
  \item $a_1 = 1,\; (n+2)a_{n+1}=na_n$ \kaihi \\
  両辺を$n+1$倍すると
  \flan{
    (n+2)(n+1)a_{n+1} = (n+1)n\;a_n \:\: \touhi
  }
  よって
  \flan{
    &(n+1)na_n = (1+1)\cdot 1 a_1 = 2 \\
    &\therefore a_n = \frac{2}{n(n+1)}
  }
  \item $a_1 = 1,\; na_{n+1}=2(n+1)a_n+n(n+1)$ \kaihitousa \\
  両辺を$n(n+1)\neq 0$でわると
  \flan{
    &\frac{a_{n+1}}{n+1} = 2\frac{a_n}{n} + 1 \\
    &\frac{a_{n+1}}{n+1} + 1 = 2\left( \frac{a_n}{n} + 1 \right) \:\: \touhi
  }
  よって
  \flan{
    &\frac{a_n}{n} + 1 = 2\cdot 2^{n-1} =2^n \\
    &\therefore a_n = n (2^n-1)
  }
  \item $\disp a_1 = 2,\; a_{n+1} = \frac{n+2}{n}a_n + 1$ \\
  両辺を$(n+1)(n+2)\neq 0$でわると
  \flan{
    \frac{a_{n+1}}{(n+1)(n+2)} = \frac{a_n}{n(n+1)}
  }
  より$\disp b_n = \frac{a_n}{n(n+1)}$とおくと$b_1=1$であり、
  \flan{
    b_{n+1}
    &= b_n + \frac{1}{(n+1)(n+2)} \:\: \kaisa
  }
  である。よって$n\geqq 2$のとき
  \flan{
    b_n
    &= b_1 + \sum_{k=1}^{n-1}\frac{1}{(k+1)(k+2)} \\
    &= 1 + \sum_{k=1}^{n-1}\left( \frac{1}{k+1}-\frac{1}{k+2} \right) \\
    &= 1 + \frac{1}{2} - \frac{1}{n+1} \\
    &= \frac{3n+1}{2(n+1)}
  }
  また、$n=1$のとき\dm{b_1 = \frac{3\cdot1 + 1}{2(1+1)} = 1}より、$n=1$のときもこれは成り立つ。
  以上より、
  \flan{
    a_n
    &= n(n+1)b_n \\
    &= \frac{n(3n+1)}{2}
  }

  \item $a_1 = 1,\; a_{n+1} = 2^{2n-2}(a_n)^2$ \\
  以下、数列$a_n$が任意の自然数$n$に対して$a_n>0$であることを示す。
  \begin{enumerate}[label=(\roman*)]
    \item $n=1$のとき$a_1=1>0$
    \item $n=k \:(k\in \mathbb{N})$のとき、$a_n>0$と仮定すると$a_k>0\quad \cdots (1)$\\
    また、
    \flan{
      a_{k+1} = 2^{2k-2}(a_k)^2 > 0
    }
    より$n=k+1$のときも成り立つ。
  \end{enumerate}
  (i),\, (ii)より数学的帰納法からすべての自然数$n$で$a_n>0$である。\\
  漸化式の両辺に底を2とする対数をとると、
  \flan{
    \log_2a_{n+1} = (2n-2)+2\log_2a_n
  }
  である。$b_n = \log_2a_n$とおくと、$b_1=0$であり$b_{n+1} = 2b_n + 2n - 2$となるが\\
  これが$b_{n+1} + p(n+1) + q = 2(b_n + pn + q)$と表せるとすると、\\
  これを展開した$b_{n+1} = 2b_n + pn + (-p+q)$と係数が等しいので
  \flan{
    &
    \begin{dcases*}
      p=2 \\
      q-p=-2
    \end{dcases*}
    \\
    &\Rightarrow
    \begin{dcases*}
      p = 2 \\
      q = 0
    \end{dcases*}
  }
  より
  $b_{n+1} + 2(n+1) = 2(b_n + 2n) \:\: \touhi$であるので、
  \flan{
    & b_n + 2n = (b_1 + 2) \cdot 2^{n-1} = 2^n \\
    & \Rightarrow b_n = 2^n - 2n \\
    & \Rightarrow a_n = 2^{2^{n}-2n}
  }

  \item $S_n = 3a_n + 2n - 1$ \\
  $S_{n+1} = 3a_{n+1} + 2n + 1$および$S_{n+1}=S_n + a_{n+1}$より
  \flan{
    &S_{n+1} - S_n = a_{n+1} =  3a_{n+1} - 3a_n + 2 \\
    &\Rightarrow a_{n+1} = \frac{3}{2}a_n - 1 \\
    &a_{n+1} - 2 = \frac{3}{2}(a_n - 2) \\
    &\therefore a_n = \frac{5}{2}\cdot\biggl(\frac{3}{2}\biggr)^{n-1}+2
  }

\end{enumerate}\end{ans*}\end{question*}

\clearpage
\section{漸化式 演習編}

\begin{enumerate}[label=\arabic*.]
  \item
  $n$を自然数として次の条件で定められた数列$\{a_n\}$について2通りの解き方を考えよう。 % (福井大 改 +α)

    \begin{align*}
      a_1 = 1,\:\: a_{n+1} = \frac{3}{n} (a_1 + a_2 + a_3 + \cdots + a_n)  \quad\cdots (*)
    \end{align*}

  \begin{enumerate}[label=(\arabic*)]
    \item $a_2,\;a_3\;a_4$を計算せよ。
    \flan{a_2 = 3,\: a_3 = 6,\:a_4 = 10}
    \vskip.5\baselineskip
    \item 一般項$\{a_n\}$を推定し、それが正しいことを数学的帰納法を用いて示せ。
    %* 推定のコツ:素因数分解、定数倍してみる
    %* 帰納法の種類:普通・前二つ仮定・それより前全部仮定(・無限降下法)
    \flan{
      &2\times a_1 = 2 = 1\times 2\\
      &2\times a_2 = 6 = 2\times 3 \\
      &2\times a_3 = 12 = 3\times 4 \\
      &2\times a_4 = 20 = 4\times 5
    }
    であるので$\disp a_n = \frac{n(n+1)}{2}\cdots (\rm{I})$であると推定できる。 \\
    \begin{enumerate}[label=(\roman*)]
      \item $n = 1$のとき
      \flan{
        a_1 = 1 = \frac{2\times 1}{2}
      }
      より、これは(1)を満たす。
      \item ある自然数$k$に対して、$n \leqq k$で(1)が成り立つとき、
      $\disp a_i = \frac{i(i+1)}{2}\quad(i\leqq k) \:\: \cdots (\rm{II})$である。\\
      また、与えられた漸化式$(*)$を用いると
      \flan{
        a_{k+1}
        &= \frac{3}{k}(a_1 + \cdots + a_k) \\
        &= \frac{3}{k}\sum_{i=1}^{k}a_i \\
        &= \frac{3}{k}\sum_{i=1}^{k}\frac{i(i+1)}{2} \quad (\because (\rm{II}))\\
        &= \frac{3}{2k}\sum_{i=1}^{k}(i^2 + i) \\
        &= \frac{3}{2k}\left( \frac{k(k+1)(2k+1)}{6} + \frac{k(k+1)}{2} \right) \\
        &= \frac{k^2+3k+2}{2} \\
        &= \frac{(k+1)(k+2)}{2}
      }
      より、$n = k + 1$でも$(\rm{I})$が成り立つ。
    \end{enumerate}

    (i),(ii)より数学的帰納法からすべての自然数$n$に対して$a_n = \cfrac{n(n+1)}{2}$である。
    \vskip1\baselineskip
    \begin{other*}
      普通の数学的帰納法で示すパターンです。 \\
      (ii)\:\: ある自然数$k$に対して、$(\rm{I})$が成り立つとき
      \flan{
        a_k = \frac{3}{k-1}(a_1 + a_2 + \cdots + a_{k-1}) = \frac{k(k+1)}{2} \quad \cdots (2)
      }
      であるので、漸化式$(*)$は
      \flan{
        a_{k+1}
        &= \frac{3}{k}(a_1 + a_2 + \cdots + a_k) = \frac{3}{k}(a_1 + a_2 + \cdots + a_{k-1}) + \frac{3a_k}{k} \\
        &= \frac{k-1}{k}\left\{ \frac{3}{k-1}(a_1 + a_2 + \cdots + a_{k-1}) \right\} + \frac{3a_k}{k} \\
        &= \frac{k-1}{k}\cdot a_k + \frac{3a_k}{k} \qquad (\because (2))\\
        &= \frac{k+2}{k}a_k = \frac{k+2}{k}\cdot \frac{k(k+1)}{2} \\
        &= \frac{(k+1)(k+2)}{2}
      }
      より、$n = k + 1$でも$(\rm{I})$が成り立つ。
    \end{other*}

    \vskip1\baselineskip
    \begin{supple*}
      数学的帰納法は主に次の3パターンがあります。
      \vskip.5\baselineskip
      \begin{enumerate}[label=(\roman*)]
        \item $n=1$で示して「$n=k$で成り立つ$\Rightarrow$$n=k+1$で成り立つ」を示す。
        \item $n=1$と$n=2$で示して「$n=k,k+1$で成り立つ$\Rightarrow$$n=k+2$で成り立つ」を示す。
        \item $n=1$で示して「$n\leqq k$で成り立つ$\Rightarrow$$n=k+1$で成り立つ」を示す。
      \end{enumerate}
      \vskip.5\baselineskip
      上の問題では(i),(iii)を用いて示す方法を述べたため、ついでに(ii)の使い所も示しておきましょう。

      \begin{practice*}
        $x,y\in \mathbb{R}$について、$x+y,xy$がいずれも偶数であるとする。
        このとき、$n\in \mathbb{N}$に対して$x^n+y^n$も偶数となることを示せ。
        \begin{ans*}
          $x+y$,$xy$が偶数であるので$l,m\in \mathbb{Z}$を用いて
          \flan{
            x+y = 2l,\, xy = 2m
          }
          と表せる。
          \begin{enumerate}[label=(\roman*)]
            \item
              $n=1$のとき
              $x^1+y^1=x+y=2l$より$x^n+y^n$は偶数である。
            \item
              $n=2$のとき
              $x^2+y^2=(x+y)^2-2xy=4l^2-4m=2(2l^2-2m)$より、 \\
              $x^n+y^n$は偶数である。
            \item
              $n=k,\,k+1\:(k\in \mathbb{N})$のときに成り立つと仮定すると
              \flan{
                &x^k + y^k = 2x_k \:\:(x_k \in \mathbb{Z})\\
                &x^{k+1} + y^{k+1} = 2x_{k+1} \:\:(x_{k+1} \in \mathbb{Z})\\
              }
              である。このとき、
              \flan{
                x^{k+2} + y^{k+2}
                &= (x+y)(x^{k+1} + y^{k+1}) - xy^{k+1} - x^{k+1}y \\
                &= 2l \cdot2x_{k+1} - 2m\cdot x_k \\
                &= 2(2lx_{k+1} - mx_{k})
              }
              より$n=k+2$のときも偶数。
          \end{enumerate}
          (i),(ii),(iii)より数学的帰納法から$x+y,xy$がいずれも偶数であるとき、
          すべての自然数$n$に対して$x^n+y^n$も偶数である。
        \end{ans*}
      \end{practice*}
    \end{supple*}

    \item 上の漸化式$(*)$について、$a_1 + a_2 + a_3 + \cdots + a_{n-1}$を$a_n$と$n$を用いて表せ。\\
    $(*)$より
      \flan{
        &a_n = \frac{3}{n-1}(a_1 + a_2 + \cdots + a_{n-1}) \\
        &\Rightarrow a_1 + a_2 + \cdots + a_{n-1} = \frac{(n-1)\;a_n}{3}
      }

    \item $a_{n+1}$と$a_n$の関係を導いた上で、一般項$a_n$を$n$を用いて表せ。\\
    (3)より、$(*)$は
    \flan{
      a_{n+1}
      &= \frac{3}{n}(a_1 + a_2 + \cdots + a_{n-1}) + \frac{3}{n}a_n \\
      &= \frac{3}{n}\frac{(n-1)\;a_n}{3} + \frac{3}{n}a_n \\
      &= \frac{n+2}{n}a_n
    }
    両辺を$(n+1)(n+2)$でわることにより
    \flan{
      &\frac{a_{n+1}}{(n+2)(n+1)} = \frac{a_n}{(n+1)n} \:\: \touhi \\
      & \Rightarrow \frac{a_n}{(n+1)n} = \frac{a_1}{2\times 1} = \frac{1}{2} \\
      & \therefore a_n = \frac{n(n+1)}{2}
    }
  \end{enumerate}

\clearpage

  \item
  \begin{enumerate}[label=(\arabic*)]
    \item 次の初項、二つの漸化式で与えられる数列$\{a_n\},\, \{b_n\}$を考える。
      \begin{align*}
        &a_1 = 5,\quad b_1 = 3\\
        &\begin{dcases*}
          a_{n+1} = 5a_n + 3b_n \cdots (\rm{i}) \\
          b_{n+1} = 3a_n + 5b_n \cdots (\rm{ii})
        \end{dcases*}
      \end{align*}
    2つの数列$\{a_n\pm b_n\}$を求め、一般項$a_n,\, b_n$を求めよ。\\

    (i)、(ii)式を足し合わせたものと引いたものを考えると
      \flan{
          \begin{dcases*}
            a_{n+1} + b_{n+1} = 8(a_n + b_n) \\
            a_{n+1} - b_{n+1} = 2(a_n - b_n) \:\: \touhi
          \end{dcases*}
        }
    となる。$a_1+b_1 = 8$、$a_1-b_1 = 2$より、
      \flan{
          \begin{dcases*}
          a_n + b_n = 8^n \\
          a_n - b_n = 2^n
          \end{dcases*}
        }
    この2式から
      \flan{
          \begin{dcases*}
          a_n = \frac{8^n+2^n}{2} \\
          b_n = \frac{8^n-2^n}{2}
          \end{dcases*}
        }


    \item (1)を踏まえて次の初項、二つの漸化式で与えられる数列$\{p_n\},\, \{q_n\}$の一般項をそれぞれ求めよ。% (大阪医科大学)
      \begin{align*}
        &p_1 = 1,\quad q_1 = 4\\
        &\begin{dcases*}
          p_{n+1} = 2p_n + q_n \\
          q_{n+1} = 4p_n - q_n
        \end{dcases*}
      \end{align*}
      % ヒント:
      % \ $\{p_n + t q_n\}$が等比数列となるような$t$を二つ求める。
      % すなわち、\ $p_{n+1} + tq_{n+1} = s(p_n + t q_n) $を満たす$s,t$の組を二つ見つける。
      数列$\{p_n + t q_n\}$が等比数列となるとき、この漸化式は
      \flan{
          p_{n+1} + tq_{n+1}
          &= (2 + 4t)p_n + (1 - t)q_n \\
          &= (2 + 4t)\biggl(p_n + \frac{1-t}{2+4t}q_n\biggr)
        }
      であるので
      \flan{
          t = \frac{1 - t}{2 + 4t} \\
          \therefore t = -1 ,\,\frac{1}{4}
        }
      よって、数列$\{p_n + t q_n\}$に関する漸化式は
      \flan{
          \begin{dcases*}
            p_{n+1} - q_{n+1} = -2(q_n + q_n) \\
            p_{n+1} + \frac{1}{4}q_{n+1} = 3\biggl(p_n + \frac{1}{4}q_n\biggr)
          \end{dcases*}
        }
      と書ける。
      $p_1 = 1,\,q_1 = 4$より、
      \flan{
          \begin{dcases*}
            p_n - q_n
            = (p_1 - q_1) (-2)^{n-1}
            = -3(-2)^{n-1} \\
            p_n + \frac{1}{4}q_n
            = \biggl(p_1 + \frac{1}{4}q_1\biggr)\cdot 3^{n-1}
            = 2\cdot 3^{n-1}
          \end{dcases*}
        }
      であるので、差をとって
      \flan{
          p_n + \frac{1}{4}q_n &  - (p_n - q_n) \\
          &= \frac{5}{4}q_n \\
          &= 2\cdot 3^{n-1} + 3(-2)^{n-1} \\
        }
      ゆえに、
      \flan{
          q_n &= \frac{8\cdot 3^{n-1} + 12(-2)^{n-1}}{5} \\
          p_n &= 3(-2)^{n-1} + q_n = \frac{8\cdot 3^{n-1} - 3 (-2)^{n-1}}{5}
        }
      \begin{other*}
        $\{p_{n}\},\,\{q_{n}\}$のうち、一方を消去する方針でも解けます。

        与えられた漸化式の第一式より
        \flan{
            q_n = p_{n+1} - 2p_{n}\quad(\eqa q_{n+1} = p_{n+2} - 2p_{n+1})
        }
        これを第二式に代入して整理すると
        \flan{
            p_{n+2} - p_{n+1} - 6p_{n} = 0
        }
        この漸化式は
        \flan{
            &\begin{dcases*}
              p_{n+2} - 3p_{n+1} = -2(p_{n+1} - 3p_{n}) \\
              p_{n+2} + 2p_{n+1} =  3(p_{n+1} + 2p_{n})
            \end{dcases*} \\
            &\begin{dcases*}
              p_{n+1} - 3p_{n} = 3(-2)^{n-1} \\
              p_{n+1} + 2p_{n} = 8\cdot3^{n-1}
            \end{dcases*} \\
        }
        ゆえに、
        \flan{
            p_n = \frac{8\cdot 3^{n-1} - 3\cdot 2^{n-1}}{5}
        }
        漸化式の第一式より
        \flan{
            q_n = p_{n+1} - 2p_{n} = \frac{8\cdot 3^{n-1} + 12(-2)^{n-1}}{5}
        }
      \end{other*}
  \end{enumerate}

\clearpage

  \item
  $n$を自然数、$x_1 = \sqrt{a}$として次の漸化式で与えられる数列$\{x_n\}$を考える。
  \begin{align*}
    x_{n+1} = \sqrt{x_n + a}\quad\cdots \rm{(i)}
  \end{align*}
  すなわち、
  \begin{align*}
    x_2 = \sqrt{a + \sqrt{a}}, \qquad x_3 = \sqrt{a + \sqrt{a + \sqrt{a}}}, \qquad\dots
  \end{align*}
  である。
  この数列が収束するかどうかを調べたい。
  次の問いに答えよ。
  \vskip.5\baselineskip
  \begin{enumerate}[label=(\arabic*)]
    \item 数列$\{x_n\}\:(n\in \mathbb{N})$が収束すると仮定して、その極限値を求めよ。
    \vskip.5\baselineskip
    $x_n \to \gra(>0)\:(n\to\infty)$とおくと(i)において両辺に$n\to\infty$の極限を考えて
    \flan{
      \gra = \sqrt{\gra + a}\quad\cdots \rm{(ii)}
    }
    この両辺を2乗して整理すれば
    \flan{
      &\gra^2 - \gra - a = 0\\
      &\Leftrightarrow \gra = \frac{1 + \sqrt{1 + 4a}}{2}\quad(\because \gra>0)
    }
    \vskip.5\baselineskip
    \item 数列$\{x_n\}\:(n\in \mathbb{N})$が(1)で得た値に実際に収束することを示せ。
    \vskip.5\baselineskip
    まず、(ii)式より

    \flan{
      &\gra^2 = a + \gra\\
      &\Rightarrow a - \gra^2 = -\gra \quad\cdots\rm{(iii)}
    }
    であるので、
    \flan{
      |x_n - \gra|
      &= \left|\frac{x_n^2 - \gra^2}{x_n + \gra}\right|\\
      &= \left|\frac{x_{n-1} + a - \gra^2}{x_n + \gra}\right|\\
      &= \left|\frac{x_{n-1} - \gra}{x_n + \gra}\right|\quad(\because \rm{(iii)})\\
      &= \frac{1}{x_n + \gra} |x_{n-1} - \gra|\\
      &\leqq \frac{1}{\gra} |x_{n-1} - \gra|\quad(\because x_n > 0)\\
      &= \cdots\\
      &= \frac{1}{\gra^{n-1}}|x_1 - \gra|
    }
    すなわち、
    \flan{
      (0 \leqq) |x_n - \gra| \leqq \frac{1}{\gra^{n-1}}|x_1 - \gra|\quad \cdots \rm{(iv)}
    }
    ここで、$a>0$より$\gra$は、
    \flan{
      \gra = \frac{1 + \sqrt{4a + 1}}{2} > \frac{1 + 1}{2} = 1
    }
    であるので、(iv)式について辺々に対して$n\to\infty$の極限を考えれば最右辺が
    \flan{
      \lim_{n\to\infty}\frac{1}{\gra^{n-1}}|x_1 - \gra| = 0
    }
    となることから、はさみうちの原理から
    \flan{
      \lim_{n\to\infty}|x_n - \gra| = 0
    }
    である。\\
    ゆえに、たしかに数列$\{x_n\}$は(1)で求めた値$\gra$に収束する。
    \end{enumerate}

    \vskip1\baselineskip
    \begin{supple*}
      比較的難しい問題ではあるがこのパターンは入試によく出るため覚えておく必要があります。 \\
      一般項が求まりそうにない数列の収束を示すのにははさみうちの原理を用いることを
      前提とした式変形をしていくのが最もよくある解法です。
      このとき、そもそも極限の収束とは
      \flan{
        \lim_{x\to a}f(x) = L \Leftrightarrow \lim_{x\to a}|f(x) - L| = 0
      }
      であることを思い出せば、$0\leqq |x_n - \gra| \leqq L(n)$であり、
      \dm{\lim_{n\to\infty}L(n)=0}であるような不等式を見つけるという方針は自然でしょう。

      すなわち、
      \flan{
          a_{n+1} = f(a_n)
      }
      のような漸化式の問題での解法は
      \begin{enumerate}[label=\arabic*.]
        \item 極限の予想
        \item 収束の証明
      \end{enumerate}
      の順です。\\
      1の極限の予想は比較的簡単で、極限が存在するとき、数列の添字にかかわらずある値に収束することがいえる
      (つまり\dm{\lim_{n\to\infty} a_n = \lim_{n\to\infty} a_{n+1}})
      のでその値を文字で置いたうえで
      漸化式からただの方程式として求めればいいです。
      ただし、複数の解をもつ場合はそれが極限の候補に過ぎないため
      実際にどれに収束するかはグラフや条件から絞る必要があります。 \\
      2は慣れないうちは難しく感じられるかもしれません。
      しかし、先ほども述べたように目標が
      \flan{
        |a_{n+1} - \gra| = r|a_n - \gra| \quad\text{かつ}\quad |r|<1
      }
      となるような関係(漸化不等式という)を導くことであるとわかっていれば、$a_{n+1}$と$a_n$、$\gra$をつなぐ関係式は
      漸化式とその極限をとった式($\gra = f(\gra)$)しかないのだから、それらの和や差をとって
      \flan{
        &a_{n+1} - \gra = f(a_n) - f(\gra)\\
        &\Rightarrow a_{n+1} - \gra = \bsq \,(a_n - \gra)
      }
      として、$|\bsq| \leqq r < 1$となるような定数$r$を見つけてやればよいとわかる。
      もちろん、$r = |\bsq|$としてもよい。
      \vskip1\baselineskip
      さて、以上を踏まえれば本問題の漸化不等式は次のように求めることもできる。\\
      \begin{other*}
        (i),(ii)の両辺について差を取れば
        \flan{
          x_{n+1} - \gra
          &= \sqrt{a + x_n} - \sqrt{a + \gra}\\
          &= \frac{1}{\sqrt{a + x_n} + \sqrt{a + \gra}}(x_n - \gra)
        }
        ここで、$\sqrt{a + x_n}\geqq 0$だから
        \flan{
          &\sqrt{a + x_n} + \sqrt{a + \gra} \geqq \sqrt{a + \gra} = \gra\quad(\because \rm{(iii)}) \\
          &\therefore |x_{n+1} - \gra|\leqq \frac{1}{\gra}|x_n - \gra|
        }
      \end{other*}
    \end{supple*}
  \end{enumerate}
\end{document}
