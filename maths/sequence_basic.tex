\documentclass[a4paper]{ltjsarticle}

% -------------------------------------------------------%
% base of preamble
% -------------------------------------------------------%
\usepackage{graphicx} % 図の挿入(includegraphics)
\usepackage{booktabs} % tableのmidrule
\usepackage{multirow} % tableのmultirow
\usepackage{float} % [H]で厳密に位置を固定

\usepackage{titlesec} % タイトルの書式を変える(titleformat)

\usepackage{amsmath} % 数式用
\usepackage{amssymb} % もっと数式記号(iintとか)
\usepackage{mathtools} % もっと数式(アクセント)
\usepackage{accents} % \undertildeで下付きチルダ
\usepackage{nccmath} % 数式左寄せ環境fleqn
\usepackage{empheq} % わからん
\usepackage{mathcomp} % tcdegree

\usepackage{amsthm} % 定理,証明など

\usepackage{siunitx} % 単位書き方(si)

\usepackage[version=4]{mhchem} % 化学式
\usepackage{chemfig}

\usepackage{enumitem}
% \setlist[enumerate,itemize]{
% 	itemsep=0pt,
% 	itemindent=4em,
% 	leftmargin=.5em,
% 	listparindent=1em,
% 	labelsep=.7em,
% 	itemindent=1.2em
% }
\usepackage{comment} % コメントアウト(begin{comment})

\usepackage{framed} % 左線
\usepackage{pict2e} % ベンゼン
\usepackage{ascmac} % itembox
\usepackage{fancybox} % itembox

\usepackage{fancyhdr} % ページ番号
\usepackage{lastpage}
\usepackage{color} % 色
\usepackage{url}
\usepackage{setspace}
% \usepackage{bm} % 太字
% \usepackage{wrapfig} % 画像回り込み
\usepackage{tikz}

% newcommand

% sequence type
\newcommand{\stype}[1]{\text{\hspace*{4pt}$\langle\text{\hspace*{.8pt}\raisebox{-.7pt}{#1}\hspace*{.8pt}}\rangle$\hspace*{4pt}}}
\newcommand{\tousa}{\stype{等差型}}
\newcommand{\touhi}{\stype{等比型}}
\newcommand{\kaisa}{\stype{階差型}}
\newcommand{\kaihi}{\stype{階比型}}
\newcommand{\kaihitousa}{\stype{階比・等差型}}
\newcommand{\tokusyukai}{\stype{特殊解型}}
\newcommand{\sankoukan}{\stype{三項間漸化式}}
\newcommand{\jisuusoui}{\stype{次数相違型}}
\newcommand{\sisu}{\stype{指数型}}
\newcommand{\sankoukanjuukai}{\stype{三項間漸化式(重解)}}
\newcommand{\sankoukanteisuukou}{\stype{三項間漸化式(定数項あり)}}
\newcommand{\bunsugyakusutikan}{\stype{分数型(逆数置換)}}
\newcommand{\bunsutokusei}{\stype{分数型(特性方程式)}}
\newcommand{\bunsujuukai}{\stype{分数型(重解)}}


% -------------------------------------------------------%

\makeatletter

% 数式や図表のref
\renewcommand{\refeq}[1]{\eqref{#1}式}
\newcommand{\reffig}[1]{図\ref{#1}}
\newcommand{\reftbl}[1]{表\ref{#1}}

% よく使う単位
\newcommand{\degC}[0]{\mathrm{{}^\circ \hspace*{-0.5pt} C}}
\renewcommand{\deg}[0]{\mathrm{{}^\circ}}

% よく使う演算子
\newcommand{\tensor}[1]{\undertilde{#1}}
\renewcommand{\rm}[1]{\mathrm{#1}}


\numberwithin{equation}{section}

% その他の環境
\newcommand{\dm}[1]{$\displaystyle #1 $}
\newcommand{\q}[1]{${\displaystyle #1}$}

\newcommand{\disp}{\displaystyle}

% -------------------------------------------------------%

% よく使う記号

% =================================
% 太字行列・ベクトル
\newcommand{\bA}{\bm{A}}
\newcommand{\bB}{\bm{B}}
\newcommand{\bE}{\bm{E}}
\newcommand{\bC}{\bm{C}}
\newcommand{\bD}{\bm{D}}
\newcommand{\bH}{\bm{H}}
\newcommand{\bI}{\bm{I}}
\newcommand{\bL}{\bm{L}}
\newcommand{\bU}{\bm{U}}
\newcommand{\bP}{\bm{P}}
\newcommand{\bQ}{\bm{Q}}

\newcommand{\bbR}{\mathbb{R}}
\newcommand{\bbC}{\mathbb{C}}
\newcommand{\bbN}{\mathbb{N}}
\newcommand{\bbZ}{\mathbb{Z}}

\newcommand{\ba}{{\bm{a}}}
\newcommand{\bb}{{\bm{b}}}
\newcommand{\bc}{{\bm{c}}}
\newcommand{\bd}{{\bm{d}}}
\newcommand{\be}{{\bm{e}}}
\newcommand{\bg}{{\bm{g}}}

\newcommand{\bbm}{{\bm{m}}}
\newcommand{\bn}{{\bm{n}}}

\newcommand{\bp}{{\bm{p}}}

\newcommand{\bt}{{\bm{t}}}

\newcommand{\bx}{{\bm{x}}}
\newcommand{\by}{{\bm{y}}}
\newcommand{\bz}{{\bm{z}}}

\newcommand{\bu}{{\bm{u}}}
\newcommand{\bv}{{\bm{v}}}
\newcommand{\bw}{{\bm{w}}}

% bold zero vector
\newcommand{\bzv}{\bm{0}}

% =================================
% ギリシャ文字
\newcommand{\ve}{\varepsilon}
\newcommand{\vp}{\varphi}

\newcommand{\gra}{{\alpha}}
\newcommand{\grg}{{\gamma}}
\newcommand{\grd}{{\delta}}
\newcommand{\grt}{{\theta}}
\newcommand{\grk}{{\kappa}}
\newcommand{\grl}{{\lambda}}
\newcommand{\grs}{{\sigma}}
\newcommand{\gro}{{\omega}}
\newcommand{\grp}{{\phi}}


\newcommand{\grG}{{\Gamma}}
\newcommand{\grL}{{\Lambda}}

% =================================
% 略記号
\newcommand{\tm}{\times}
\newcommand{\lra}{\longrightarrow}
\newcommand{\eqa}{\Leftrightarrow} % equivalent arrowのつもり

% -------------------------------------------------------%


% -------------------------------------------------------%


% 虚部の「ℑ」を更新(像として使いたい)
\renewcommand{\Im}{\operatorname{Im}}

% -------------------------------------------------------%

% 二項演算子
\renewcommand{\parallel}{\mathbin{/\!/}}

% 床関数(ガウス関数)
\newcommand{\flr}[1]{\lfloor #1 \rfloor} % ふつーの床関数
\newcommand{\gflr}[1]{\left[ #1 \right]} % ガウス記号を使った床関数

% -------------------------------------------------------%

% 演算子
\newcommand{\ppar}[2]{\frac{\partial #1}{\partial #2}}
\renewcommand{\d}{\partial}
\newcommand{\pd}{\partial}

% -------------------------------------------------------%
% ベクトル
\renewcommand{\vec}[1]{\begin{Bmatrix}#1\end{Bmatrix}}
% 行列
\newcommand{\bmat}[1]{\begin{bmatrix}#1\end{bmatrix}} % [A]
\newcommand{\Bmat}[1]{\begin{Bmatrix}#1\end{Bmatrix}} % {A}
\newcommand{\pmat}[1]{\begin{pmatrix}#1\end{pmatrix}} % (A)
\newcommand{\vmat}[1]{\begin{vmatrix}#1\end{vmatrix}} % |A|

% ロピタル
\newcommand{\lhopital}{L'H\^{o}pital}
% 集合
\newcommand{\Dset}[1]{\left\{ #1 \right\}}
% 集合の{(x,y)|x<0}みたいに書くときの縦線
\newcommand{\relmiddle}{\mathrel{}\middle| \mathrel{}}

% 内積
\newcommand{\ip}[1]{\langle#1\rangle}

% 記号
\newcommand{\bsq}{\raisebox{-1.2pt}{\blacksquare}}

% -------------------------------------------------------%
% set styles and environments
% -------------------------------------------------------%

\pagestyle{plain}

% -------------------------------------------------------%

\titleformat*{\section}{\Large\bfseries}
\titleformat*{\subsection}{\large\bfseries}

% -------------------------------------------------------%

% 数式番号にセクション番号を併記する
\renewcommand{\theequation}{\thesection.\arabic{equation}}
\makeatletter
\@addtoreset{equation}{section}
\makeatother

% -------------------------------------------------------%

% 定理スタイルの定義
\newtheoremstyle{mystyle}
  {\topsep}   % スペース上
  {\topsep}   % スペース下
  {\normalfont}  % 本文のフォント
  {0pt}       % インデント
  {\bfseries} % タイトルのフォント
  {.}         % タイトルあとの句読点
  {.5em}      % タイトルと本文のスペース
  {\thmname{#1}\thmnumber{#2}\thmnote{#3}} % タイトルのスタイル
% 定理環境の作成
\theoremstyle{mystyle}
\newtheorem*{question*}{問題}
\newtheorem{problem}{} % 問題番号のみ
\newtheorem*{ans*}{解答}
\newtheorem*{practice*}{例題}
\newtheorem*{other*}{別解}
\newtheorem*{supple*}{補足}
\newtheorem*{append*}{補遺}
\newtheorem*{prf*}{証明}

% -------------------------------------------------------%
% % 不等式のため
\newtheoremstyle{inequationmystyle}
  {\topsep}   % スペース上
  {\topsep}   % スペース下
  {\normalfont}  % 本文のフォント
  {0pt}       % インデント
  {} % タイトルのフォント
  {}         % タイトルあとの句読点
  {3pt}      % タイトルと本文のスペース
  {【\,\textbf{\thmname{#1}}\thmnumber{#2}\,】\thmnote{#3}} % タイトルのスタイル
% 定理環境の作成
\theoremstyle{inequationmystyle}
\newtheorem*{syoumei*}{証明}
\newtheorem*{kai*}{解}
\newtheorem*{rei*}{例}


% -------------------------------------------------------%

\newtheoremstyle{problemstyle}
  {}   % スペース上
  {}   % スペース下
  {\normalfont}  % 本文のフォント
  {0pt}       % インデント
  {} % タイトルのフォント
  {}         % タイトルあとの句読点
  {.5em}      % タイトルと本文のスペース
  {\thmname{#1}\thmnumber{(#2)}\thmnote{#3}} % タイトルのスタイル
\theoremstyle{problemstyle}
\newtheorem{myprob}{}

% -------------------------------------------------------%

\newtheoremstyle{intproblemstyle}{20pt}{10pt}{\normalfont}{0pt}{}{}{.5em}
{\thmname{#1}\thmnumber{#2.\hspace*{5pt}}\thmnote{#3}} % タイトルのスタイル
\theoremstyle{intproblemstyle}
\newtheorem{intprob}{} % 積分

% -------------------------------------------------------%

\newtheoremstyle{cfstyle}{}{}{\normalfont}{0pt}{\itshape}{}{.5em}
{\thmname{#1}\thmnumber{#2}\thmnote{#3}} % タイトルのスタイル
\theoremstyle{cfstyle}
\newtheorem*{confer*}{cf.} % 積分

% -------------------------------------------------------%

% (1)のような環境。セクションごとに
\newtheoremstyle{lineupstyle}
  {}   % スペース上
  {}   % スペース下
  {\normalfont}  % 本文のフォント
  {0pt}       % インデント
  {} % タイトルのフォント
  {}         % タイトルあとの句読点
  {6pt}      % タイトルと本文のスペース
  {\thmname{#1}\thmnumber{(#2)}\thmnote{\textbf{#3}}} % タイトルのスタイル
\theoremstyle{lineupstyle}
\newtheorem{lineup}{}[section]
\makeatletter
\@addtoreset{lineup}{section} % lineupカウンターがsectionが更新されるたびにリセットされる
\makeatother
\renewcommand{\thelineup}{\arabic{lineup}}

% -------------------------------------------------------%

% 問題文の左の線の定義
\renewenvironment{leftbar}{%
\def\FrameCommand{\hspace{10pt}\vrule width 1.2pt \hspace{10pt}}%
\MakeFramed {\advance\hsize-\width \FrameRestore}}%
{\endMakeFramed}

% subsubsectionを太字の「問1」表示にする
\renewcommand{\thesubsubsection}{\large\textbf{問\arabic{subsubsection}}}

\newcommand{\prob}[1]{%
  \begin{question*}%
    ${}$%
    \vspace{-.5\baselineskip}%
    \begin{leftbar}%
      #1%
    \end{leftbar}%
  \end{question*}%
}


\title{
  {\Large 数列と漸化式} \\
  --- 基本編 ---
}
\author{
  Yuta Suzuki
  \thanks{https://github.com/suzuyut4}
}
\date{}


%**************************************************************
\begin{document}

\maketitle

\section{数列と漸化式の基本}

\subsection{漸化式}
そもそも、公式を用いて簡単に解ける漸化式は次の3つに限られます。

\begin{enumerate}[label=(\roman*)]
  \item
  $a_{n+1} = a_n + d$  \rightarrow 公差$d$の等差数列
  \begin{fleqn}[20pt]
    \begin{align}
      \Rightarrow a_n = a_1 + (n-1)d \label{eq:tousa}
    \end{align}
  \end{fleqn}
  \item
  $a_{n+1} = ra_n$     \rightarrow 公比$r$の等比数列
  \begin{fleqn}[20pt]
    \begin{align}
      \Rightarrow a_n = a_1 r^{n-1} \label{eq:touhi}
    \end{align}
  \end{fleqn}
  \item
  $a_{n+1} = a_n + b_n$\rightarrow 階差数列が$\{b_n\}$の数列$\{a_n\}$
  \begin{fleqn}[20pt]
    \begin{align}
      \Rightarrow (n\geqq 2 \text{のとき、})\:\:a_n = a_1 + \sum_{k=1}^{n-1}b_k \label{eq:kaisa}
    \end{align}
  \end{fleqn}
\end{enumerate}
以下、\eqref{eq:tousa}の形を等差型、\eqref{eq:touhi}の形を等比型、\eqref{eq:kaisa}の形を階差型
と呼ぶことにします。
この形以外のほぼすべての漸化式はなんとかしてこの形に帰着させることが目的で、
多くの場合は等比数列の形に変形してから一般項を求めるということも覚えておくとよいです。

せっかくなので\eqref{eq:kaisa}の形だけはここで証明しておきましょう。
\begin{prf*}
$a_{n+1} = a_n + b_n \Leftrightarrow a_{n+1} - a_n = b_n$であるので、
$n$を$n-1,n-2,\cdots,2,1$として足し合わせると

{\newcommand{\dl}[1]{\raisebox{-2pt}{$#1$}}
\begin{fleqn}[20pt]
  \begin{align*}
    &\begin{matrix}
      & a_n     & - & a_{n-1} & = & b_{n-1} \\
      & a_{n-1} & - & a_{n-2} & = & b_{n-2} \\
      & a_{n-2} & - & a_{n-3} & = & b_{n-3} \\
      &       & \vdots &    &\vdots &       \\
      & a_3     & - & a_2     & = & b_2     \\
  +) & a_2     & - & a_1     & = & b_1     \\
  \hline
      & \dl{a_n}     & \dl{-} & \dl{a_1}     & \dl{=} & \dl{b_{n-1} + b_{n-2} + \cdots + b_2 + b_1}
    \end{matrix}
    \\
    &\Leftrightarrow a_n = a_1 + \sum_{k=1}^{n-1}b_k \qquad \qed
  \end{align*}
\end{fleqn}
}
\end{prf*}

ここで、証明一行目において$n-1$以下のケースを考えることによってこの式を得ています。
しかし、この変形ができるのは$n-1\geqq 1$すなわち$n\geqq 2$のときのみです。
そのため、階差型においては条件$n\geqq 2$を忘れてはいけません。
すなわち、一般項を求めたとき、$n=1$でもその式が成り立っているか必ず確認し、
成り立っていればそのように書き、成り立っていなければ場合分けして一般項を示す必要があります。
解答の書き方など詳しくは、\ref*{sec:basic_of_recurrenceformula}漸化式 基本編で確認してください。


\subsection{数列の和$S_n$}
一般項$a_n$で表される数列について第一項から第$n$項までの和を$S_n$で表すことがあります。
すなわち
\begin{fleqn}[20pt]
  \begin{align*}
    S_n = \sum_{k=1}^{n}a_k
  \end{align*}
\end{fleqn}
です。以下、特に断りがなければ$S_n$を数列$\{a_n\}$に対する和を表すものとします。

\clearpage
\section{数列とその周辺}
\begin{question*}
  次の値を$n$を用いて表せ。
\begin{enumerate}[label=\arabic*., itemsep=.5em]
  \item
    \dm{\sum_{k=1}^{n}k}
  \item
    \dm{\sum_{k=1}^{n}k^2}
  \item
    \dm{\sum_{k=1}^{n}k^3}
  \item
    \dm{\sum_{k=1}^{n}\frac{3}{k(k+2)}}
  \item
    \dm{\sum_{k=1}^{n}\frac{3}{\sqrt{k+2} + \sqrt{k}}}
  \item
    \dm{\sum_{k=1}^{n}\{(2k-1)\cdot 2^{k-1}\}}
  \item
    \dm{\frac{1}{1} + \frac{1}{1+2} + \frac{1}{1+2+3} + \cdots + \frac{1}{1+2+\cdots +k} + \cdots + \frac{1}{1+2+3+\cdots +(n-1)+n}}
  \item
    \dm{1+\frac{2}{3}+\frac{3}{3^2} + \cdots + \frac{k}{3^{k-1}} + \cdots + \frac{n}{3^{n-1}}}
\end{enumerate}
\end{question*}

\clearpage
\section{漸化式 基本編}\label{sec:basic_of_recurrenceformula}

\begin{question*}
  次の式を満たす数列$\{a_n\}$を$n$を用いて表せ。
\begin{enumerate}[label=\arabic*.]

  \item $\disp a_1 = 2,\; a_{n+1} = a_n + 3$\\

  \item $\disp a_1 = 5,\; a_{n+1} = 7a_n$\\

  \item $\disp a_1 = 3,\; a_{n+1} = 3a_n -4$\\

  \item $\disp a_1 = 1,\; a_{n+1} = a_n + 2n$\\

  \item $\disp a_1 = 1,\; a_{n+1} = a_n + 3^n - 4n$\\

  \item $\disp a_1 = 1,\; a_2 = 2,\; a_{n+2} = a_{n+1} + 6a_n$\\

  \item $\disp a_1 = 2,\; a_{n+1} = 16a_n^5$\\

  \item $\disp a_1 = 1,\; a_{n+1} = 2a_n + n^2 -6$\\

  \item $\disp a_1 = 1,\; a_{n+1} = 4a_n + n\cdot 2^n$\\

  \item $\disp a_1 = 1,\; a_2 = 3,\; a_{n+2} = 4a_{n+1} - 4a_n$\\

  \item $\disp a_1 = 1,\; a_2 = 4,\; a_{n+2} = 4a_{n+1} - 3a_n - 2$\\

  \item $\disp a_1 = 1,\: a_{n+1} = \frac{a_n}{2a_n + 3}$\\

  \item $\disp a_1 = 3,\; a_{n+1} = \frac{3a_n-4}{a_n-2}$\\

  \item $\disp a_1 = 3,\; a_{n+1} = \frac{3a_n-4}{a_n-1}$\\

  \item $\disp a_1 = 1,\; a_{n+1}=(n+1)a_n$\\

  \item $\disp a_1 = 1,\; (n+2)a_{n+1}=na_n$\\

  \item $\disp a_1 = 1,\; na_{n+1}=2(n+1)a_n+n(n+1)$\\

  \item $\disp a_1 = 2,\; a_{n+1} = \frac{n+2}{n} a_n + 1$\\

  \item $\disp a_1 = 1,\; a_{n+1} = 2^{2n-2}(a_n)^2$\\

  \item $\disp S_n = 3a_n + 2n - 1$

\end{enumerate}
\end{question*}

\end{document}
