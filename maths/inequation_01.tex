\documentclass[twocolumn]{ltjsarticle}

\usepackage[top=25truemm,bottom=30truemm,left=10truemm,right=10truemm]{geometry}
\usepackage{setspace}
%-------------------------------------------------------%
\newcommand{\filesectionstyle}{lecture}
\newcommand{\filesubsectionstyle}{lecture}

% -------------------------------------------------------%
% base of preamble
% -------------------------------------------------------%
\usepackage{graphicx} % 図の挿入(includegraphics)
\usepackage{booktabs} % tableのmidrule
\usepackage{multirow} % tableのmultirow
\usepackage{float} % [H]で厳密に位置を固定

\usepackage{titlesec} % タイトルの書式を変える(titleformat)

\usepackage{amsmath} % 数式用
\usepackage{amssymb} % もっと数式記号(iintとか)
\usepackage{mathtools} % もっと数式(アクセント)
\usepackage{accents} % \undertildeで下付きチルダ
\usepackage{nccmath} % 数式左寄せ環境fleqn
\usepackage{empheq} % わからん
\usepackage{mathcomp} % tcdegree

\usepackage{amsthm} % 定理,証明など

\usepackage{siunitx} % 単位書き方(si)

\usepackage[version=4]{mhchem} % 化学式
\usepackage{chemfig}

\usepackage{enumitem}
% \setlist[enumerate,itemize]{
% 	itemsep=0pt,
% 	itemindent=4em,
% 	leftmargin=.5em,
% 	listparindent=1em,
% 	labelsep=.7em,
% 	itemindent=1.2em
% }
\usepackage{comment} % コメントアウト(begin{comment})

\usepackage{framed} % 左線
\usepackage{pict2e} % ベンゼン
\usepackage{ascmac} % itembox
\usepackage{fancybox} % itembox

\usepackage{fancyhdr} % ページ番号
\usepackage{lastpage}
\usepackage{color} % 色
\usepackage{url}
\usepackage{setspace}
% \usepackage{bm} % 太字
% \usepackage{wrapfig} % 画像回り込み
\usepackage{tikz}

% -------------------------------------------------------%
% set styles and environments
% -------------------------------------------------------%

\pagestyle{plain}

% -------------------------------------------------------%

\titleformat*{\section}{\Large\bfseries}
\titleformat*{\subsection}{\large\bfseries}

% -------------------------------------------------------%

% 数式番号にセクション番号を併記する
\renewcommand{\theequation}{\thesection.\arabic{equation}}
\makeatletter
\@addtoreset{equation}{section}
\makeatother

% -------------------------------------------------------%

% 定理スタイルの定義
\newtheoremstyle{mystyle}
  {\topsep}   % スペース上
  {\topsep}   % スペース下
  {\normalfont}  % 本文のフォント
  {0pt}       % インデント
  {\bfseries} % タイトルのフォント
  {.}         % タイトルあとの句読点
  {.5em}      % タイトルと本文のスペース
  {\thmname{#1}\thmnumber{#2}\thmnote{#3}} % タイトルのスタイル
% 定理環境の作成
\theoremstyle{mystyle}
\newtheorem*{question*}{問題}
\newtheorem{problem}{} % 問題番号のみ
\newtheorem*{ans*}{解答}
\newtheorem*{practice*}{例題}
\newtheorem*{other*}{別解}
\newtheorem*{supple*}{補足}
\newtheorem*{append*}{補遺}
\newtheorem*{prf*}{証明}

% -------------------------------------------------------%
% % 不等式のため
\newtheoremstyle{inequationmystyle}
  {\topsep}   % スペース上
  {\topsep}   % スペース下
  {\normalfont}  % 本文のフォント
  {0pt}       % インデント
  {} % タイトルのフォント
  {}         % タイトルあとの句読点
  {3pt}      % タイトルと本文のスペース
  {【\,\textbf{\thmname{#1}}\thmnumber{#2}\,】\thmnote{#3}} % タイトルのスタイル
% 定理環境の作成
\theoremstyle{inequationmystyle}
\newtheorem*{syoumei*}{証明}
\newtheorem*{kai*}{解}
\newtheorem*{rei*}{例}


% -------------------------------------------------------%

\newtheoremstyle{problemstyle}
  {}   % スペース上
  {}   % スペース下
  {\normalfont}  % 本文のフォント
  {0pt}       % インデント
  {} % タイトルのフォント
  {}         % タイトルあとの句読点
  {.5em}      % タイトルと本文のスペース
  {\thmname{#1}\thmnumber{(#2)}\thmnote{#3}} % タイトルのスタイル
\theoremstyle{problemstyle}
\newtheorem{myprob}{}

% -------------------------------------------------------%

\newtheoremstyle{intproblemstyle}{20pt}{10pt}{\normalfont}{0pt}{}{}{.5em}
{\thmname{#1}\thmnumber{#2.\hspace*{5pt}}\thmnote{#3}} % タイトルのスタイル
\theoremstyle{intproblemstyle}
\newtheorem{intprob}{} % 積分

% -------------------------------------------------------%

\newtheoremstyle{cfstyle}{}{}{\normalfont}{0pt}{\itshape}{}{.5em}
{\thmname{#1}\thmnumber{#2}\thmnote{#3}} % タイトルのスタイル
\theoremstyle{cfstyle}
\newtheorem*{confer*}{cf.} % 積分

% -------------------------------------------------------%

% (1)のような環境。セクションごとに
\newtheoremstyle{lineupstyle}
  {}   % スペース上
  {}   % スペース下
  {\normalfont}  % 本文のフォント
  {0pt}       % インデント
  {} % タイトルのフォント
  {}         % タイトルあとの句読点
  {6pt}      % タイトルと本文のスペース
  {\thmname{#1}\thmnumber{(#2)}\thmnote{\textbf{#3}}} % タイトルのスタイル
\theoremstyle{lineupstyle}
\newtheorem{lineup}{}[section]
\makeatletter
\@addtoreset{lineup}{section} % lineupカウンターがsectionが更新されるたびにリセットされる
\makeatother
\renewcommand{\thelineup}{\arabic{lineup}}

% -------------------------------------------------------%

% 問題文の左の線の定義
\renewenvironment{leftbar}{%
\def\FrameCommand{\hspace{10pt}\vrule width 1.2pt \hspace{10pt}}%
\MakeFramed {\advance\hsize-\width \FrameRestore}}%
{\endMakeFramed}

% subsubsectionを太字の「問1」表示にする
\renewcommand{\thesubsubsection}{\large\textbf{問\arabic{subsubsection}}}

\newcommand{\prob}[1]{%
  \begin{question*}%
    ${}$%
    \vspace{-.5\baselineskip}%
    \begin{leftbar}%
      #1%
    \end{leftbar}%
  \end{question*}%
}

% newcommand

% sequence type
\newcommand{\stype}[1]{\text{\hspace*{4pt}$\langle\text{\hspace*{.8pt}\raisebox{-.7pt}{#1}\hspace*{.8pt}}\rangle$\hspace*{4pt}}}
\newcommand{\tousa}{\stype{等差型}}
\newcommand{\touhi}{\stype{等比型}}
\newcommand{\kaisa}{\stype{階差型}}
\newcommand{\kaihi}{\stype{階比型}}
\newcommand{\kaihitousa}{\stype{階比・等差型}}
\newcommand{\tokusyukai}{\stype{特殊解型}}
\newcommand{\sankoukan}{\stype{三項間漸化式}}
\newcommand{\jisuusoui}{\stype{次数相違型}}
\newcommand{\sisu}{\stype{指数型}}
\newcommand{\sankoukanjuukai}{\stype{三項間漸化式(重解)}}
\newcommand{\sankoukanteisuukou}{\stype{三項間漸化式(定数項あり)}}
\newcommand{\bunsugyakusutikan}{\stype{分数型(逆数置換)}}
\newcommand{\bunsutokusei}{\stype{分数型(特性方程式)}}
\newcommand{\bunsujuukai}{\stype{分数型(重解)}}


% -------------------------------------------------------%

\makeatletter

% 数式や図表のref
\renewcommand{\refeq}[1]{\eqref{#1}式}
\newcommand{\reffig}[1]{図\ref{#1}}
\newcommand{\reftbl}[1]{表\ref{#1}}

% よく使う単位
\newcommand{\degC}[0]{\mathrm{{}^\circ \hspace*{-0.5pt} C}}
\renewcommand{\deg}[0]{\mathrm{{}^\circ}}

% よく使う演算子
\newcommand{\tensor}[1]{\undertilde{#1}}
\renewcommand{\rm}[1]{\mathrm{#1}}


\numberwithin{equation}{section}

% その他の環境
\newcommand{\dm}[1]{$\displaystyle #1 $}
\newcommand{\q}[1]{${\displaystyle #1}$}

\newcommand{\disp}{\displaystyle}

% -------------------------------------------------------%

% よく使う記号

% =================================
% 太字行列・ベクトル
\newcommand{\bA}{\bm{A}}
\newcommand{\bB}{\bm{B}}
\newcommand{\bE}{\bm{E}}
\newcommand{\bC}{\bm{C}}
\newcommand{\bD}{\bm{D}}
\newcommand{\bH}{\bm{H}}
\newcommand{\bI}{\bm{I}}
\newcommand{\bL}{\bm{L}}
\newcommand{\bU}{\bm{U}}
\newcommand{\bP}{\bm{P}}
\newcommand{\bQ}{\bm{Q}}

\newcommand{\bbR}{\mathbb{R}}
\newcommand{\bbC}{\mathbb{C}}
\newcommand{\bbN}{\mathbb{N}}
\newcommand{\bbZ}{\mathbb{Z}}

\newcommand{\ba}{{\bm{a}}}
\newcommand{\bb}{{\bm{b}}}
\newcommand{\bc}{{\bm{c}}}
\newcommand{\bd}{{\bm{d}}}
\newcommand{\be}{{\bm{e}}}
\newcommand{\bg}{{\bm{g}}}

\newcommand{\bbm}{{\bm{m}}}
\newcommand{\bn}{{\bm{n}}}

\newcommand{\bp}{{\bm{p}}}

\newcommand{\bt}{{\bm{t}}}

\newcommand{\bx}{{\bm{x}}}
\newcommand{\by}{{\bm{y}}}
\newcommand{\bz}{{\bm{z}}}

\newcommand{\bu}{{\bm{u}}}
\newcommand{\bv}{{\bm{v}}}
\newcommand{\bw}{{\bm{w}}}

% bold zero vector
\newcommand{\bzv}{\bm{0}}

% =================================
% ギリシャ文字
\newcommand{\ve}{\varepsilon}
\newcommand{\vp}{\varphi}

\newcommand{\gra}{{\alpha}}
\newcommand{\grg}{{\gamma}}
\newcommand{\grd}{{\delta}}
\newcommand{\grt}{{\theta}}
\newcommand{\grk}{{\kappa}}
\newcommand{\grl}{{\lambda}}
\newcommand{\grs}{{\sigma}}
\newcommand{\gro}{{\omega}}
\newcommand{\grp}{{\phi}}


\newcommand{\grG}{{\Gamma}}
\newcommand{\grL}{{\Lambda}}

% =================================
% 略記号
\newcommand{\tm}{\times}
\newcommand{\lra}{\longrightarrow}
\newcommand{\eqa}{\Leftrightarrow} % equivalent arrowのつもり

% -------------------------------------------------------%


% -------------------------------------------------------%


% 虚部の「ℑ」を更新(像として使いたい)
\renewcommand{\Im}{\operatorname{Im}}

% -------------------------------------------------------%

% 二項演算子
\renewcommand{\parallel}{\mathbin{/\!/}}

% 床関数(ガウス関数)
\newcommand{\flr}[1]{\lfloor #1 \rfloor} % ふつーの床関数
\newcommand{\gflr}[1]{\left[ #1 \right]} % ガウス記号を使った床関数

% -------------------------------------------------------%

% 演算子
\newcommand{\ppar}[2]{\frac{\partial #1}{\partial #2}}
\renewcommand{\d}{\partial}
\newcommand{\pd}{\partial}

% -------------------------------------------------------%
% ベクトル
\renewcommand{\vec}[1]{\begin{Bmatrix}#1\end{Bmatrix}}
% 行列
\newcommand{\bmat}[1]{\begin{bmatrix}#1\end{bmatrix}} % [A]
\newcommand{\Bmat}[1]{\begin{Bmatrix}#1\end{Bmatrix}} % {A}
\newcommand{\pmat}[1]{\begin{pmatrix}#1\end{pmatrix}} % (A)
\newcommand{\vmat}[1]{\begin{vmatrix}#1\end{vmatrix}} % |A|

% ロピタル
\newcommand{\lhopital}{L'H\^{o}pital}
% 集合
\newcommand{\Dset}[1]{\left\{ #1 \right\}}
% 集合の{(x,y)|x<0}みたいに書くときの縦線
\newcommand{\relmiddle}{\mathrel{}\middle| \mathrel{}}

% 内積
\newcommand{\ip}[1]{\langle#1\rangle}

% 記号
\newcommand{\bsq}{\raisebox{-1.2pt}{\blacksquare}}


\renewcommand{\v}[1]{\overrightarrow{\mathstrut{#1}}}

\title{
  \leftline{\normalsize Citation: 東京出版 大学への数学 2020-07, 安田 亨}
  \leftline{\large 要点の整理/数I II}
  不等式 --- 基本の総括
}
\author{}
\date{}

%**************************************************************

\begin{document}

\maketitle

\section{不等式の3つの意味}\label{sec:3-meanings}
不等式にはいくつかの意味があり,
どの意味で使っているのかを間違えると誤答につながるので注意しよう.

\begin{lineup}[単なる大小関係をあらわすときに使う]
  ${}$

  \begin{align*}
    a\geq b \text{ は } a > b \text{ または } a = b
  \end{align*}
  の少なくとも一方が成り立つと主張しているだけである.例えば $3\geq1$ は $3>1$ が成り立つので正しい.
\end{lineup}

\begin{lineup}[解集合を表す]
  ${}$

  たとえば $x^2-3x+2=0$ を解くと $x$ は $1$ と $2$ となり,$\{1,2\}$  という解の集合が求められる.
  $x^2-3x+2<0$ を解くと$1<x<2$ となり,$1<x<2$ を満たす $x$ 全体のことである.
\end{lineup}

\begin{lineup}[取りうる値の範囲を表す]
  ${}$

  実数 $x$ が $1$ と $2$ の間のすべての値をとって動くことを $1<x<2$ と表す場合がある.
\end{lineup}



\section{不等式を解く}\label{sec:solve-ineq}
$a<b$ のとき
\begin{lineup}
  \begin{gather*}
    (x-a)(x-b)   >  0  \Longleftrightarrow x   <  a, \: x   >  b \\
    (x-a)(x-b) \geq 0  \Longleftrightarrow x \leq a, \: x \geq b
  \end{gather*}
  「$a,\, b$ の外側」と理解する.
\end{lineup}
\begin{lineup}
  \begin{gather*}
    (x-a)(x-b)   <  0  \Longleftrightarrow a   <  x  <   b \\
    (x-a)(x-b) \leq 0  \Longleftrightarrow a \leq x \leq b
  \end{gather*}
  「$a,\, b$ の内側」と理解する.
\end{lineup}

分数不等式を解く場合は,分母を払う人が多いが,
符号を考えないでいきなり払ってしまうミスが少なくない.
それを防ぐためには「移項して通分」「各区間の $x$ に対して符号を判別」を行う方がよい.


\section{不等式証明の基本}\label{sec:basis-ineq}
「不等式 $A\geq B$ を証明せよ」という問題の場合,$A-B$ を作り,
整理し,因数分解か平方完成するというのは基本的手法である.

\section{絶対値の外し方}\label{sec:how-to-getridof-abs}
「絶対値をはずすときには符号で場合分けせよ」と習うが,
それは基本であり,込み入った問題では場合分けを減らすことが,
その後の処理を簡単にしてくれる.

\begin{lineup}
  $|X| = |A|$のときは,$X\geq \pm A$と外す.
  $X$の正負には言及しない.

  $|X| = A$のときは$A\geq 0$かつ$X = \pm A$と外す.
\end{lineup}

\begin{lineup}
  $|X|\leq A$のときは$-A\leq X\leq A$と外す.

  このとき,$A\geq 0$でなければならないが,
  $-A\leq X\leq A$のときには$-A\leq A$だから$A\geq 0$となる.
  $A,\:X$の符号に言及する必要はない.
\end{lineup}

\begin{lineup}
  $|X|\geq A$の解について,$A<0$のとき$X$は任意,
  $A\geq 0$のとき$X\leq -A$,$X\geq A$

  実は,$A$の符号によらず$X\leq -A,\:X\geq A$を外すことができる.
\end{lineup}

\section{有名な不等式を利用する}\label{sec:use-famous-ineqs}
\subsection{三角不等式}

\begin{lineup}
  実数について$\Bigl||x|-|y|\Bigr|\leq |x+y|\leq |x|+|y|$\\
  左の等号は$xy\leq 0$,右の等号は$xy\geq 0$で成り立つ.

  数学IIIの複素数ではこの不等式を使うチャンスは多いが,
  実数では,使う問題がほとんど出ない.\\
\end{lineup}

\begin{lineup}
  空間の3点$\rm{P}$,$\rm{Q}$,$\rm{R}$について$\rm{PR}\leq \rm{PQ} + \rm{QR}$\\
  等号は$\rm{P}$,$\rm{Q}$,$\rm{R}$の順で一直線上にあるとき成り立つ.\\
\end{lineup}

\begin{lineup}
  $\Bigl||\v{x}|-|\v{y}|\Bigr| \leq |\v{x} + \v{y}| \leq |\v{x}| + |\v{y}|$
  \vskip0.1\baselineskip
  左の等号は$\v{x}$,$\v{y}$の一方が$\v{0}$か逆向きに平行のとき,
  右の等号は$\v{x}$,$\v{y}$の一方が$\v{0}$か同じ向きに平行のとき
  成り立つ.
\end{lineup}

\subsection{$x^2 + y^2 + z^2 \geq xy + yz + zx$}
\begin{syoumei*}
  \begin{fleqn}[20pt]
    \begin{align*}
      & x^2 + y^2 + z^2 - (xy + yz + zx)\\
      =& \frac{1}{2} {(x-y)^2 + (y-z)^2 + (z-x)^2} \geq 0
    \end{align*}
  \end{fleqn}
\end{syoumei*}


\subsection{相加・相乗平均の不等式(AM-GM Inequality)}
文字はすべて正,$n$は自然数とする.
\begin{fleqn}[20pt]
  \begin{align*}
    &\frac{x+y}{2} \geq \sqrt{xy} \tag*{(A)}\label{eq:a}\\
    &\frac{x+y+z}{3} \geq \sqrt[3]{xyz} \tag*{(B)}\label{eq:b}\\
    &\frac{x_{1}+ \cdots +x_{n}}{n} \geq \sqrt[n]{x_{1}\cdots\cdots x_{n}}
  \end{align*}
\end{fleqn}
\ref*{eq:a}の等号は$x=y$のとき成り立つ.
他も同様である.
\ref*{eq:a}の証明は
\begin{fleqn}[20pt]
  \begin{align*}
    \frac{x+y}{2} - \sqrt{xy} = \frac{1}{2}(\sqrt{x}-\sqrt{y})^2 \geq 0
  \end{align*}
\end{fleqn}
\ref*{eq:b}の証明は
\begin{fleqn}[20pt]
  \begin{align*}
    &A^3+B^3+C^3-3ABC\\
    =&(A+B+C)(A^2+B^2+C^2-AB-BC-CA)\\
    =&(A+B+C)\frac{(A-B)^2 + (B-C)^2 + (C-A)^2}{2}\geq 0
  \end{align*}
\end{fleqn}
\begin{spacing}{1.5}
\noindent として,$A^3+B^3+C^3-3ABC\geq 0$で
$A=\sqrt[3]{x}$,
$B=\sqrt[3]{y}$,
$C=\sqrt[3]{z}$
とおけばよい.
次のコーシーの証明方法も有名である.
\ref*{eq:a}を2回使って,
\begin{fleqn}[20pt]
  \begin{align*}
    \frac{x+y+z+w}{4}
    &\geq \frac{2\sqrt{xy}+2\sqrt{zw}}{4} = \frac{\sqrt{xy}+\sqrt{zw}}{2}\\
    &\geq \sqrt{\sqrt{xy}\cdot \sqrt{zw}}=\sqrt[4]{xyzw}
  \end{align*}
\end{fleqn}
よって\dm{\frac{x+y+z+w}{4}\geq \sqrt[4]{xyzw}}となり,ここで
\dm{w=\frac{x+y+z}{3}}とおくと,不等式は\\
\dm{\frac{x+y+z}{3}\geq \sqrt[4]{xyz\cdot \frac{x+y+z}{3}}}となる.
両辺を4乗し\dm{\frac{x+y+z}{3}}で割ると
\dm{\biggl(\frac{x+y+z}{3}\biggr)^{\hspace*{-1pt}\raisebox{-1pt}{\footnotesize 3}}\geq xyz} \\
両辺の3乗根をとると\ref*{eq:b}を得る.
\end{spacing}


\subsection{コーシー・シュワルツの不等式(Cauchy-Schwarz inequality)}
\begin{spacing}{1.3}
\begin{gather*}
  (a_{1}^2+a_{2}^2 + \cdots\cdots + a_{n}^2)(b_{1}^2+b_{2}^2 + \cdots\cdots + b_{n}^2) \qquad \\
  \geq (a_{1}b_{1} + a_{2}b_{2} + \cdots \cdots + a_{n}b_{n})^2
\end{gather*}
等号は\dm{\frac{b_{1}}{a_{1}}=\frac{b_{2}}{a_{2}}= \cdots\cdots =\frac{b_{n}}{a_{n}}}
(分母が0の項は分子も0とする)のとき成り立つ,と書くのが普通である.
厳密には「$a_{1}$から$a_{n}$までがすべて0か,
または$a_{1}$から$a_{n}$までの中に0でないものがあるときは
\dm{\frac{b_{1}}{a_{1}}=\frac{b_{2}}{a_{2}}= \cdots\cdots =\frac{b_{n}}{a_{n}}}
(分母が0の項は分子も0とする)」のときだが,
0の記述は実用上意味がない.
\end{spacing}

\begin{syoumei*}
  $A=a_{1}^2+\cdots\cdots + a_{n}$,$B=b_{1}+\cdots\cdots+b_{n}$,
  $C=a_{1}b_{1}+\cdots\cdots +a_{n}b_{n}$とおいて$AB\geq C^2$を示す.\\
  $A=0$のときは$a_{1} = \cdots\cdots =a_{n}=0$で,
  証明すべき不等式の等号が成り立つ.$A>0$のときは
  \begin{fleqn}[20pt]
    \begin{align*}
      \sum_{k=1}^{n}(a_{k}t-b_{k})^2
      &= \sum_{k=1}^{n}(a_{k}^2t^2 + 2a_{k}b_{k}t + b_{k}^2) \\
      &= At^2 -2Ct + B
      = A\left(t-\frac{C}{A}\right)^2 - \frac{C^2}{A} + B
    \end{align*}
  \end{fleqn}
  が任意の$t$で成り立つから,特に\dm{t=\frac{C}{A}}とおくと
  \begin{fleqn}[20pt]
    \begin{align*}
      \sum_{k=1}^{n}\left(a_{k}\cdot\frac{C}{A}-b_{k}\right)^2 = \frac{AB-C^2}{A} \tag*{(1)} \label{eq:ineq}
    \end{align*}
  \end{fleqn}
  左辺は$0$以上なので,右辺も$0$以上であり$AB\geq C^2$\\
  とくに\ref*{eq:ineq}で$AB=C^2$のとき,
  \dm{\sum_{k=1}^{n}\Bigl(a_{k}\cdot\frac{C}{A}-b_{k}\Bigr)^2=0}\\
  \dm{a_{k}\cdot\frac{C}{A}-b_{k}=0}が\dm{{k=1,\,2,\, \cdots,\, n}}で成り立つから\\
  \begin{fleqn}[20pt]
    \begin{align*}
      \frac{C}{A} = \frac{b_{k}}{a_{k}}\qquad
      \therefore \frac{b_{1}}{a_{1}}
                =\frac{b_{2}}{a_{2}}
                =\cdots
                =\frac{b_{n}}{a_{n}}
                =\frac{C}{A}
    \end{align*}
  \end{fleqn}
\end{syoumei*}

$n=2,\, 3$のときはベクトルの内積を用いる方法もあります.

\subsection{Jensenの定理(グラフの凹凸の利用)}
$f(x)$のグラフが下に凸のとき,
\begin{align*}
  \frac{f(x_{1})+f(x_{2})}{2}\geq f\Bigl(\frac{x_{1}+x_{2}}{2}\Bigr)
\end{align*}
等号は$x_{1} = x_{2}$のとき成り立つ.
曲線$y=f(x)$上の2点P$(x_{1},f(x_{1}))$,Q$(x_{2},f(x_{2}))$と
線分PQの中点
M\dm{\biggl(\frac{x_{1}+x_{2}}{2}, \frac{f(x_{1})+f(x_{2})}{2}\biggr)}
を考え,Mは曲線より上側(または曲線上)にあり,その真下の曲線上の点
N\dm{\biggl(\frac{x_{1}+x_{2}}{2}, f\Bigl( \frac{x_{1}+x_{2}}{2} \Bigr)\biggr)}
と$y$座標を比べればよい.

\subsection{パートナー交換}
$a<b,x<y$のとき$ax+by>ay+bx$

\section{最大値・最小値への応用}\label{sec:apply-to-max-min}
大小関係$f(x)\geq m$($m$は定数)が任意の$x$で成り立ち,
等号が成り立つような$x$が存在するならば$f(x)$の最小値は$m$である.
特に,$x>0$のとき,
\begin{gather*}
  x+\frac{1}{x}\geq 2\sqrt{x\cdot\frac{1}{x}}=2
  \: \text{等号は$x=\frac{1}{x}$,つまり$x=1$のとき成り立つ}
\end{gather*}
の形は頻出である.
そして,不等式の最大・最小値への応用では
「相加・相乗平均の不等式を使った後に右辺が定数になる場合に使えることが\textbf{多い}」
のであり,その形でなければ使えないわけではない.\\
\begin{rei*}
  \dm{p^2+q^2+p^2q^2=\frac{1}{3}} \: ($p,q$は正の整数)のとき,$pq$の最大値を求めよ.
  (2018,2001\quad 一橋大の解答中で)
  \begin{kai*}
    \begin{fleqn}[20pt]
      \begin{align*}
        \frac{1}{3}-p^2q^2=p^2+q^2\geq 2\sqrt{p^2q^2}=2pq
      \end{align*}
    \end{fleqn}
    より
    \begin{fleqn}[20pt]
      \begin{align*}
        (pq)^2 + 2(pq) - \frac{1}{3} \leq 0
      \end{align*}
    \end{fleqn}
    $pq$について解いて\dm{0<pq\leq -1+\sqrt{1+\frac{1}{3}}} \\
    等号は\dm{p=q=\sqrt{-1+\sqrt{\frac{4}{3}}}}のとき成り立つ.
  \end{kai*}
\end{rei*}


\end{document}
