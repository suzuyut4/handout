% -------------------------------------------------------%
% set styles and environments
% -------------------------------------------------------%

\pagestyle{plain}

% -------------------------------------------------------%

\titleformat*{\section}{\Large\bfseries}
\titleformat*{\subsection}{\large\bfseries}

% -------------------------------------------------------%

% 数式番号にセクション番号を併記する
\renewcommand{\theequation}{\thesection.\arabic{equation}}
\makeatletter
\@addtoreset{equation}{section}
\makeatother

% -------------------------------------------------------%

% 定理スタイルの定義
\newtheoremstyle{mystyle}
  {\topsep}   % スペース上
  {\topsep}   % スペース下
  {\normalfont}  % 本文のフォント
  {0pt}       % インデント
  {\bfseries} % タイトルのフォント
  {.}         % タイトルあとの句読点
  {.5em}      % タイトルと本文のスペース
  {\thmname{#1}\thmnumber{#2}\thmnote{#3}} % タイトルのスタイル
% 定理環境の作成
\theoremstyle{mystyle}
\newtheorem*{question*}{問題}
\newtheorem{problem}{} % 問題番号のみ
\newtheorem*{ans*}{解答}
\newtheorem*{practice*}{例題}
\newtheorem*{other*}{別解}
\newtheorem*{supple*}{補足}
\newtheorem*{append*}{補遺}
\newtheorem*{prf*}{証明}

% -------------------------------------------------------%
% % 不等式のため
\newtheoremstyle{inequationmystyle}
  {\topsep}   % スペース上
  {\topsep}   % スペース下
  {\normalfont}  % 本文のフォント
  {0pt}       % インデント
  {} % タイトルのフォント
  {}         % タイトルあとの句読点
  {3pt}      % タイトルと本文のスペース
  {【\,\textbf{\thmname{#1}}\thmnumber{#2}\,】\thmnote{#3}} % タイトルのスタイル
% 定理環境の作成
\theoremstyle{inequationmystyle}
\newtheorem*{syoumei*}{証明}
\newtheorem*{kai*}{解}
\newtheorem*{rei*}{例}


% -------------------------------------------------------%

\newtheoremstyle{problemstyle}
  {}   % スペース上
  {}   % スペース下
  {\normalfont}  % 本文のフォント
  {0pt}       % インデント
  {} % タイトルのフォント
  {}         % タイトルあとの句読点
  {.5em}      % タイトルと本文のスペース
  {\thmname{#1}\thmnumber{(#2)}\thmnote{#3}} % タイトルのスタイル
\theoremstyle{problemstyle}
\newtheorem{myprob}{}

% -------------------------------------------------------%

\newtheoremstyle{intproblemstyle}{20pt}{10pt}{\normalfont}{0pt}{}{}{.5em}
{\thmname{#1}\thmnumber{#2.\hspace*{5pt}}\thmnote{#3}} % タイトルのスタイル
\theoremstyle{intproblemstyle}
\newtheorem{intprob}{} % 積分

% -------------------------------------------------------%

\newtheoremstyle{cfstyle}{}{}{\normalfont}{0pt}{\itshape}{}{.5em}
{\thmname{#1}\thmnumber{#2}\thmnote{#3}} % タイトルのスタイル
\theoremstyle{cfstyle}
\newtheorem*{confer*}{cf.} % 積分

% -------------------------------------------------------%

% (1)のような環境。セクションごとに
\newtheoremstyle{lineupstyle}
  {}   % スペース上
  {}   % スペース下
  {\normalfont}  % 本文のフォント
  {0pt}       % インデント
  {} % タイトルのフォント
  {}         % タイトルあとの句読点
  {6pt}      % タイトルと本文のスペース
  {\thmname{#1}\thmnumber{(#2)}\thmnote{\textbf{#3}}} % タイトルのスタイル
\theoremstyle{lineupstyle}
\newtheorem{lineup}{}[section]
\makeatletter
\@addtoreset{lineup}{section} % lineupカウンターがsectionが更新されるたびにリセットされる
\makeatother
\renewcommand{\thelineup}{\arabic{lineup}}

% -------------------------------------------------------%

% 問題文の左の線の定義
\renewenvironment{leftbar}{%
\def\FrameCommand{\hspace{10pt}\vrule width 1.2pt \hspace{10pt}}%
\MakeFramed {\advance\hsize-\width \FrameRestore}}%
{\endMakeFramed}

% subsubsectionを太字の「問1」表示にする
\renewcommand{\thesubsubsection}{\large\textbf{問\arabic{subsubsection}}}

\newcommand{\prob}[1]{%
  \begin{question*}%
    ${}$%
    \vspace{-.5\baselineskip}%
    \begin{leftbar}%
      #1%
    \end{leftbar}%
  \end{question*}%
}
