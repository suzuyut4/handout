\documentclass[autodetect-engine,ja=standard, 10.5pt, a4paper, titlepage]{bxjsarticle}
% fleqn:数式を左詰めにする(titlepageの前に挿入可能)
% titlepage:表紙を独立させる
%\setlength{\mathindent}{50pt}
%-------------------------------------------------------%


\usepackage{graphicx} % Required for inserting images
\usepackage{titlesec}
\usepackage{caption}
\usepackage{amsmath} % 数式用
\usepackage{amssymb}
\usepackage{amsmath}
\usepackage{enumerate} % 箇条書き
\usepackage{comment} % コメントアウト
\usepackage[super]{cite} % 参考文献 上付き
\usepackage[version=4]{mhchem}
\usepackage{booktabs} % tableのmidrule
\usepackage{multirow} % tableのmultirow
\usepackage{float} % [H]で厳密に位置を固定
\usepackage{nccmath} % 数式を左に動かす
\usepackage{mathtools}
\usepackage{empheq}
\usepackage{accents} % \undertildeで下付きチルダ
\usepackage{nccmath}
\usepackage{amsthm}

%-------------------------------------------------------%

\pagestyle{plain} % empty:ページ番号削除

%-------------------------------------------------------%

% セクション・サブセクションの見出しのサイズ
\titleformat*{\section}{\Large\bfseries} % サイズ・太字
\titleformat*{\subsection}{\large\bfseries}

%-------------------------------------------------------%

\newcommand{\reference}[0]{\setlength{\hangindent}{18pt}\noindent}
\renewcommand{\refeq}[1]{\eqref{#1}式}
\newcommand{\reffig}[1]{図\ref{#1}}
\newcommand{\reftable}[1]{表\ref{#1}}
\newcommand{\degree}[0]{\mathrm{{}^\circ \hspace*{-0.5pt} C}}
\renewcommand{\deg}[0]{\mathrm{{}^\circ}}
\newcommand{\Vector}[1]{{\mbox{\boldmath$#1$}}}

\newcommand{\refcite}[2]{\cite{#1}${}^{\text{#2}}$}
\renewcommand{\citeform}[1]{#1)}
\makeatletter % \usepackage以外で@を含むときはこれで囲む
\renewcommand{\@biblabel}[1]{#1)}
\makeatother

%\numberwithin{equation}{section} % 式番号にセクションを併記する場合

%**************************************************************
\begin{document}
%\parindent = 0pt % 常に字下げなし
\centerline{\LARGE 数列と漸化式}
\vskip.3cm
\rightline{author\;:\;Yuta\;Suzuki}
\vskip.5cm
%twocolumn

\section*{数列と漸化式の基本}

\subsection*{漸化式}
そもそも,公式を用いて簡単に解ける漸化式は次の3つに限られる。

\begin{enumerate}[(i)]
  \item $a_{n+1} = a_n + d$  \rightarrow 公差$d$の等差数列
          \begin{fleqn}[20pt]
            \begin{align}\label{eq:tousa}
              \Rightarrow a_n = a_1 + (n-1)d
            \end{align}
          \end{fleqn}
  \item $a_{n+1} = ra_n$     \rightarrow 公比$r$の等比数列
          \begin{fleqn}[20pt]
            \begin{align}\label{eq:touhi}
              \Rightarrow a_n = a_1 r^{n-1}
            \end{align}
          \end{fleqn}
  \item $a_{n+1} = a_n + b_n$\rightarrow 階差数列が$\{b_n\}$の数列$\{a_n\}$
          \begin{fleqn}[20pt]
            \begin{align}\label{eq:kaisa}
              \Rightarrow (n\geqq 2 \text{のとき,})\:\:a_n = a_1 + \sum_{k=1}^{n-1}b_k
            \end{align}
          \end{fleqn}
\end{enumerate}
以下,\eqref{eq:tousa}の形を等差型,\eqref{eq:touhi}の形を等比型,\eqref{eq:kaisa}の形を階差型と呼ぶこととする。
よって,この形以外のほぼすべての漸化式はなんとかしてこの形に帰着させることが目的であり,
多くの場合は等比数列の形を導いて一般項を求めるということも覚えておくとよい。

せっかくなので\eqref{eq:kaisa}の形だけはここで証明しておこう。\\
\vskip.4\baselineskip
\noindent(証明)\\
$a_{n+1} = a_n + b_n \Leftrightarrow a_{n+1} - a_n = b_n$であるので,
$n$を$n-1,n-2,\cdots,2,1$として足し合わせると
  \begin{fleqn}[20pt]
    \begin{align*}
      &\begin{matrix}
       & a_n     & - & a_{n-1} & = & b_{n-1} \\
       & a_{n-1} & - & a_{n-2} & = & b_{n-2} \\
       & a_{n-2} & - & a_{n-3} & = & b_{n-3} \\
       &       & \vdots &    &\vdots &       \\
       & a_3     & - & a_2     & = & b_2     \\
    +) & a_2     & - & a_1     & = & b_1     \\
    \hline
       & a_n     & - & a_1     & = & b_{n-1} + b_{n-2} + \cdots + b_2 + b_1
      \end{matrix}
      \\
      &\Leftrightarrow a_n = a_1 + \sum_{k=1}^{n-1}b_k \qquad \qed
    \end{align*}
  \end{fleqn}

ここで,証明一行目において$n$が$n-1$以下のケースを考えることによってこの式を得ている。
しかし,この変形ができるのは$n-1\geqq 1$すなわち$n\geqq 2$のときのみである。
そのため,階差型においては条件$n\geqq 2$を忘れてはならない。
すなわち,一般項を求めたとき,$n=1$でもその式が成り立っているか必ず確認し,
成り立っていればそのように書き,成り立っていなければ場合分けして一般項を示す必要がある。
詳しくは,典型例の演習の解答を参照。

% \subsection*{等差数列型}
% 等差数列は
% \subsection*{等比数列型}

% \subsection*{階差数列型}

\subsection*{数列の和$S_n$}
また,一般項$a_n$で表される数列について第一項から第$n$項までの和を$S_n$で表すことがある。
すなわち
  \begin{fleqn}[20pt]
    \begin{align*}
      S_n = \sum_{k=1}^{n}a_k
    \end{align*}
  \end{fleqn}
であり。以下,特に断りがなければ$S_n$を数列$\{a_n\}$に対する和を表すものとする。

\section*{数列とその周辺}
\noindent 次の値を求めよ。
\begin{enumerate}[1.]
  \item
    \begin{fleqn}[20pt]
      \begin{align*}
        \sum_{k=1}^{n}k
        &= \frac{1}{2}n(n+1)
      \end{align*}
    \end{fleqn}

  \item
    \begin{fleqn}[20pt]
      \begin{align*}
        \sum_{k=1}^{n}k^2
        &= \frac{1}{6}n (n+1)(2n+1)
      \end{align*}
    \end{fleqn}
  \item
    \begin{fleqn}[20pt]
      \begin{align*}
        \sum_{k=1}^{n}k^3
        &= \left\{ \frac{1}{2}n(n+1) \right\}^2
      \end{align*}
    \end{fleqn}
  \item
    \begin{fleqn}[20pt]
      \begin{align*}
        \sum_{k=1}^{n}\cfrac{3}{k(k+2)}
        &= \frac{3}{2}\sum_{k=1}^{n}\left( \frac{1}{k} - \frac{1}{k+2} \right) \\
        &= \frac{3}{2} \left\{
          \left(1-\frac{1}{3}\right) + \left(\frac{1}{2}-\frac{1}{4}\right)
          + \left(\frac{1}{3}-\frac{1}{5}\right)
          + \cdots + \left(\frac{1}{n-1}-\frac{1}{n+1}\right)
          + \left(\frac{1}{n}-\frac{n}{n+2}\right)
          \right\} \\
        &= \frac{3}{2}\left(1+\frac{1}{2}-\frac{1}{n+1}-\frac{1}{n+2}\right) \\
        &= \frac{9}{4} - \frac{3(2n+3)}{2(n+1)(n+2)}
      \end{align*}
    \end{fleqn}
  \item
    \begin{fleqn}[20pt]
      \begin{align*}
        \sum_{k=1}^{n}\cfrac{3}{\sqrt{k+2} + \sqrt{k}}
        &= 3 \sum_{k=1}^{n}\frac{\sqrt{k+2}-\sqrt{k}}{(\sqrt{k+2}+\sqrt{k})(\sqrt{k+2}-\sqrt{k})} \\
        &= 3 \sum_{k=1}^{n}\frac{\sqrt{k+2}-\sqrt{k}}{2} \\
        &= \frac{3}{2}\left\{
          \left(\sqrt{3}-\sqrt{1}\right)
          + \left(\sqrt{4}-\sqrt{2}\right)
          + \cdots
          + \left(\sqrt{n+1}-\sqrt{n-1}\right)
          + \left(\sqrt{n+2}-\sqrt{n}\right)
        \right\} \\
        &= \frac{3}{2}\left( \sqrt{n+2} + \sqrt{n+1} - 1 - \sqrt{2} \right)
      \end{align*}
    \end{fleqn}
  \item
    \begin{fleqn}[20pt]
      \begin{align*}
        S_n = \sum_{k=1}^{n}\{(2k-1)\cdot 2^{k-1}\}
      \end{align*}
    \end{fleqn}
    とおく。
    このとき
      \begin{fleqn}[20pt]
        \begin{align*}
          &\begin{matrix}
            & S_n  & = & 1\cdot1 & + \hspace*{5pt} 3\cdot2 \hspace*{5pt} + \hspace*{5pt} 5\cdot2^2 &+& \cdots &+& (2n-1)\cdot2^{n-1} &                  \\
         -) & 2S_n & = &         & + \hspace*{5pt} 1\cdot2 \hspace*{5pt} + \hspace*{5pt} 3\cdot2^2 &+& \cdots &+& (2n-3)\cdot2^{n-1} & + (2n-1)\cdot2^n \\
         \hline
            & -S_n & = & 1\cdot1 & + \hspace*{5pt} 2\cdot2 \hspace*{5pt} + \hspace*{5pt} 2\cdot2^2 &+& \cdots &+& 2\cdot2^{n-1}      & - (2n-1)\cdot2^n \\
           \end{matrix} \\
           &\Leftrightarrow -S_n = 1 + 2^2 + 2^3 + \cdots + 2^{n-1} + 2^n - (2n-1)\cdot2^n \\
           &\Leftrightarrow -S_n = 1 + \frac{4(2^{n-1}-1)}{2-1} - (2n-1)\cdot2^n \\
           &\Leftrightarrow -S_n = 1 + 2^{n+1} - 4 - (2n-1)\cdot2^n \\
           &\Leftrightarrow  S_n = (2n-1-2)\cdot2^n + 3 \\
           &\therefore S_n = (2n-3)\cdot2^n + 3
        \end{align*}
      \end{fleqn}

  \item
          \begin{fleqn}[20pt]
            \begin{align*}
              &\cfrac{1}{1} + \cfrac{1}{1+2} + \cfrac{1}{1+2+3} + \cdots + \cfrac{1}{1+2+3+\cdots +(n-1)+n} \\
              &= \cdots = \frac{2n}{n+1}
            \end{align*}
          \end{fleqn}
  \item
          \begin{fleqn}[20pt]
            \begin{align*}
              &1+\cfrac{2}{3}+\cfrac{3}{3^2}+\cdots+\cfrac{n}{3^{n-1}} \\
              &= \cdots = \frac{9}{4} - \frac{2n+3}{4\cdot 3^{n-1}}
            \end{align*}
          \end{fleqn}
\end{enumerate}


\clearpage
\section*{漸化式\:\:典型例}
% %: 片方だけ使う別解
% 漸化式の問題はしばしば誘導がされる。
% そのため,何も誘導がなければ好きな方法で解けばよいが,
% 誘導があったりするとそれに乗らなくてはならないということになる。
% 典型例の演習を通して様々なパターンの漸化式の処理を学び,
% 様々な問題に対応できるようにしよう。

\begin{enumerate}[1.]

  \item $a_1 = 2,\; a_{n+1} = a_n + 3$
  \vskip.5\baselineskip
        <等差型>

          \begin{fleqn}[20pt]
            \begin{align*}
              a_n
              &= 2 + (n-1) \times 3 \\
              &= 3n - 1
            \end{align*}
          \end{fleqn}
  \vskip1.5\baselineskip
  \item $a_1 = 5,\; a_{n+1} = 7a_n$
  \vskip.5\baselineskip
        <等比型>
          \begin{fleqn}[20pt]
            \begin{align*}
              a_n
              &= 5\cdot 7^{n-1}
            \end{align*}
          \end{fleqn}
\vskip1.5\baselineskip
  \item $a_1 = 3,\; a_{n+1} = 3a_n -4$
  \vskip.5\baselineskip
        <特殊解型> \\
        特性方程式$\alpha = 3\alpha -4 \Leftrightarrow \alpha = 2$より
        $a_{n+1} - 2 = 3(a_n - 2) \quad \text{(等比型)}$と変形できるので
          \begin{fleqn}[20pt]
            \begin{align*}
              &a_n - 2 \\
              &= (a_1 - 2) \times 3^{n-1} \\
              &= 3^{n-1}
            \end{align*}
          \end{fleqn}
        よって,
          \begin{fleqn}[20pt]
            \begin{align*}
              a_n = 3^{n-1} + 2
            \end{align*}
          \end{fleqn}
        \noindent (補足) 慣れない間は$b_n = a_n -2$とおく習慣をつけるとよい。

\vskip1.5\baselineskip
  \item $a_1 = 1,\; a_{n+1} = a_n + 2n$
  \vskip.5\baselineskip
        <階差型> \\
        $n\geqq 2$のとき
          \begin{fleqn}[20pt]
            \begin{align*}
              a_n
              &= 1 + 2\sum_{k=1}^{n-1} k \\
              &= n^2 - n + 1
            \end{align*}
          \end{fleqn}
        これは$n = 1$のときも成り立つ。
\vskip1.5\baselineskip
  \item $a_1 = 1,\; a_{n+1} = a_n + 3^n - 4n$
  \vskip.5\baselineskip
        <階差型> \\
        $n\geqq 2$のとき
          \begin{fleqn}[20pt]
            \begin{align*}
              a_n
              &= 1 + \sum_{k=1}^{n-1} (3^k-4k) \\
              &= 1 + \frac{3^{n} -3}{3-1} - \frac{1}{2} (n-1) n\\
              &= \frac{1}{2}\cdot 3^{n} - 2n^2 + 2n - \frac{1}{2}
            \end{align*}
          \end{fleqn}
        これは$n = 1$のときも成り立つ。
\vskip1.5\baselineskip
  \item $a_1 = 1,\; a_2 = 2,\; a_{n+2} = a_{n+1} + 6a_n$
  \vskip.5\baselineskip
        <三項間漸化式> \\
        特性方程式$\alpha^2 = \alpha + 6 \Leftrightarrow \alpha = -2,3$より,
          \begin{fleqn}[20pt]
            \begin{align*}
              &\begin{dcases*}
                a_{n+2} + 2a_{n+1} = 3(a_{n+1} + 2a_n)  \\
                a_{n+2} - 3a_{n+1} = -2(a_{n+1} - 3a_n) \:\: \text{(等比型)}
              \end{dcases*}\\
              \vspace*{10pt}
              &\Rightarrow
              \begin{dcases*}
                a_{n+1} + 2a_n = 4\cdot 3^{n-1}\\
                a_{n+1} - 3a_n = -(-2)^{n-1}
              \end{dcases*}
            \end{align*}
          \end{fleqn}
        この2式より,
          \begin{fleqn}[20pt]
            \begin{align*}
              a_n = \cfrac{4\cdot 3^{n-1} + (-2)^{n-1}}{5}
            \end{align*}
          \end{fleqn}

        \vskip1\baselineskip
        \noindent(別解)片方だけ使う方法 \\
        特性方程式より
          \begin{fleqn}[20pt]
            \begin{align*}
              a_{n+2} + 2a_{n+1} = 3(a_{n+1} + 2a_n)
            \end{align*}
          \end{fleqn}
        よって,
          \begin{fleqn}[20pt]
            \begin{align*}
              a_{n+1} + 2a_n
              &= (a_2 + 2a_1)\cdot 3^{n-1} \\
              &= 4\cdot3^{n-1} \\
              \Leftrightarrow & a_{n+1} = -2a_n + 4\cdot3^{n-1} \\
              \Rightarrow     & \frac{a_{n+1}}{3^{n+1}} = -\frac{2}{3}\frac{a_n}{3^n} + \frac{4}{9} \:\: (\because 3^{n+1} \neq 0)
            \end{align*}
          \end{fleqn}
        すなわち,
          \begin{fleqn}[20pt]
            \begin{align*}
              & \frac{a_{n+1}}{3^{n+1}} - \frac{4}{15} = -\frac{2}{3}\left(\frac{a_n}{3^n} - \frac{4}{15} \right) \:\: \text{(等比型)} \\
              & \frac{a_n}{3^n} - \frac{4}{15} = \frac{1}{15} \cdot \left( -\frac{2}{3} \right)^{n-1} \\
              & a_n = \left( \frac{4}{15} + \frac{1}{15} \cdot \left( -\frac{2}{3} \right)^{n-1} \right) \cdot 3^n \\
              & \therefore a_n = \frac{
                4\cdot3^{n-1} + (-2)^{n-1}
              }{5}
            \end{align*}
          \end{fleqn}


\vskip1.5\baselineskip
  \item $a_1 = 2,\; a_{n+1} = 16a_n^5$
  \vskip.5\baselineskip
        <次数相違型> \\
        与えられた漸化式より$a_n> \\0$より,両辺に底を2として対数をとると,
          \begin{fleqn}[20pt]
            \begin{align*}
              \log_2 a_{n+1} = 5\log_2 a_n + 4
            \end{align*}
          \end{fleqn}
        ここで,$b_n = \log_2 a_n$とおくと$b_1 = 1$であるので
          \begin{fleqn}[20pt]
            \begin{align*}
              &b_{n+1} = 5b_n + 4\\
              &b_{n+1} + 1 = 5(b_n + 1) \:\: \text{(等比型)}\\
              &b_n = 2\cdot 5^{n-1} -1
            \end{align*}
          \end{fleqn}
        よって,
          \begin{fleqn}[20pt]
            \begin{align*}
              a_n = 2^{2\cdot 5^{n-1} -1}
            \end{align*}
          \end{fleqn}
\vskip1.5\baselineskip
  \item $a_1 = 1,\; a_{n+1} = 2a_n + n^2 -6$
  \vskip.5\baselineskip
        この漸化式が,$a_{n+1} + p(n+1)^2 + q(n+1) + r = 2(a_n + pn^2 + qn + r)$と
        表せるとすると,これを展開した$a_{n+1} = 2a_n + pn^2 + (-2p + q)n + (-p-q+r)$と
        与えられた漸化式の係数は常に等しいので(恒等式)
          \begin{fleqn}[20pt]
            \begin{align*}
              &\begin{dcases*}
                 p = 1\\
                -2p +q =0 \\
                - p -q +r =-6
              \end{dcases*}
              \\
              &\therefore (p,q,r) = (1\;,\;2\;,\;-3)
            \end{align*}
          \end{fleqn}
        ゆえに
          \begin{fleqn}[20pt]
            \begin{align*}
              &a_{n+1} + (n+1)^2 + 2(n+1) - 3
                      = 2( a_n + n^2 + 2n - 3 ) \\
              & a_n + n^2 + 2n - 3 = (a_1 + 1 + 2 - 3 )\cdot 2^{n-1} = \cdot 2^{n-1} \\
              & \therefore a_n = 2^{n-1} - n^2 - 2n + 3
            \end{align*}
          \end{fleqn}
% $b_n = a_{n+1} - a_n$ と置いたあと,$c_n = b_{n+1} - b_n$と置かせる誘導
\vskip1\baselineskip
この種の問題は上のような解法が最もよいと思われるが,次の誘導がされる場合がある。\\
        \noindent(問題)\\
        $b_n = a_{n+1} - a_n$ とおいたときの$b_n$の漸化式を導き,さらに
        $c_n = b_{n+1} - b_n$とおいたときの$c_n$を求めることで$a_n$の一般項を求めよ。 \\
        \noindent(略解)\\
        与えられた漸化式より$a_{n+2} = 2a_{n+1} + (n+1)^2 - 6$であるので,与式とこの式の差をとって
          \begin{fleqn}[20pt]
            \begin{align*}
              &a_{n+2} - a_{n+1} = 2(a_{n+1} - a_n) + 2n +1 \\
              & b_{n+1} = 2b_n + 2n + 1 \:\: (b_1 = -4,\: b_2 = -5)
            \end{align*}
          \end{fleqn}
        さらに,$b_{n+2} = 2b_{n+1} + 2(n+1) + 1$であるので,上式とこの式の差をとって
          \begin{fleqn}[20pt]
            \begin{align*}
              &b_{n+2} - b_{n+1} = 2(b_{n+1} - b_n) + 2 \\
              &c_{n+1} = 2c_n + 2 \:\: (c_1 = -1) \\
              & \Leftrightarrow c_{n+1} + 2 = 2(c_n + 2) \:\: \text{(等比型)}
            \end{align*}
          \end{fleqn}
        すなわち,
          \begin{fleqn}[20pt]
            \begin{align*}
              &c_n = 2^{n-1} - 2
            \end{align*}
          \end{fleqn}
        また,$c_n$は$b_n$の階差数列より,
          \begin{fleqn}[20pt]
            \begin{align*}
              &b_n = b_1 + \sum_{k=1}^{n-1}c_k \:\:(n\geqq 2) \\
              &\Rightarrow b_n = 2^{n-1} - 2n - 3 \:\:(\text{これは$n=1$でも成り立つ})
            \end{align*}
          \end{fleqn}
        $a_n$についても,階差数列$b_n$が上のように得られるので,
          \begin{fleqn}[20pt]
            \begin{align*}
              &a_n = a_1 + \sum_{k=1}^{n-1}b_k \:\:(n\geqq 2) \\
              &\therefore a_n = 2^{n+1} - n^2 - 2n + 3 \:\:(\text{これは$n=1$でも成り立つ})
            \end{align*}
          \end{fleqn}

\vskip1.5\baselineskip
  \item $a_1 = 1,\; a_{n+1} = 4a_n + n\cdot 2^n$
  \vskip.5\baselineskip
        <指数型> \\
        両辺を$2^{n+1}$でわると
          \begin{fleqn}[20pt]
            \begin{align*}
              \frac{a_{n+1}}{2^{n+1}} = 2 \frac{a_n}{2^n} + \frac{n}{2}
            \end{align*}
          \end{fleqn}
        これが$\cfrac{a_{n+1}}{2^{n+1}} + p(n+1) + q = 2 \left( \cfrac{a_n}{2^n} + pn + q \right) $
        と表せるとすると,これを展開した\\
        $\cfrac{a_{n+1}}{2^{n+1}}=2\cfrac{a_n}{2^n} + pn + (-p + q)$
        と与えられた漸化式の係数は常に等しいので
          \begin{fleqn}[20pt]
            \begin{align*}
              (p,q) = \left(\frac{1}{2}\;,\;\frac{1}{2}\right)
            \end{align*}
          \end{fleqn}
        すなわち
          \begin{fleqn}[20pt]
            \begin{align*}
              \frac{a_{n+1}}{2^{n+1}} + \frac{1}{2}(n+1) + \frac{1}{2}
              = 2 \left( \frac{a_n}{2^n} + \frac{1}{2}n + \frac{1}{2} \right) \:\: \text{(等比型)}
            \end{align*}
          \end{fleqn}
        ゆえに
          \begin{fleqn}[20pt]
            \begin{align*}
              &\frac{a_n}{2^n}+\frac{1}{2}n+\frac{1}{2} = \frac{3}{2}2^{n-1} \\
              &a_n =2^n \left( 3\cdot 2^{n-2} - \frac{n}{2} - \frac{1}{2} \right) \\
              &\therefore a_n = 3\cdot 2^{2n-2} - (n+1)2^{n-1}
            \end{align*}
          \end{fleqn}

\vskip1.5\baselineskip
  \item $a_1 = 1,\; a_2=3,\; a_{n+2} = 4a_{n+1} - 4a_n$
  \vskip.5\baselineskip
        <三項間漸化式-重解-> \\
        特性方程式$\alpha^2 = 4\alpha - 4 \Leftrightarrow \alpha = 2$より
          \begin{fleqn}[20pt]
            \begin{align*}
              a_{n+2} - 2a_{n+1} = 2(a_{n+1} - 2a_n) \:\: \text{(等比型)}
            \end{align*}
          \end{fleqn}
        よって,
          \begin{fleqn}[20pt]
            \begin{align*}
              a_{n+1} - 2a_n = 2^{n-1}
            \end{align*}
          \end{fleqn}
        $-2a_n$を移項し,両辺を$2^{n+1}$でわると
          \begin{fleqn}[20pt]
            \begin{align*}
              &\frac{a_{n+1}}{2^{n+1}} = \frac{a_n}{2^n} + \frac{1}{4} \:\: \text{(等差型)}\\
              &\frac{a_n}{2^n} = \frac{1}{2} + \frac{1}{4}(n-1) = \frac{1}{4}(n+1) \\
              & \therefore a_n = (n+1)2^{n-2}
            \end{align*}
          \end{fleqn}

\vskip1.5\baselineskip
  \item $a_1 = 1,\; a_2 = 4,\; a_{n+2} = 4a_{n+1} - 3a_n - 2$
  \vskip.5\baselineskip
        <三項間漸化式-定数項あり-> \\
        特性方程式$\alpha^2 = 4\alpha -3 \Leftrightarrow \alpha = 1,3$より
        \begin{fleqn}[20pt]
          \begin{align*}
            &\begin{dcases*}
              a_{n+2} - a_{n+1} = 3(a_{n+1} - a_n) - 2  \\
              a_{n+2} - 3a_{n+1} = (a_{n+1} - 3a_n) - 2
            \end{dcases*}
          \end{align*}
        \end{fleqn}
        また,$b_n = a_{n+1} - a_n,\:c_n = a_{n+1} - 3a_n$とおくと$b_1 = 3,\:c_1 = 1$であり
          \begin{fleqn}[20pt]
            \begin{align*}
              &\begin{dcases*}
                b_{n+1} = 3b_n - 2 \\
                c_{n+1} = c_n -2
              \end{dcases*}
              \\
              &\Leftrightarrow
              \begin{dcases*}
                b_{n+1} - 1 = 3(b_n - 1) \:\: \text{(等比型)} \\
                c_{n+1} = c_n - 2  \:\: \text{(等差型)}
              \end{dcases*}
              \\
              &\Leftrightarrow
              \begin{dcases*}
                b_n = a_{n+1} - a_n = 2\cdot 3^{n-1} + 1 \\
                c_n = a_{n+1} - 3a_n = -2n + 3
              \end{dcases*}
            \end{align*}
          \end{fleqn}
        この2式から$a_n = \cfrac{b_n - c_n}{2}$であるので一般項は
        \begin{fleqn}[20pt]
          \begin{align*}
            a_n
            &= \frac{2\cdot 3^{n-1} + 2n - 2}{2} \\
            &= 3^{n-1} + n - 1
          \end{align*}
        \end{fleqn}
\vskip1.5\baselineskip
  \item $a_1 = 1,\: a_{n+1} = \cfrac{a_n}{2a_n + 3}$
  \vskip.5\baselineskip
        <分数型-逆数置換-> \\
        ある$k\in \mathbb{N}$に対して$a_k = 0$であると仮定すると
          \begin{fleqn}[20pt]
            \begin{align*}
              0 = a_k = \cfrac{a_{k-1}}{2a_{k-1} + 3} \Leftrightarrow a_{k-1} = 0
            \end{align*}
          \end{fleqn}
        であり,これを繰り返すことで$0 = a_{k-1} = a_{k-2} = \cdots = a_1$となるが,これは$a_1 = 1$に矛盾。\\
        よって,任意の自然数$n$で$a_n \neq 0$であるので,$b_n = \cfrac{1}{a_n}$とおくと,$b_1 = 1$で
          \begin{fleqn}[20pt]
            \begin{align*}
              b_{n+1}
              &= \frac{1}{a_{n+1}} \\
              &= \frac{2a_n + 3}{a_n} \\
              &= \frac{3}{a_n} + 2 \\
              &= 3b_n + 2 \\
              \Leftrightarrow b_{n+1} &+ 1 = 3(b_n + 1)
            \end{align*}
          \end{fleqn}
        すなわち,
          \begin{fleqn}[20pt]
            \begin{align*}
              &b_n = 2\cdot 3^{n-1} - 1 \\
              & \Rightarrow a_n = \frac{1}{2\cdot 3^{n-1} - 1}
            \end{align*}
          \end{fleqn}

\vskip1.5\baselineskip
  \item $a_1 = 3,\; a_{n+1} = \cfrac{3a_n-4}{a_n-2}$
  \vskip.5\baselineskip
        <分数型-特性方程式-> \\
        特性方程式$\alpha =\cfrac{3\alpha - 4}{\alpha - 2} \Leftrightarrow \alpha = 1,4$より \\
        ここで,ある$k\in \mathbb{N}$に対して,$a_k = 1$であると仮定すると
          \begin{fleqn}[20pt]
            \begin{align*}
              a_k = 1 = \frac{3a_{k-1} - 4}{a_{k-1} - 2} \Leftrightarrow a_{k-1} = 1
            \end{align*}
          \end{fleqn}
        であり,これを繰り返すことで$1 = a_k = a_{k-1} = \cdots = a_1$となるが,これは$a_1 = 3$に矛盾。\\
        よって,任意の自然数$n$で$a_n \neq 1$であるので,$b_n = \cfrac{a_n - 4}{a_n - 1}$とおくと$b_1 = -\cfrac{1}{2}$で
          \begin{fleqn}[20pt]
            \begin{align*}
              b_{n+1}
              &= \frac{3a_{n+1} - 4}{a_{n+1} - 1}
              = \cfrac{\cfrac{3a_n-4}{a_n-2}-4}{\cfrac{3a_n-4}{a_n-2}-1} \\
              &= -\frac{1}{2}b_n \:\: \text{(等比型)}
            \end{align*}
          \end{fleqn}
        よって,$b_n = \left(-\frac{1}{2}\right)^n$であり,$b_n \neq 1$ \\
        また,
          \begin{fleqn}[20pt]
            \begin{align*}
              b_{n} = \frac{a_n - 4}{a_n - 1} \Rightarrow a_n = \frac{b_n - 4}{b_n - 1} \:\:(\because b_n \neq 1)
            \end{align*}
          \end{fleqn}
        であるので,
          \begin{fleqn}[20pt]
            \begin{align*}
              a_n
              = \cfrac{\left(-\cfrac{1}{2}\right)^n - 4}{\left(-\cfrac{1}{2}\right)^n - 1}
              = \frac{2^{n+2} - (-1)^n}{2^n - (-1)^n}
            \end{align*}
          \end{fleqn}

\vskip1.5\baselineskip
  \item $a_1 = 3,\; a_{n+1} = \cfrac{3a_n-4}{a_n-1}$
  \vskip.5\baselineskip
        <分数型-重解-> \\
        特性方程式$\alpha =\cfrac{3\alpha - 4}{\alpha - 1} \Leftrightarrow \alpha = 2$より \\
        ここで,ある$k\in \mathbb{N}$に対して,$a_k = 2$であると仮定すると,
          \begin{fleqn}[20pt]
            \begin{align*}
              a_k = 2 = \frac{3a_{k-1} - 4}{a_{k-1} - 1} \Leftrightarrow a_{k-1} = 2
            \end{align*}
          \end{fleqn}
        であり,これを繰り返すことで$2 = a_k = a_{k-1} = \cdots = a_1$となるが,これは$a_1 = 3$に矛盾。\\
        よって,任意の自然数$n$で$a_n \neq 2$であるので,$b_n = \cfrac{1}{a_n - 2}$とおくと$b_1 = 1$で
          \begin{fleqn}[20pt]
            \begin{align*}
              b_{n+1}
              &= \frac{1}{a_{n+1} - 2}
               = \cfrac{1}{\cfrac{3a_n - 4}{a_n - 1} - 2} \\
              &= \frac{a_n - 1}{a_n - 2} \\
              &= b_n + 1 \:\: \text{(等差型)} \\
              &\Rightarrow b_n = 1 + (n - 1) \cdot 1 = n \:\: \text{であり} b_n \neq 0
            \end{align*}
          \end{fleqn}
        また,
          \begin{fleqn}[20pt]
            \begin{align*}
              b_n = \frac{1}{a_n - 2} \Rightarrow a_n = \frac{1}{b_n} + 2 \:\: (\because b_n \neq 0)
            \end{align*}
          \end{fleqn}
        であるので,
          \begin{fleqn}[20pt]
            \begin{align*}
              a_n = \frac{1}{n} + 2 \left( = \frac{2n + 1}{n} \right)
            \end{align*}
          \end{fleqn}

\vskip1\baselineskip
        \noindent(補遺) \\
        分数型$a_{n+1} = \cfrac{pa_n + q}{ra_n + s}$について特性方程式$\alpha = \cfrac{p\alpha + q}{r\alpha + s}$
        を考え,その2解を$\lambda_1,\lambda_2$とおいたときに
          \begin{fleqn}[20pt]
            \begin{align*}
              b_n = \frac{a_n - \lambda_1}{a_n - \lambda_2}
            \end{align*}
          \end{fleqn}
        あるいは
          \begin{fleqn}[20pt]
            \begin{align*}
              b_n = \frac{1}{a_n - \lambda_1}
            \end{align*}
          \end{fleqn}
        とおくことで漸化式を解くことができた。 \\
        さて,分数型での$q=0$である場合に数列$a_n$の逆数を置き換えたのは
        特性方程式の解のうち一つが必ず$0$であるからである。
          \begin{fleqn}[20pt]
            \begin{align*}
              &\alpha = \frac{p\alpha}{r\alpha + s} \Leftrightarrow \alpha = 0,\frac{p-s}{r}
            \end{align*}
          \end{fleqn}
        よって,上の置換における$\lambda_1 = 0$の場合であると思えば
        よいということである。
        また,逆数置換をするということはあくまで特性方程式の解の一つが$\lambda_1=0$であるということだが
        $\lambda_1 = \cfrac{p-s}{r}$のようにもう一つの解を置いても問題なく解くことはできる。 \\
        加えて,$b_n = \cfrac{1}{a_n - \lambda_1}$とおくパターンはと特性方程式が重解の場合も
        解けるということだから,重解でない場合にも同じようにこの置換で解くことができる。

\vskip1.5\baselineskip
  \item $a_1 = 1,\; a_{n+1}=(n+1)a_n$
  \vskip.5\baselineskip
        <階比型> \\
        両辺を$(n+1)!\neq 0$でわると
          \begin{fleqn}[20pt]
            \begin{align*}
              \frac{a_{n+1}}{(n+1)!} = \frac{a_n}{n!} \:\: \text{(等比型)}
            \end{align*}
          \end{fleqn}
        よって
          \begin{fleqn}[20pt]
            \begin{align*}
              &\frac{a_n}{n!} = \frac{a_1}{1!} = 1 \\
              &\therefore a_n = n!
            \end{align*}
          \end{fleqn}
\vskip1.5\baselineskip
  \item $a_1 = 1,\; (n+2)a_{n+1}=na_n$
  \vskip.5\baselineskip
        <階比型> \\
        両辺を$n+1$倍すると
          \begin{fleqn}[20pt]
            \begin{align*}
              (n+2)(n+1)a_{n+1} = (n+1)n\;a_n \:\: \text{(等比型)}
            \end{align*}
          \end{fleqn}
        よって
          \begin{fleqn}[20pt]
            \begin{align*}
              &(n+1)na_n = (1+1)\cdot 1 a_1 = 2 \\
              &\therefore a_n = \frac{2}{n(n+1)}
            \end{align*}
          \end{fleqn}
\vskip1.5\baselineskip
  \item $a_1 = 1,\; na_{n+1}=2(n+1)a_n+n(n+1)$
  \vskip.5\baselineskip
        <階比・等差型> \\
        両辺を$n(n+1)\neq 0$でわると
          \begin{fleqn}[20pt]
            \begin{align*}
              &\frac{a_{n+1}}{n+1} = 2\frac{a_n}{n} + 1 \\
              &\frac{a_{n+1}}{n+1} + 1 = 2\left( \frac{a_n}{n} + 1 \right) \:\: \text{(等比型)}
            \end{align*}
          \end{fleqn}
        よって
          \begin{fleqn}[20pt]
            \begin{align*}
              &\frac{a_n}{n} + 1 = 2\cdot 2^{n-1} =2^n \\
              &\therefore a_n = n (2^n-1)
            \end{align*}
          \end{fleqn}
  \item $a_1 = 2,\; a_{n+1} = \cfrac{n+2}{n}a_n + 1$
  \vskip.5\baselineskip
       両辺を$(n+1)(n+2)\neq 0$でわると
         \begin{fleqn}[20pt]
          \begin{align*}
            \frac{a_{n+1}}{(n+1)(n+2)} = \frac{a_n}{n(n+1)}
          \end{align*}
        \end{fleqn}
      より$b_n = \cfrac{a_n}{n(n+1)}$とおくと$b_1=1$であり,
        \begin{fleqn}[20pt]
          \begin{align*}
            b_{n+1}
            &= b_n + \frac{1}{(n+1)(n+2)} \:\: \text{(階差型)}
          \end{align*}
        \end{fleqn}
      である。よって$n\geqq 2$のとき
        \begin{fleqn}[20pt]
          \begin{align*}
            b_n
            &= b_1 + \sum_{k=1}^{n-1}\frac{1}{(k+1)(k+2)} \\
            &= 1 + \sum_{k=1}^{n-1}\left( \frac{1}{k+1}-\frac{1}{k+2} \right) \\
            &= 1 + \frac{1}{2} - \frac{1}{n+1} \\
            &= \frac{3n+1}{2(n+1)}
          \end{align*}
        \end{fleqn}
      また,$n=1$のとき$b_1 = \cfrac{3\cdot1 + 1}{2(1+1)} = 1$より,$n=1$のときもこれは成り立つ。
      以上より,
        \begin{fleqn}[20pt]
          \begin{align*}
            a_n
            &= n(n+1)b_n \\
            &= \frac{n(3n+1)}{2}
          \end{align*}
        \end{fleqn}


  \vskip1.5\baselineskip
  \item $a_1 = 1,\; a_{n+1} = 2^{2n-2}(a_n)^2$
  \vskip.5\baselineskip
        以下,数列$a_n$が任意の自然数$n$に対して$a_n>0$であることを示す。
        \begin{enumerate}[(i)]
          \item $n=1$のとき
                $a_1=1>0$
          \item $n=k \:(k\in \mathbb{N})$のとき,$a_n>0$と仮定すると$a_k>0\quad \cdots (1)$\\
                また,
                  \begin{fleqn}[20pt]
                    \begin{align*}
                      a_{k+1} = 2^{2k-2}(a_k)^2 > 0
                    \end{align*}
                  \end{fleqn}
                より$n=k+1$のときも成り立つ。
        \end{enumerate}
        (i),(ii)より数学的帰納法からすべての自然数$n$で$a_n>0$である。\\
        漸化式の両辺に底を2とする対数をとると,
          \begin{fleqn}[20pt]
            \begin{align*}
              \log_2a_{n+1} = (2n-2)+2\log_2a_n
            \end{align*}
          \end{fleqn}
        である。$b_n = \log_2a_n$とおくと,$b_1=0$であり$b_{n+1} = 2b_n + 2n - 2$となるが\\
        これが$b_{n+1} + p(n+1) + q = 2(b_n + pn + q)$と表せるとすると,\\
        これを展開した$b_{n+1} = 2b_n + pn + (-p+q)$と係数が等しいので
          \begin{fleqn}[20pt]
            \begin{align*}
              &
              \begin{dcases*}
                p=2 \\
                q-p=-2
              \end{dcases*}
              \\
              &\Rightarrow
              \begin{dcases*}
                p = 2 \\
                q = 0
              \end{dcases*}
            \end{align*}
          \end{fleqn}
        より
        $b_{n+1} + 2(n+1) = 2(b_n + 2n) \:\: \text{(等比型)}$であるので,
          \begin{fleqn}[20pt]
            \begin{align*}
              &b_n + 2n = (b_1 + 2) \cdot 2^{n-1} = 2^n \\
              &\Rightarrow b_n = 2^n - 2n \\
              &\Rightarrow a_n = 2^{2^n-2n}
            \end{align*}
          \end{fleqn}


  \vskip1.5\baselineskip
  \item $S_n = 3a_n + 2n - 1$
  \vskip.5\baselineskip
        $S_{n+1} = 3a_{n+1} + 2n + 1$および$S_{n+1}=S_n + a_{n+1}$より
          \begin{fleqn}[20pt]
            \begin{align*}
              &S_{n+1} - S_n = 3a_{n+1} - 3a_n + 2 = a_{n+1} \\
              &\Rightarrow a_{n+1} = \frac{3}{2}a_n - 1 \\
              &a_{n+1} - 2 = \frac{3}{2}(a_n - 2) \\
              &\therefore a_n = \frac{5}{2}\cdot\left(\frac{3}{2}\right)^{n-1}+2
            \end{align*}
          \end{fleqn}



\end{enumerate}


\clearpage
\section*{漸化式\:\:応用編}

\begin{enumerate}[1.]
  \item
  $n$を自然数として次の条件で定められた数列$\{a_n\}$について2通りの解き方を考えよう。 % (福井大 改 +α)

    \begin{align*}
      a_1 = 1,\:\: a_{n+1} = \frac{3}{n} (a_1 + a_2 + a_3 + \cdots + a_n)  \quad\cdots (*)
    \end{align*}

  \begin{enumerate}[(1)]
    \item $a_2,\;a_3\;a_4$を計算せよ。\\
          $a_2 = 3,\: a_3 = 6,\:a_4 = 10$
\vskip1.5\baselineskip
    \item 一般項$\{a_n\}$を推定し,それが正しいことを数学的帰納法を用いて示せ。\\
    %* 推定のコツ:素因数分解,定数倍してみる
    %* 帰納法の種類:普通・前二つ仮定・それより前全部仮定(・無限降下法)
            \begin{fleqn}[20pt]
              \begin{align*}
                &2\times a_1 = 2 = 1\times 2\\
                &2\times a_2 = 6 = 2\times 3 \\
                &2\times a_3 = 12 = 3\times 4 \\
                &2\times a_4 = 20 = 4\times 5
              \end{align*}
            \end{fleqn}
          であるので$a_n = \cfrac{n(n+1)}{2}\cdots (1)$であると推定できる。 \\
          \begin{enumerate}[(i)]
            \item $n = 1$のとき
                    \begin{fleqn}[20pt]
                      \begin{align*}
                        a_1 = 1 = \frac{2\times 1}{2}
                      \end{align*}
                    \end{fleqn}
                  より,これは(1)を満たす。
            \item ある自然数$k$に対して,$n \leqq k$で(1)が成り立つとき,
                  $a_i = \cfrac{i(i+1)}{2}\quad(i\leqq k) \:\: \cdots (2)$である。\\
                  また,与えられた漸化式$(*)$を用いると
                    \begin{fleqn}[20pt]
                      \begin{align*}
                        a_{k+1}
                        &= \frac{3}{k}(a_1 + \cdots + a_k) \\
                        &= \frac{3}{k}\sum_{i=1}^{k}a_i \\
                        &= \frac{3}{k}\sum_{i=1}^{k}\frac{i(i+1)}{2} \quad (\because (2))\\
                        &= \frac{3}{2k}\sum_{i=1}^{k}(i^2 + i) \\
                        &= \frac{3}{2k}\left( \frac{k(k+1)(2k+1)}{6} + \frac{k(k+1)}{2} \right) \\
                        &= \frac{k^2+3k+2}{2} \\
                        &= \frac{(k+1)(k+2)}{2}
                      \end{align*}
                    \end{fleqn}
                  より,$n = k + 1$でも(1)が成り立つ。

            (i),(ii)より数学的帰納法からすべての自然数$n$に対して$a_n = \cfrac{n(n+1)}{2}$である。
          \vskip1\baselineskip
            \noindent(別解)普通の数学的帰納法で示すパターン\\
            (ii)\:\: ある自然数$k$に対して,(1)が成り立つとき
              \begin{fleqn}[20pt]
                \begin{align*}
                  a_k = \frac{3}{k-1}(a_1 + a_2 + \cdots + a_{k-1}) = \frac{k(k+1)}{2} \quad \cdots (2)
                \end{align*}
              \end{fleqn}
            であるので,漸化式$(*)$は
              \begin{fleqn}[20pt]
                \begin{align*}
                  a_{k+1}
                  &= \frac{3}{k}(a_1 + a_2 + \cdots + a_k) = \frac{3}{k}(a_1 + a_2 + \cdots + a_{k-1}) + \frac{3a_k}{k} \\
                  &= \frac{k-1}{k}\left\{ \frac{3}{k-1}(a_1 + a_2 + \cdots + a_{k-1}) \right\} + \frac{3a_k}{k} \\
                  &= \frac{k-1}{k}\cdot a_k + \frac{3a_k}{k} \qquad (\because (2))\\
                  &= \frac{k+2}{k}a_k = \frac{k+2}{k}\cdot \frac{k(k+1)}{2} \\
                  &= \frac{(k+1)(k+2)}{2}
                \end{align*}
              \end{fleqn}
            より,$n = k + 1$でも(1)が成り立つ。

          \end{enumerate}
\vskip1\baselineskip
        \noindent(補遺) \\
        数学的帰納法は主に次の3パターンがある。
        \vskip.5\baselineskip
          \begin{enumerate}[(i)]
            \item $n=1$で示して「$n=k$で成り立つ$\Rightarrow$$n=k+1$で成り立つ」を示す。
            \item $n=1$と$n=2$で示して「$n=k,k+1$で成り立つ$\Rightarrow$$n=k+2$で成り立つ」を示す。
            \item $n=1$で示して「$n\leqq k$で成り立つ$\Rightarrow$$n=k+1$で成り立つ」を示す。
          \end{enumerate}
        \vskip.5\baselineskip
        上の問題では(i),(iii)を用いて示す方法を述べたため,ついでに(ii)の使い所も示しておこう。

        \noindent(例題) \\
        $x,y\in \mathbb{R}$について,$x+y,xy$がいずれも偶数であるとする。
        このとき,$n\in \mathbb{N}$に対して$x^n+y^n$も偶数となることを示せ。\\
        <解> \\
        $x+y$,$xy$が偶数であるので$l,m\in \mathbb{Z}$を用いて
          \begin{fleqn}[20pt]
            \begin{align*}
              x+y = 2l, xy = 2m
            \end{align*}
          \end{fleqn}
        と表せる。
        \begin{enumerate}[(i)]
            \item $n=1$のとき
                  $x^1+y^1=x+y=2l$より$x^n+y^n$は偶数である。
            \item $n=2$のとき
                  $x^2+y^2=(x+y)^2-2xy=4l^2-4m=2(2l^2-2m)$より, \\
                  $x^n+y^n$は偶数である。
            \item $n=k,k+1\:(k\in \mathbb{N})$のときに成り立つと仮定すると
                    \begin{fleqn}[20pt]
                      \begin{align*}
                        &x^k + y^k = 2x_k (x_k \in \mathbb{Z})\\
                        &x^{k+1} + y^{k+1} = 2x_{k+1} (x_{k+1} \in \mathbb{Z})\\
                      \end{align*}
                    \end{fleqn}
                  である。このとき,
                    \begin{fleqn}[20pt]
                      \begin{align*}
                        x_{k+2} + y_{k+2}
                        &= (x+y)(x^{k+1} + y^{k+1}) - xy^{k+1} - x^{k+1}y \\
                        &= 2l \cdot2x_{k+1} - 2m\cdot x_k \\
                        &= 2(2lx_{k+1} - 2mx_{k})
                      \end{align*}
                    \end{fleqn}
                  より$n=k+2$のときも偶数。
        \end{enumerate}
        (i),(ii),(iii)より数学的帰納法から$x+y,xy$がいずれも偶数であるとき,
        すべての自然数$n$に対して$x^n+y^n$も偶数である。




\vskip1.5\baselineskip
    \item 上の漸化式$(*)$について,$a_1 + a_2 + a_3 + \cdots + a_{n-1}$を$a_n$と$n$を用いて表せ。\\
          $(*)$より
            \begin{fleqn}[20pt]
              \begin{align*}
                &a_n = \frac{3}{n-1}(a_1 + a_2 + \cdots + a_{n-1}) \\
                &\Rightarrow a_1 + a_2 + \cdots + a_{n-1} = \frac{(n-1)\;a_n}{3}
              \end{align*}
            \end{fleqn}

\vskip1.5\baselineskip
    \item $a_{n+1}$と$a_n$の関係を導いた上で,一般項$a_n$を$n$を用いて表せ。\\
          (3)より,$(*)$は
            \begin{fleqn}[20pt]
              \begin{align*}
                a_{n+1}
                &= \frac{3}{n}(a_1 + a_2 + \cdots + a_{n-1}) + \frac{3}{n}a_n \\
                &= \frac{3}{n}\frac{(n-1)\;a_n}{3} + \frac{3}{n}a_n \\
                &= \frac{n+2}{n}a_n
              \end{align*}
            \end{fleqn}
          両辺を$(n+1)(n+2)$でわることにより
            \begin{fleqn}[20pt]
              \begin{align*}
                &\frac{a_{n+1}}{(n+2)(n+1)} = \frac{a_n}{(n+1)n} \:\: \text{(等比型)} \\
                & \Rightarrow \frac{a_n}{(n+1)n} = \frac{a_1}{2\times 1} = \frac{1}{2} \\
                & \therefore a_n = \frac{n(n+1)}{2}
              \end{align*}
            \end{fleqn}
  \end{enumerate}

\clearpage

  \item
  \begin{enumerate}[(1)]
    \item 次の初項,二つの漸化式で与えられる数列$\{a_n\},\, \{b_n\}$を考える。\\
      \begin{align*}
        &a_1 = 5,\quad b_1 = 3\\
        &a_{n+1} = 5a_n + 3b_n \cdots (i) \\
        &b_{n+1} = 3a_n + 5b_n \cdots (ii) \\
      \end{align*}
    2つの数列$\{a_n\pm b_n\}$を求め,一般項$a_n,\, b_n$を求めよ。\\

    (i),(ii)式を足し合わせたものと引いたものを考えると
      \begin{fleqn}[20pt]
        \begin{align*}
          \begin{dcases*}
            a_{n+1} + b_{n+1} = 8(a_n + b_n) \\
            a_{n+1} - b_{n+1} = 2(a_n - b_n) \:\: \text{(等比型)}
          \end{dcases*}
        \end{align*}
      \end{fleqn}
    となる。$a_1+b_1 = 8$,$a_1-b_1 = 2$より,
      \begin{fleqn}[20pt]
        \begin{align*}
          \begin{dcases*}
          a_n + b_n = 8^n \\
          a_n - b_n = 2^n
          \end{dcases*}
        \end{align*}
      \end{fleqn}
    この2式から
      \begin{fleqn}[20pt]
        \begin{align*}
          \begin{dcases*}
          a_n = \frac{8^n+2^n}{2} \\
          b_n = \frac{8^n-2^n}{2}
          \end{dcases*}
        \end{align*}
      \end{fleqn}


    \item (1)を踏まえて次の初項,二つの漸化式で与えられる数列$\{p_n\},\, \{q_n\}$の一般項をそれぞれ求めよ。% (大阪医科大学)

        \begin{align*}
          &p_1 = 1,\quad q_1 = 4\\
          &p_{n+1} = 2p_n + q_n \\
          &q_{n+1} = 4p_n - q_n \\
        \end{align*}
      % ヒント:
      % \ $\{p_n + t q_n\}$が等比数列となるような$t$を二つ求める。
      % すなわち,\ $p_{n+1} + tq_{n+1} = s(p_n + t q_n) $を満たす$s,t$の組を二つ見つける。
  \end{enumerate}
% add: 片方消して三項間漸化式に落ち着かせるパターン
\clearpage

  \item
  $n$を自然数,$x_1 = \sqrt{a}$として次の漸化式で与えられる数列$\{x_n\}$を考える。
  \begin{align*}
    x_{n+1} = \sqrt{x_n + a}\quad\cdots \mathrm{(i)}
  \end{align*}
  すなわち,

    \begin{align*}
      x_2 = \sqrt{a + \sqrt{a}}, \qquad x_3 = \sqrt{a + \sqrt{a + \sqrt{a}}}, \qquad\dots
    \end{align*}
  である。
  この数列が収束するかどうかを調べたい。
  次の問いに答えよ。



    \begin{enumerate}[(1)]
    \item 数列$\{x_n\}\:(n\in \mathbb{N})$が収束すると仮定して,その極限値を求めよ。
    \vskip.5\baselineskip
    $x_n \to \alpha(>0)\:(n\to\infty)$とおくと(i)において両辺に$n\to\infty$の極限を考えて

        \begin{fleqn}[20pt]
          \begin{align*}
            \alpha = \sqrt{\alpha + a}\quad\cdots \mathrm{(ii)}
          \end{align*}
        \end{fleqn}
    この両辺を2乗して整理すれば
      \begin{fleqn}[20pt]
        \begin{align*}
          &\alpha^2 - \alpha - a = 0\\
          &\Leftrightarrow \alpha = \frac{1 + \sqrt{1 + 4a}}{2}\quad(\because \alpha>0)
        \end{align*}
      \end{fleqn}

    \vskip1.5\baselineskip
    \item 数列$\{x_n\}\:(n\in \mathbb{N})$が(1)で得た値に実際に収束することを示せ。
    \vskip.5\baselineskip
    まず,(ii)式より

      \begin{fleqn}[20pt]
        \begin{align*}
          &\alpha^2 = a + \alpha\\
          &\Rightarrow a - \alpha^2 = -\alpha \quad\cdots\mathrm{(iii)}
        \end{align*}
      \end{fleqn}
    であるので,
      \begin{fleqn}[20pt]
        \begin{align*}
          |x_n - \alpha|
          &= \left|\frac{x_n^2 - \alpha^2}{x_n + \alpha}\right|\\
          &= \left|\frac{x_{n-1} + a - \alpha^2}{x_n + \alpha}\right|\\
          &= \left|\frac{x_{n-1} - \alpha}{x_n + \alpha}\right|\quad(\because \mathrm{(iii)})\\
          &= \frac{1}{x_n + \alpha} |x_{n-1} - \alpha|\\
          &\leqq \frac{1}{\alpha} |x_{n-1} - \alpha|\quad(\because x_n > 0)\\
          &= \cdots\\
          &= \frac{1}{\alpha^{n-1}}|x_1 - \alpha|
        \end{align*}
      \end{fleqn}
    すなわち,
      \begin{fleqn}[20pt]
        \begin{align*}
          (0 \leqq) |x_n - \alpha| \leqq \frac{1}{\alpha^{n-1}}|x_1 - \alpha|\quad \cdots \mathrm{(iv)}
        \end{align*}
      \end{fleqn}
    ここで,$a>0$より$\alpha$は,
      \begin{fleqn}[20pt]
        \begin{align*}
          \alpha = \frac{1 + \sqrt{4a + 1}}{2} > \frac{1 + 1}{2} = 1
        \end{align*}
      \end{fleqn}
    であるので,(iv)式について辺々に対して$n\to\infty$の極限を考えれば最右辺が
      \begin{fleqn}[20pt]
        \begin{align*}
          \lim_{n\to\infty}\frac{1}{\alpha^{n-1}}|x_1 - \alpha| = 0
        \end{align*}
      \end{fleqn}
    となることから,はさみうちの原理から
      \begin{fleqn}[20pt]
        \begin{align*}
          \lim_{n\to\infty}|x_n - \alpha| = 0
        \end{align*}
      \end{fleqn}
    である。\\
    ゆえに,たしかに数列$\{x_n\}$は(1)で求めた値$\alpha$に収束する。
    \end{enumerate}
    \vskip1\baselineskip
    \noindent 補足\\
    比較的難しい問題ではあるがこのパターンは入試によく出るため覚えておく必要がある。 \\
    一般項が求まりそうにない数列の収束を示すのにははさみうちの原理をもちいることを
    前提とした式変形をしていくのが最もよくある解法である。
    このとき,そもそも極限の収束とは
      \begin{fleqn}[20pt]
        \begin{align*}
          \lim_{x\to a}f(x) = L \Leftrightarrow \lim_{x\to a}|f(x) - L| = 0
        \end{align*}
      \end{fleqn}
    であることを思い出せば,自然と$0\leqq |x_n - \alpha| \leqq L(n)$であり,
    $\lim \!\!\!\!\!\!\!\!\!\!\! \raisebox{-4pt}{$_{n\to\infty}$}L(n) = 0$であるような不等式を見つければよいとわかる。

    すなわち,
    \begin{fleqn}[20pt]
      \begin{align*}
        a_{n+1} = f(a_n)
      \end{align*}
    \end{fleqn}
    のような漸化式の問題での解法は
    \begin{enumerate}[1.]
      \item 極限の予想
      \item 収束の証明
    \end{enumerate}
    の段階を踏めばよいとわかる。\\
    1の極限の予想は比較的簡単で,極限が存在するとき,数列の添字にかかわらずある値に収束することがいえる
    (つまり$\lim \!\!\!\!\!\!\!\!\!\!\! \raisebox{-4pt}{$_{n\to\infty}$}a_n = \lim \!\!\!\!\!\!\!\!\!\!\! \raisebox{-4pt}{$_{n\to\infty}$}a_{n+1}$)
    のでその値を文字で置いて
    漸化式からただの方程式として求めればよい。
    ただし,複数の解をもつ場合はそれが極限の候補に過ぎないため実際にどれに収束するかはグラフや条件から絞る必要がある。 \\
    2は慣れないうちは難しく感じられるかもしれない。
    しかし,目標が
      \begin{fleqn}[20pt]
        \begin{align*}
          |a_{n+1} - \alpha| = r|a_n - \alpha|\: \text{かつ}|r|<1
        \end{align*}
      \end{fleqn}
    となるような関係(漸化不等式という)を導くことであるとわかっていれば,$a_{n+1}$と$a_n$,$\alpha$をつなぐ関係式は
    漸化式とその極限をとった式($\alpha = f(\alpha)$)しかないのだから,それらの和や差をとって
      \begin{fleqn}[20pt]
        \begin{align*}
          &a_{n+1} - \alpha = f(a_n) - f(\alpha)\\
          &\Rightarrow a_{n+1} - \alpha = \text{■}(a_n - \alpha)
        \end{align*}
      \end{fleqn}
    として,$|\text{■}|\leqq r < 1$となるような定数$r$を見つけてやればよいとわかる。
    もちろん,$r = |\text{■}|$としてもよい。
    \vskip1\baselineskip
    \noindent さて,以上を踏まえれば本問題の漸化不等式は次のように求めることもできる。\\
    \noindent (別解)\\
    (i),(ii)の両辺について差を取れば
      \begin{fleqn}[20pt]
        \begin{align*}
          x_{n+1} - \alpha
          &= \sqrt{a + x_n} - \sqrt{a + \alpha}\\
          &= \frac{1}{\sqrt{a + x_n} + \sqrt{a + \alpha}}(x_n - \alpha)
        \end{align*}
      \end{fleqn}

    ここで,$\sqrt{a + x_n}\geqq 0$だから
      \begin{fleqn}[20pt]
        \begin{align*}
          \sqrt{a + x_n} + \sqrt{a + \alpha} \geqq \sqrt{a + \alpha} = \alpha\quad(\because \mathrm{(iii)})
        \end{align*}
      \end{fleqn}
      \begin{fleqn}[20pt]
        \begin{align*}
          \therefore |x_{n+1} - \alpha|\leqq \frac{1}{\alpha}|x_n - \alpha|
        \end{align*}
      \end{fleqn}



\end{enumerate}
\end{document}
