\documentclass[autodetect-engine,ja=standard, 10.5pt, a4paper, titlepage]{bxjsarticle}
% fleqn:数式を左詰めにする(titlepageの前に挿入可能)
% titlepage:表紙を独立させる
%\setlength{\mathindent}{50pt}
%-------------------------------------------------------%


\usepackage{graphicx} % Required for inserting images
\usepackage{titlesec}
\usepackage{caption}
\usepackage{amsmath} % 数式用
\usepackage{amssymb}
\usepackage{amsmath}
\usepackage{enumerate} % 箇条書き
\usepackage{comment} % コメントアウト
\usepackage[super]{cite} % 参考文献 上付き
\usepackage[version=4]{mhchem}
\usepackage{booktabs} % tableのmidrule
\usepackage{multirow} % tableのmultirow
\usepackage{float} % [H]で厳密に位置を固定
\usepackage{nccmath} % 数式を左に動かす
\usepackage{mathtools}
\usepackage{empheq}
\usepackage{accents} % \undertildeで下付きチルダ
\usepackage{nccmath} % 数式左寄せ環境fleqn

%-------------------------------------------------------%

\pagestyle{plain} % empty:ページ番号削除

%-------------------------------------------------------%

% セクション・サブセクションの見出しのサイズ
\titleformat*{\section}{\Large\bfseries} % サイズ・太字
\titleformat*{\subsection}{\large\bfseries}

%-------------------------------------------------------%

\newcommand{\reference}[0]{\setlength{\hangindent}{18pt}\noindent}
\renewcommand{\refeq}[1]{\eqref{#1}式}
\newcommand{\reffig}[1]{図\ref{#1}}
\newcommand{\reftable}[1]{表\ref{#1}}
\newcommand{\degree}[0]{\mathrm{{}^\circ \hspace*{-0.5pt} C}}
\renewcommand{\deg}[0]{\mathrm{{}^\circ}}
\newcommand{\Vector}[1]{{\mbox{\boldmath$#1$}}}
\newcommand{\tensor}[1]{\undertilde{#1}}

\newcommand{\refcite}[2]{\cite{#1}${}^{\text{#2}}$}
\renewcommand{\citeform}[1]{#1)}
\makeatletter % \usepackage以外で@を含むときはこれで囲む
\renewcommand{\@biblabel}[1]{#1)}
\makeatother

\numberwithin{equation}{section} % 式番号にセクションを併記する場合

%**************************************************************
\begin{document}
%\parindent = 0pt % 常に字下げなし
\centerline{\LARGE 数I\!I\!I積分}
\vskip.3cm
\rightline{author\;:\;Yuta\;Suzuki}
\vskip.5cm

  \begin{fleqn}[0pt]
    \begin{align*}
      \:\:1.\quad \int  \:dx
    \end{align*}
  \end{fleqn}
  \begin{fleqn}[0pt]
    \begin{align*}
      \:\:2.\quad \int x^n \:dx
    \end{align*}
  \end{fleqn}
  \begin{fleqn}[0pt]
    \begin{align*}
      \:\:3.\quad \int 2^x \:dx
    \end{align*}
  \end{fleqn}
  \begin{fleqn}[0pt]
    \begin{align*}
      \:\:4.\quad \int \log x \:dx
    \end{align*}
  \end{fleqn}
  \begin{fleqn}[0pt]
    \begin{align*}
      \:\:5.\quad \int x\log x \:dx
    \end{align*}
  \end{fleqn}
  \begin{fleqn}[0pt]
    \begin{align*}
      \:\:6.\quad \int \sin 2x \sin 9x \:dx
    \end{align*}
  \end{fleqn}
  \begin{fleqn}[0pt]
    \begin{align*}
      \:\:7.\quad \int \tan x \:dx
    \end{align*}
  \end{fleqn}
  \begin{fleqn}[0pt]
    \begin{align*}
      \:\:8.\quad \int \frac{dx}{\sin x + 1}
    \end{align*}
  \end{fleqn}
  \begin{fleqn}[0pt]
    \begin{align*}
      \:\:9.\quad \int \frac{dx}{\cos x}
    \end{align*}
  \end{fleqn}
  \begin{fleqn}[0pt]
    \begin{align*}
      10.\quad \int \frac{dx}{x^2 + 2x + 1}
    \end{align*}
  \end{fleqn}
  \begin{fleqn}[0pt]
    \begin{align*}
      11.\quad \int \frac{dx}{x^2 - 4x + 3}
    \end{align*}
  \end{fleqn}
  \begin{fleqn}[0pt]
    \begin{align*}
      12.\quad \int_{-2}^{\sqrt{5}-2} \frac{dx}{x^2 + 4x + 9}
    \end{align*}
  \end{fleqn}
  \begin{fleqn}[0pt]
    \begin{align*}
      13.\quad \int \frac{2x^2 + 12x + 7}{x^2 + 5x + 1} \:dx
    \end{align*}
  \end{fleqn}
  \begin{fleqn}[0pt]
    \begin{align*}
      14.\quad \int \frac{3x + 1}{(x+2)^2} \:dx
    \end{align*}
  \end{fleqn}
  \begin{fleqn}[0pt]
    \begin{align*}
      15.\quad \int_{-\frac{1}{2}}^{\frac{1}{2}} \sqrt{\frac{1-x}{1+x}} \:dx
    \end{align*}
  \end{fleqn}
  \begin{fleqn}[0pt]
    \begin{align*}
      16.\quad \int x^x(1+\log x) \:dx
    \end{align*}
  \end{fleqn}
  \begin{fleqn}[0pt]
    \begin{align*}
      17.\quad \int_{0}^{\frac{\pi}{2}} \frac{\sin x}{\sin x + \cos x} \:dx
    \end{align*}
  \end{fleqn}
  \begin{fleqn}[0pt]
    \begin{align*}
      18.\quad \int \frac{dx}{e^x + 1}
    \end{align*}
  \end{fleqn}
  \begin{fleqn}[0pt]
    \begin{align*}
      19.\quad \int \sqrt{e^x} + 1 \:dx
    \end{align*}
  \end{fleqn}
  \begin{fleqn}[0pt]
    \begin{align*}
      20.\quad \int \frac{dx}{\sqrt{x^2 + 1}}
    \end{align*}
  \end{fleqn}
  \begin{fleqn}[0pt]
    \begin{align*}
      21.\quad \int \sqrt{x^2 + 1} \:dx
    \end{align*}
  \end{fleqn}
  \begin{fleqn}[0pt]
    \begin{align*}
      22.\quad \int \frac{x}{\sqrt{x+1} + 1} \:dx
    \end{align*}
  \end{fleqn}
  \begin{fleqn}[0pt]
    \begin{align*}
      23.\quad \int_{-1}^{1} \frac{x^2}{1 + e^x} \:dx
    \end{align*}
  \end{fleqn}
  \begin{fleqn}[0pt]
    \begin{align*}
      24.\quad \int e^x \sin x \:dx
    \end{align*}
  \end{fleqn}
  \begin{fleqn}[0pt]
    \begin{align*}
      25.\quad \int_{\alpha}^{\beta} (x-\alpha)^n (x-\beta) \: dx
    \end{align*}
  \end{fleqn}
  \begin{fleqn}[0pt]
    \begin{align*}
      26.\quad \int_{0}^{\pi} \frac{x\sin x}{3 + \sin^2 x} \:dx
    \end{align*}
  \end{fleqn}

 
\end{document}
